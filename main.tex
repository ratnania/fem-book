%%%%%%%%%%%%%%%%%%%%%%%%%%%%%%%%%%%%%%%%%
% The Legrand Orange Book
% LaTeX Template
% Version 2.4 (26/09/2018)
%
% This template was downloaded from:
% http://www.LaTeXTemplates.com
%
% Original author:
% Mathias Legrand (legrand.mathias@gmail.com) with modifications by:
% Vel (vel@latextemplates.com)
%
% License:
% CC BY-NC-SA 3.0 (http://creativecommons.org/licenses/by-nc-sa/3.0/)
%
% Compiling this template:
% This template uses biber for its bibliography and makeindex for its index.
% When you first open the template, compile it from the command line with the 
% commands below to make sure your LaTeX distribution is configured correctly:
%
% 1) pdflatex main
% 2) makeindex main.idx -s StyleInd.ist
% 3) biber main
% 4) pdflatex main x 2
%
% After this, when you wish to update the bibliography/index use the appropriate
% command above and make sure to compile with pdflatex several times 
% afterwards to propagate your changes to the document.
%
% This template also uses a number of packages which may need to be
% updated to the newest versions for the template to compile. It is strongly
% recommended you update your LaTeX distribution if you have any
% compilation errors.
%
% Important note:
% Chapter heading images should have a 2:1 width:height ratio,
% e.g. 920px width and 460px height.
%
%%%%%%%%%%%%%%%%%%%%%%%%%%%%%%%%%%%%%%%%%

%----------------------------------------------------------------------------------------
%	PACKAGES AND OTHER DOCUMENT CONFIGURATIONS
%----------------------------------------------------------------------------------------

\documentclass[11pt,fleqn]{book} % Default font size and left-justified equations

\input{structure.tex} % Insert the commands.tex file which contains the majority of the structure behind the template
\usepackage{subcaption}
\usepackage{float}
\hypersetup{pdftitle={Variational Formulations and Finite Elements Method: Theory and Practice},pdfauthor={AL-KHWARIZMI}} % Uncomment and fill out to include PDF metadata for the author and title of the book

%----------------------------------------------------------------------------------------

%----------------------------------------------------------------------------------------
%	USEFUL TEXT VARIABLES
%----------------------------------------------------------------------------------------

\newcommand{\repourl}{https://gitlab.com/bvermeir/book-cfd}
\newcommand{\binderurl}{https://tinyurl.com/cfd-binder}

%----------------------------------------------------------------------------------------
%	USEFUL commands 
%----------------------------------------------------------------------------------------
\usepackage[ruled,vlined,linesnumbered]{algorithm2e}
\usepackage{bm}
\usepackage{tcolorbox} % box that includes text + math formula

\newtheorem{lemma}{Lemma}
\newcommand{\todo}[1]{\textcolor{red}{#1}}

\newcommand{\nn}{\bm{n}}
\newcommand{\ww}{\bm{w}}
\newcommand{\ee}{\bm{e}}
\newcommand{\ff}{\bm{f}}
\newcommand{\uu}{\bm{u}}
\newcommand{\vv}{\bm{v}}
\newcommand{\ii}{\bm{i}}
\newcommand{\jj}{\bm{j}}
\newcommand{\pp}{\bm{p}}
\newcommand{\kk}{\bm{k}}
\newcommand{\xx}{\bm{x}}
\newcommand{\UU}{\bm{U}}
\newcommand{\EE}{\bm{E}}
\newcommand{\FF}{\bm{F}}
\newcommand{\HH}{\bm{H}}
\newcommand{\GG}{\bm{G}}
\newcommand{\JJ}{\bm{J}}
\newcommand{\DD}{\bm{D}}
\newcommand{\BB}{\bm{B}}
\newcommand{\dotphi}{\dot{\phi}}
\newcommand{\PsiPsi}{\boldsymbol{\Psi}}
\newcommand{\etaeta}{\boldsymbol{\eta}}
\newcommand{\ttheta}{\bm{\theta}}
\newcommand{\mumu}{\boldsymbol{\mu}}
\newcommand{\xixi}{\boldsymbol{\xi}}
\newcommand{\intline}{\int_{\mathbb{R}}}
\newcommand{\fracp}[2]{\frac{\partial #1}{\partial #2}}
\newcommand{\dd}{\,{\rm d} }
\newcommand{\diff}{\mathrm{d}\,}
\newcommand{\Rotv}{\boldsymbol{\nabla} \times}
\newcommand{\Rots}{\nabla \times}
\newcommand{\Curl}{\nabla \times}
\newcommand{\Div}{\nabla \cdot}
\newcommand{\Grad}{\boldsymbol{\nabla}}
\newcommand{\Gradh}{\mathbb{G}}
\newcommand{\Curlvh}{\pmb{\mathbb{C}}}
\newcommand{\Curlh}{\mathbb{C}}
\newcommand{\Divh}{\mathbb{D}}
\newcommand{\Hgrad}{H^1(\Omega)}
\newcommand{\Hgradv}{\bm{H}^1(\Omega)}
\newcommand{\Hcurl}{\bm{H}(\mbox{curl}, \Omega)}
\newcommand{\Hdiv}{\bm{H}(\mbox{div}, \Omega)}
\newcommand{\Ltwo}{L^2(\Omega)}
\newcommand{\Ltwov}{\bm{L}^2(\Omega)}
\newcommand{\Hgradzero}{H^1_0(\Omega)}
\newcommand{\Hgradvzero}{\bm{H}^1_0(\Omega)}
\newcommand{\Hcurlzero}{\bm{H}_0(\mbox{curl}, \Omega)}
\newcommand{\Hdivzero}{\bm{H}_0(\mbox{div}, \Omega)}
\newcommand{\Ltwozero}{L^2_0(\Omega)}
\newcommand{\Ltwovzero}{\bm{L}^2_0(\Omega)}
\newcommand{\Vgrad}{V_h(\mbox{grad}, \Omega)}
\newcommand{\Vgradv}{\bm{V}_h(\mbox{grad}, \Omega)}
\newcommand{\Vcurl}{\bm{V}_h(\mbox{curl}, \Omega)}
\newcommand{\Vdiv}{\bm{V}_h(\mbox{div}, \Omega)}
\newcommand{\Vltwo}{V_h(L^2, \Omega)}
\newcommand{\Cinfinity}{\mathcal{C}^{\infty}(\Omega)}
\newcommand{\HgradLogical}{H^1(\hat{\Omega})}
\newcommand{\HcurlLogical}{{H}(\mbox{curl}, \hat{\Omega})}
\newcommand{\HdivLogical}{{H}(\mbox{div}, \hat{\Omega})}
\newcommand{\LtwoLogical}{L^2(\hat{\Omega})}
\newcommand{\VgradLogical}{V_h(\mbox{grad}, \hat{\Omega})}
\newcommand{\VcurlLogical}{{V}_h(\mbox{curl}, \hat{\Omega})}
\newcommand{\VdivLogical}{{V}_h(\mbox{div}, \hat{\Omega})}
\newcommand{\VltwoLogical}{V_h(L^2, \hat{\Omega})}
\newcommand{\igrad}{\imath^{0}}
\newcommand{\icurl}{\imath^{1}}
\newcommand{\idiv}{\imath^{2}}
\newcommand{\iltwo}{\imath^{3}}
\newcommand{\Pigrad}{P_h^{\mbox{\footnotesize{grad}}}}
\newcommand{\Picurl}{P_h^{\mbox{\footnotesize{curl}}}}
\newcommand{\Pidiv}{P_h^{\mbox{\footnotesize{div}}}}
\newcommand{\Piltwo}{P_h^{\footnotesize{L^2}}}
\newcommand{\Ker}[1]{\mbox{Ker}~ #1}
\newcommand{\Nip}{{N_{\mumu,}^{\pp}}_{\ii}}
\newcommand{\Njp}{{N_{\mumu,}^{\pp}}_{\jj}}
\newcommand{\Nione}{{N_{\mumu_1,}^{p_1}}_{i_1}}
\newcommand{\Njone}{{N_{\mumu_1,}^{p_1}}_{i_1}}
\newcommand{\Nitwo}{{N_{\mumu_2,}^{p_2}}_{i_2}}
\newcommand{\Njtwo}{{N_{\mumu_2,}^{p_2}}_{i_2}}
\newcommand{\Nitre}{{N_{\mumu_3,}^{p_3}}_{i_3}}
\newcommand{\Njtre}{{N_{\mumu_3,}^{p_3}}_{i_3}}
\newcommand{\Vgradspline}{\mathcal{S}^{p,p,p}}
\newcommand{\Vltwospline}{\mathcal{S}^{p-1,p-1,p-1}}
\newcommand{\Vcurlspline}{
  \begin{pmatrix}
    \mathcal{S}^{p-1,p,p} \\ 
    \mathcal{S}^{p,p-1,p} \\ 
    \mathcal{S}^{p,p,p-1}
  \end{pmatrix}
}
\newcommand{\Vdivspline}{
  \begin{pmatrix}
    \mathcal{S}^{p,p-1,p-1} \\ 
    \mathcal{S}^{p-1,p,p-1} \\ 
    \mathcal{S}^{p-1,p-1,p}
  \end{pmatrix}
}

\newcommand{\Mione}{{M_{\mumu_1-1,}^{p_1-1}}_{i_1}}
\newcommand{\Mjone}{{M_{\mumu_1-1,}^{p_1-1}}_{i_1}}
\newcommand{\Mitwo}{{M_{\mumu_2-1,}^{p_2-1}}_{i_2}}
\newcommand{\Mjtwo}{{M_{\mumu_2-1,}^{p_2-1}}_{i_2}}
\newcommand{\Mitre}{{M_{\mumu_3-1,}^{p_3-1}}_{i_3}}
\newcommand{\Mjtre}{{M_{\mumu_3-1,}^{p_3-1}}_{i_3}}
\newcommand{\matrixIdentity}{\mathbb{I}}

\newcommand{\glteq}{\sim_{\rm GLT}}
\newcommand{\gltgrad}{\boldsymbol{\delta [\mathfrak{m}_{p}]}}
\newcommand{\gltmm}{\boldsymbol{\mathfrak{m}}_{\mathbf{p}}}
\newcommand{\TrialS}{\mathcal{T}(\mathcal{S}_n^p)}
\newcommand{\TestS}{\mathcal{T}^{\prime}(\mathcal{S}_n^p)}

%%%%%%%%%%%%%%%%%%%%%%%%%%%% Maths Operators 
\newcommand{\abs}[2][]{\mathopen#1\lvert #2 \mathclose#1\rvert}
\newcommand{\norme}[1]{\|#1\|}
\newcommand{\intf}[2]{\left[#1,#2\right]}
\newcommand{\into}[2]{\left]#1,#2\right[}
\newcommand{\intr}{\operatorname{int}}
\newcommand\fr[1]{\operatorname{fr}\!\left(#1\right)}
%%%%%%%%%%%%%%%%%%%%%%%%%%%%%%%

\newcommand{\dis}{\displaystyle}

\newcommand{\rot}{\overrightarrow{\mbox{rot}}}
\newcommand{\grad}{ \overrightarrow{\mbox{grad}}}
\newcommand{\calC}{\mathcal{C}}
\newcommand{\calF}{\mathcal{F}}
\newcommand{\cN}{\mathcal N}
\newcommand{\calO}{\mathcal{O}}
\newcommand{\R}{\mathbb R}
\newcommand{\C}{\mathbb C}
\newcommand{\CC}{\mathbb C}
\newcommand{\N}{\mathbb N}
\newcommand{\Z}{\mathbb Z}
\newcommand{\Q}{\mathbb Q}
\newcommand{\KK}{\mathbb K}
\newcommand{\dx}{\,{\mathrm d}x}
\newcommand{\dy}{\,{\mathrm d}y}
\newcommand{\du}{\,{\mathrm d}u}
\newcommand{\dv}{\,{\mathrm d}v}
\newcommand{\dw}{\,{\mathrm d}w}
\newcommand{\dt}{\,{\mathrm d}t}
\newcommand{\dr}{\,{\mathrm d}r}
\newcommand{\df}{\,{\mathrm d}f}
\newcommand{\dl}{\,{\mathrm d}l}
\newcommand{\ds}{\,{\mathrm d}s}
\newcommand{\dz}{\,{\mathrm d}z}
\newcommand{\dg}{\,{\mathrm d}g}
\newcommand{\dA}{\,{\mathrm d}A}
\newcommand{\Nabla}{\overrightarrow{\nabla}}
\newcommand{\et}{\  \mbox{et} \ }
\newcommand{\tq}{\  \mbox{tel que} \ }
\newcommand{\V}{\overrightarrow{V}}
\newcommand{\W}{\overrightarrow{W}}
\newcommand{\U}{\overrightarrow{U}}
\newcommand{\I}{\overrightarrow{i}}
\newcommand{\J}{\overrightarrow{j}}
\newcommand{\K}{\overrightarrow{k}}
\newcommand{\oij}{(O, \overrightarrow{i}, \overrightarrow{j})}
\newcommand{\oijk}{(O, \overrightarrow{i}, \overrightarrow{j}, \overrightarrow{k})}
\newcommand{\veps}{\varepsilon}


\begin{document}

%----------------------------------------------------------------------------------------
%	TITLE PAGE
%----------------------------------------------------------------------------------------

\begingroup
\thispagestyle{empty} % Suppress headers and footers on the title page
\begin{tikzpicture}[remember picture,overlay]
\node[inner sep=0pt] (background) at (current page.center) {\includegraphics[width=\paperwidth]{titlepage-fem.pdf}};
\end{tikzpicture}
\vfill
\endgroup

%%----------------------------------------------------------------------------------------
%%	HALF TITLE PAGE
%%----------------------------------------------------------------------------------------
%
%\begingroup
%\newpage
%\thispagestyle{empty} % Suppress headers and footers on the title page
%\begin{tikzpicture}[remember picture,overlay]
%\node[inner sep=0pt] (background) at (current page.center) {\includegraphics[width=\paperwidth]{halftitle.pdf}};
%\end{tikzpicture}
%\vfill
%\endgroup


%----------------------------------------------------------------------------------------
%	COPYRIGHT PAGE
%----------------------------------------------------------------------------------------

\newpage
~\vfill
\thispagestyle{empty}

\noindent Copyright \copyright\ 2021 AL-KHWARIZMI\\ % Copyright notice

\noindent \textsc{Published by UM6P University}\\ % Publisher

\noindent \textsc{um6p.ma}\\ % URL

\noindent Licensed under the Creative Commons Attribution-NonCommercial 3.0 Unported License (the ``License''). You may not use this file except in compliance with the License. You may obtain a copy of the License at \url{http://creativecommons.org/licenses/by-nc/3.0}. Unless required by applicable law or agreed to in writing, software distributed under the License is distributed on an \textsc{``as is'' basis, without warranties or conditions of any kind}, either express or implied. See the License for the specific language governing permissions and limitations under the License.\\ % License information, replace this with your own license (if any)

\noindent \textit{First printing, October 2021} % Printing/edition date

%----------------------------------------------------------------------------------------
%	FOREWORD
%----------------------------------------------------------------------------------------
\clearpage
\section*{Foreword}\index{Foreword}
TODO
%\vspace{0.5cm}
%
\noindent Abdeladim El-Akri 
\noindent Lahcen Maniar 
\noindent Ahmed Ratnani 



%----------------------------------------------------------------------------------------
%	TABLE OF CONTENTS
%----------------------------------------------------------------------------------------

\usechapterimagefalse % If you don't want to include a chapter image, use this to toggle images off - it can be enabled later with \usechapterimagetrue

\chapterimage{chapter_head_1.pdf} % Table of contents heading image

\pagestyle{empty} % Disable headers and footers for the following pages

\tableofcontents % Print the table of contents itself

\cleardoublepage % Forces the first chapter to start on an odd page so it's on the right side of the book

\pagestyle{fancy} % Enable headers and footers again

%----------------------------------------------------------------------------------------
%	PART 1: Functional Analysis 
%----------------------------------------------------------------------------------------
\part{Functional Analysis}
\chapter{Topology}
Dans ce chapitre, on s'intéresse à l'approximation numérique d'équations aux dérivées partielles linéaires qui admettent une formulation variationnelle. 

Exemple :  Considérons le problème suivant. 

Étant donné $f \in \mathrm{C}([a, b])$ trouver une fonction $u$ vérifiant
$$
(1)\;  \left\{\begin{array}{l}
	-u^{\prime \prime}+u=f \quad \text { sur } \quad]a, b[ \\
	u(a)=u(b)=0.
\end{array}\right.
$$
Une solution classique $-$ ou solution forte $-$ du problème (1) est une fonction de classe $\mathrm{C}^{2}$ sur $[a, b]$ vérifiant (1) au sens usuel. Bien entendu (1) peut être résolu explicitement par un calcul très simple, mais nous ignorerons cet aspect des choses afin d'illustrer la méthode sur cet exemple élémentaire.
On multiplie (1) par $\varphi \in C^{1}([a, b])$ et on intègre par parties; il vient
$$(2) \; \int_{a}^{b} u^{\prime} \varphi^{\prime}+\int_{a}^{b} u \varphi=\int_{a}^{b} f \varphi,    \qquad \forall \varphi \in \mathrm{C}^{1}([a, b]), \quad \varphi(a)=\varphi(b)=0.
$$
On notera que (2) a un sens dès que $u \in \mathrm{C}^{1}([a, b])$ (contrairement à (1) qui suppose $u$ deux fois dérivable); en fait il suffirait même d'avoir $u, u^{\prime} \in \mathrm{L}^{1}(a, b), u^{\prime}$, dérivée généralisée, ou dérivée au  un sens faible, voir plus loin. 

 Disons (provisoirement) qu'une fonction $u$ de classe $\mathrm{C}^{1}$ qui vérifie (2) est une solution faible de (1).

Le programme suivant décrit les grandes lignes de l'approche variationnelle en théorie des équations aux dérivées partielles :

Etape A. - On précise la notion de solution faible ; celle-ci fait intervenir les espaces de Sobolev qui sont les outils de base.

Etape B. - On établit l'existence et l'unicité d'une solution faible par la méthode variationnelle, via le théorème de Lax-Milgram.

Etape C. - On prouve quc la solution faible est de classe $\mathrm{C}^{2}$ (par exemple): c'est un résultat de régularité.


Etape D. - Retour aux solutions classiques. On montre qu'une solution faible de classe $\mathrm{C}^{2}$ est une solution classique.


L'équation (2),  peut être écrite sous la forme du problème "abstrait" général suivant :

$$
(PV) \text{ trouver } u \in W\;  \text{tel que } a(u, v)=L(v) \;\, \text{pour tout}\;\;  v \in V.
$$

 $W$ et  $V$ sont des espaces vectoriels normés. Dans plusieurs applications, $W$ and $V$ sont des espaces de Hilbert, mais  des cas où $V, W$ sont des  espace de  Banach  réflexifs  peuvent être considérés, $a:W\times V\longrightarrow \mathbb{R}$ une application bilinéaire continue
et $L$ une forme linéaire continue sur $V$, i.e. $L\in V'$. 

Ces formulations sont importantes pour les raisons suivantes :

1. De nombreux problèmes issus de la physique et de la mécanique admettent de telles formulations, et celles-ci reflètent souvent une propriété fondamentale du modèle, typiquement la minimisation d'une énergie sous-jacente.

2. Ces formulations donnent accès à des résultats fondamentaux sur le caractère bien posé de l'équation, c'est à dire l'existence et l'unicité de la solution, et la stabilité de cette solution par rapport à des perturbations des données.

3. Elles sont à la base de méthodes performantes pour l'approximation numérique des solutions, par la résolution d'un problème approché : 
\bigskip

\centerline{
trouver $u_{h} \in X_{h}$ tel que $a\left(u_{h}, v_{h}\right)=L\left(v_{h}\right)$ pour tout $v_{h} \in X_{h}$,} 

où $X_{h}$ est un sous-espace de dimension finie de $X$.

Le cours se concentre autour de ce dernier aspect qui pose la question du contrôle de l'erreur $u-u_{h}$ entre la solution exacte et la solution approchée. On s'interessera tout particulièrement à la méthode des éléments finis dans laquelle les fonctions de $X_{h}$ sont polynomiales par morceaux sur une partition du domaine de la solution de l'équation.


On définit quelques outils mathématiques nécessaires pour développer cette étude.

\section{Espaces Complets}


\subsection{ Normes et produits scalaires}
Soit $E$ un espace vectoriel.
\begin{definition}\
	
	$\|\cdot\|: E \rightarrow \mathbb{R}_{+}$ est une norme sur $E$ s'elle vérifie:
	
(N1) $\|x\|=0\Longrightarrow x=0$.

(N2) $\forall \lambda \in \mathbb{R}, \forall x \in E, \quad\|\lambda x\|=|\lambda|\|x\|$.

(N3) $\forall x, y \in E, \quad\|x+y\| \leq\|x\|+\|y\|$
(inégalité triangulaire).


- Un espace vectoriel muni d'une norme est appelé espace normé ou un evn.


\end{definition}
Exemples : 

1. Pour $E=\mathbb{R}^{n}$ et $x=\left(x_{1}, \ldots, x_{n}\right) \in \mathbb{R}^{n}$, on définit les normes
$$
\|x\|_{1}=\sum_{i=1}^{n}\left|x_{i}\right|,  \quad\|x\|_{2}=\left(\sum_{i=1}^{n} x_{i}^{2}\right)^{1 / 2},    \quad\|x\|_{\infty}=\sup _{i}\left|x_{i}\right|.
$$

2. Sur $E=\mathcal{C}([0,1], \R)$, le $\R$-espace vectoriel des fonction réelles continues sur $[0,1]$,  l'application 
$$
f\longmapsto\|f\|=\sup_{t\in[0,1]}|f(t)|
$$
est une  norme.



\begin{definition}\
	
On appelle produit scalaire sur $E$ toute forme bilinéaire symétrique définie positive : 

1. \; L'application $<\cdot, \cdot >: E \times E \rightarrow \mathbb{R}$ est donc un produit scalaire sur $E$ s'elle   vérifie :

(S1) $\forall x, y \in E, \quad\langle x, y\rangle =\langle y, x\rangle $.

(S2) $\forall x_{1}, x_{2}, y \in E, \quad\langle x_{1}+x_{2}, y\rangle =\langle x_{1}, y\rangle +\langle x_{2}, y\rangle $.

(S3) $\forall x, y \in E, \forall \lambda \in \mathbb{R}, \quad\langle \lambda x, y\rangle =\lambda\langle x, y\rangle $.

(S4) $\forall x \in E, x \neq 0, \quad\langle x, x\rangle >0$.


2.\; Un espace vectoriel $E$ muni d'un produit scalaire est appelé espace préhilbertien. 
\end{definition}

A partir d'un produit scalaire, on peut définir une norme induite: $\|x\|=\sqrt{\langle x, x\rangle}$.

 On a alors, d'après (N3), l'inégalité de Cauchy-Schwarz: $|\langle x, y\rangle | \leq\|x\|\|y\|$.

Exemple: 

Pour $E=\mathbb{R}^{n}$, on définit le produit scalaire $\langle x, y\rangle =\sum_{i=1}^{n} x_{i} y_{i} .$ 

Sa norme induite est $\|\cdot \|_{2}$ définie précédemment.

En général, si $E$ et $F$ sont deux evn, on peut définir l'espace produit 

$E\times F=\{ (x,y): x\in E, y\in F\}$, qui est un evn pour les normes 
$$
\|(x,y) \|_{1}= \|x \|_E+\|y \|_F,\quad  \|(x,y) \|_{2}= \sqrt{\|x \|_E^2+\|y \|_F^2}, \quad \|(x,y) \|_{\infty}=max(\|x \|_E,\|y \|_F). 
$$ 

Si $E$ et $F$ sont deux Hilbert, l'espace produit $E\times F$ est un Hilbert. 
\subsection{Suites de Cauchy : espaces et complets}

\begin{definition} \
	
1. 	\; Soit $E$ un espace vectoriel et $\left(x_{n}\right)_{n}$ une suite de $E $. $\left(x_{n}\right)_{n}$ est une suite de Cauchy ssi 
$$
\forall \varepsilon>0, \exists N >0 : \forall p>N, \forall q>N, \quad\left\|x_{p}-x_{q}\right\|<\varepsilon.
$$
	

2. \; Un espace vectoriel est complet si toute suite de Cauchy y est convergente.

3. \;  Un espace normé complet est un espace de Banach.

4. \; Un espace préhilbertien complet est un espace de Hilbert.

5. \;  Un espace de Hilbert de dimension finie est appelé espace euclidien.

\end{definition}


\subsection{Espaces fonctionnels}
\begin{definition}
	
Un espace fonctionnel est un espace vectoriel dont les éléments sont des fonctions.

\end{definition} 


Exemples :

1. L'espace des fonctions continues sur un intervale $[a, b]$ à valeurs réelles, noté $\mathcal{C}^{0}([a ; b])$ ou juste $\mathcal{C}([a ; b])$.   $\mathcal{C}([a ; b])$ muni de la norme $\|f\|_{\infty}= \sup_{[a, b]} |f(x)|$ est un espace complet (Banach).

2.  $\mathcal{C}^{p}([a ; b])$ désigne l'espace des fonctions définies sur l'intervalle $[a, b]$ à valeurs dans $\mathbb{R}$, dont toutes les dérivées jusqu'à l'ordre $p$ existent et sont continues sur $[a, b]$.

Dans la suite, les fonctions seront définies sur un sous-ensemble de $\mathbb{R}^{n}$ (le plus souvent un ouvert noté $\Omega$ ), à valeurs dans $\mathbb{R}$ ou $\mathbb{R}^{m}$.

Exemple: La température $T(x, y, z, t)$ en tout point d'un objet $\Omega \subset \mathbb{R}^{3}$ est une fonction de $\Omega \times \mathbb{R} \longrightarrow \mathbb{R}$.

Les normes usuelles les plus simples sur les espaces fonctionnels sont les normes $L^p$  définies par :
$$
\|u\|_{L^p}=\left(\int_{\Omega}|u(x)|^{p}\right)^{1 / p},  \quad p \in[1,+\infty[, \quad \text{ et } \quad\|u\|_{L^{\infty}}= \sup_{\Omega} |u(x)|. 
$$

Ces applications  ne sont pas nécessairement des normes; et lorsqu'elles le sont, les espaces fonctionnels munis de ces normes ne sont pas nécessairement des espaces de Banach. Par exemple, les applications  $\|\cdot \|_{L^1}, \|\cdot \|_{L^{\infty}}$  sont bien des normes sur l'espace $\mathcal{C}^{0}([a ; b])$, et cet espace est complet si on le munit de la norme $\|\cdot \|_{L^{\infty}}$, mais ne l'est pas si on le munit de la norme $\|\cdot \|_{L^1}$.

Pour cette raison, on va définir les espaces $\mathcal{L}^{p}(\Omega)(p \in[1,+\infty[)$ par
$$
\mathcal{L}^{p}(\Omega)=\left\{u: \Omega \rightarrow \mathbb{R}, \text { mesurable, et telle que } \int_{\Omega}|u|^{p}<\infty\right\}.
$$
(on rappelle qu'une fonction $u$ est mesurable ssi $\{x \in \Omega /|u(x)|<r\}$ est mesurable dans $\mathbb{R}^n$,  pour tout $ r>0$. )

Sur ces espaces $\mathcal{L}^{p}(\Omega)$, les applications  $\|\cdot \|_{L^p}$ ne sont pas des normes. En effet, 

$\|u\|_{L^{p}}=0$ implique que $u$ est nulle presque partout dans $\mathcal{L}^{p}(\Omega)$, et non pas $u=0$. 

C'est pourquoi on va définir les espaces $L^p(\Omega)$ :

\begin{definition}\
	
L'espace  $L^{p}(\Omega), p\geq 1,$ est l'ensemble des classes d'équivalence des fonctions de $\mathcal{L}^{p}(\Omega)$ pour la relation d'équivalence "égalité presque partout". Autrement dit, on confondra deux fonctions dès lors qu'elles sont égales presque partout, c'est à dire qu'elles ne different que sur un ensemble de mesure nulle.

\end{definition}

\begin{theorem}\
	
L'application  $\|\cdot \|_{L^p}$ est une norme sur $L^{p}(\Omega)$, et $L^{p}(\Omega)$ muni de cette  norme  est un espace de Banach.

Pour  $p=2$,   l'espace fonctionnel $L^{2}(\Omega)$,  des fonctions de carré sommable sur $\Omega$, la norme $\|\cdot \|_{L^{2}}$, est la norme induite  du produit scalaire $$<u, v>_{L^{2}}=\int_{\Omega} u v dx,
$$

 et donc   $L^{2}(\Omega)$ est un espace de Hilbert.
 
 On défint aussi l'espace $L^\infty(\Omega)$ par 
 
 $$
 \mathrm{L}^{\infty}(\Omega)=\{f: \quad \Omega \rightarrow \mathbb{R} ; f \; \text{mesurable et } \, \exists  \mathrm{C}>0 :  |f(x)| \leq \mathrm{C} \; \text{ p.p. sur } \Omega\}
 $$
 
 L'application 
 $$
 \|f\|_{L^{\infty}}= \sup_{\Omega} |f(x)|
 $$

est une norme sur $L^\infty(\Omega)$ et  $(L^\infty(\Omega), \|\cdot \|_{L^{\infty}})$ est un espace de Banach.

\end{theorem}

\section{Dual Topologique}

Soit $V$ un espace de Banach (un $\mathbb{K}$-espace vectoriel),  muni d'une norme $\|\cdot \|_V$.  Le dual topologique de $V$ est l'ensemble des applications linéaires continues de $V$ dans $\mathbb{K}$,  noté $V'$. Donc, $V'=\mathcal{L}_c(V,\mathbb{K}),, \mathcal{L}(V,\mathbb{K})$ est un espace de Banach pour la norme 
$$
\displaystyle\|f \|=\sup_{v\in V,v\neq0}\frac{|f(v) |}{\|v\|_V} \quad = \sup_{v\in V,\|v\|_V\leq 1}|f(v) |, 
$$  

pour $f\in V'$.  Les éléments de $V'$ sont dites des formes linéaires de $V$. 

Pour un espace de Hilbert $H$, on a $H'=H$. On identifie $H$ et son dual topologique 

(via une isométrie, Théorème de représentation de  Riesz). 

Un espcace de Banach $X$, est dit réflexif si $X^{\prime\prime}=X$.  

Exemple :  Soit $p>1$, l'espace dual de $L^p(\Omega)$ est  l'espace $L^q(\Omega)$, avec $\frac1p+\frac1q=1$.  

En particulier, pour $p=2$, $q=2$ ($\frac12+\frac12=1$) et  donc $(L^2(\Omega))^\prime=L^2(\Omega)$, 

c'est normal puisque c'est un Hilbert.

Pour $p=1$,  $(L^1(\Omega))^\prime=L^\infty(\Omega)$. 

Pour $p>1$, l'espace $L^p(\Omega)$ est un Banach réflexif.   En effet ...

\begin{remark}\
	
1. Si $V$ et $H$ sont deux espaces de Hilbert tels que $V\subset H$.   On identifie $H$ avec son dual $H'$, mais pas $V$ et on a 
$$
	V\subset H=H'\subset V',
	$$
	
	car, en général,  si $V$ et $W$  sont deux Banach tel que $V\subset W$ alors $W'\subset V'$.  
	
	On dit que $V'$ est le dual de $V$ par rapport au pivot $H$.


Exemple (exercice):	 Soient les epaces 
$$
\begin{aligned}
	&H=l^{2}=\left\{u=\left(u_{n}\right) ; \sum u_{n}^{2}<\infty\right\}  \text { muni du produit scalaire }\langle u, v\rangle=\sum u_{n} v_{n} \\
	&V=\left\{u=\left(u_{n}\right) ; \Sigma n^{2} u_{n}^{2}<\infty\right\} \text { muni du produit scalaire }\langle \langle u, v\rangle\rangle=\sum n^{2} u_{n} v_{n}
\end{aligned}
$$

Calculer $V'$. 


2.  Pour un $\phi\in V'$,   et $v\in V$, on note $\phi(v)=\langle \phi, v\rangle_{V^{\prime}, V}$, qu'on appelle  le crochet de dualité entre $V$ et $V'$. Ce crochet est identifié au produit scalaire dans le cas d'un Hiblbert $H$  tel que $H=H'$.

\end{remark} 

\section{Problems}

\begin{exercise}
  TODO
\end{exercise}


\chapter{Introduction to Functional Analysis}
\section{Dérivée généralisée et Espaces de Sobolev}
Nous venons de définir des espaces fonctionnels complets, ce qui sera un bon cadre pour démontrer l'existence et l'unicité de solutions d'équations aux dérivées partielles, comme on le verra plus loin notamment avec le théorème de Lax-Milgram. Toutefois, on a vu que les éléments de ces espaces $L^{p}$ ne sont pas nécessairement des fonctions très régulières. Dès lors, les dérivées partielles de telles fonctions ne sont pas forcément définies partout. Pour s'affranchir de ce problème, on va étendre la notion de dérivation à  la notion  de dérivée généralisée.  Ceci  pertmettra d'introduire de nouveux espaces fonctionnels, sous espaces des $L^p$, analogue aux espaces $C^p(\Omega)$.  

Dans la suite, $\Omega$ sera un ouvert (pas nécessairement borné) de $\mathbb{R}^{n}$.

\subsection{Fonctions tests} 

\begin{definition}\
	
	
	Soit $\varphi: \Omega \rightarrow \mathbb{R}$. On appelle support de $\varphi$ l'adhérence de $\{x \in \Omega :  \varphi(x) \neq 0\}$. 
	
	On le note $supp(\varphi)$. 
	\end{definition}
Exemple : Pour $\Omega=]-1,1[$, et $\varphi$ la fonction constante égale à 1, $supp(\varphi)=[-1,1]$.

\begin{definition}\
	
On note $\mathcal{D}(\Omega)$ l'espace des fonctions de $\Omega$ vers $\mathbb{R}$, de classe $\mathcal{C}^{\infty}$, et à support
compact inclus dans $\Omega.$  $ \mathcal{D}(\Omega)$ est parfois appelé espace des fonctions-tests.

\end{definition}

Exemple: L'exemple le plus classique dans le cas de $\mathbb{R}$ est la fonction.
$$
\varphi(x)= \begin{cases}e^{-\frac{1}{1-x^{2}}} & \text { si }|x|<1 \\ 0 & \text { si }|x| \geq 1\end{cases}
$$
$\varphi$ est une fonction de $\mathcal{D}( ]-1, 1[ )$  et $supp(\varphi)=[-1,1]$.

\begin{theorem}\
	
$\overline{\mathcal{D}(\Omega)}=L^{p}(\Omega)$, $1\leq p\leq \infty$,  i.e. $\mathcal{D}(\Omega)$ est dense dans $L^{p}(\Omega)$ : pour tout $f\in L^{p}(\Omega)$, il existe une suite $(f_n)\subset \mathcal{D}(\Omega)$ convergente vers $f$.
\end{theorem}


\subsection{Dérivée généralisée}
On définit une notion de dérivée pour des fonctions qui ne sont pas  nécessairement de classe $\mathcal{C}^{1}$. 

\begin{definition}\ 
	
Soit $I$ un intervalle de $\mathbb{R}$, pas forcément borné. On dit que $u \in L^{2}(I)$ admet une dérivée généralisée dans $L^{2}(I)$ si 
 
 $$\exists u_{1} \in L^{2}(I) :  \forall \varphi \in \mathcal{D}(I), \quad \int_{I} u \varphi^{\prime}=-\int_{I} u_{1} \varphi .
 $$ 
 
La fonction $u_1$ est unique dans  $ L^{2}(I)$, et elle est dite la dérivée généralisée, dérivée au sens des distributions, ou dérivée au sens faible, de $u$. On la note aussi $u_1=u'$.
 
 En itérant, on dit que $u$ admet une dérivée généralisée d'ordre $k$ dans
 $L^{2}(I)$, notée $u_{k}$, si $$\forall \varphi \in \mathcal{D}(I), \quad \int_{I} u_{k} \varphi=(-1)^{k} \int_{I} u \varphi^{(k)}.
 $$
 
 On note aussi $u_k=u^{(k)}$, la $k$-ème dérivée de $u$. 
 
 Ces définitions s'étendent naturellement pour la définition de dérivés partielles généralisées du premier ordre, notées $\partial_{i}, \partial_{x_i}, $ ou $\frac{\partial}{\partial x_i}$, $ i=1, \dots, n$,  dans le cas multidimension (dans $\mathbb{R}^n, n>1$) et aussi pour les dérivés partielles généralisées d'ordre supérieur ($m\geq 2$), notées $\partial^\alpha$, ou $\frac{\partial^{\alpha_{1}}}{\partial x_{1}^{\alpha_{1}}} \frac{\partial^{\alpha_{2}}}{\partial x_{2}^{\alpha_{2}}} \cdots \frac{\partial^{\alpha_{N}}}{\partial x_{n}^{\alpha_{n}}}$, pour $\alpha =(\alpha_1, \cdots, \alpha_n)$ tel que $|\alpha|=\alpha_{1}+\cdots+\alpha_{n} =m$. 
 
\end{definition}

Exemple : 

Soit $I=] a, b[$ un intervalle borné, et $c$ un point de $I$. On considère une fonction $u$ formée de deux branches de classe $\mathcal{C}^{1}$,  l'une sur $] a, c[$,  l'autre sur 
$] c, b[$ et se raccordant de  façon continue mais non dérivable en $c$. Alors $u$ admet une dérivée généralisée définie par $u_{1}(x)=u^{\prime}(x),   \quad \forall x \neq c$. 
 En effet :
\begin{align*}
	\forall \varphi \in \mathcal{D}(]a, b[),  \quad \int_{a}^{b} u \varphi^{\prime}&=\int_{a}^{c}u \varphi^{\prime}+\int_{c}^{b}u \varphi^{\prime}\\
	&=-\int_{a}^{c} u^{\prime} \varphi-\int_{c}^{b} u^{\prime} \varphi+\underbrace{\left(u\left(c^{-}\right)-u\left(c^{+}\right)\right)}_{=0} \varphi(c)\\
	&=-\int_{a}^{b} u^{\prime} \varphi.
\end{align*}

Donc, $u_1=u'$ est la dérivée généralisée de $u$. 
 La valeur $u_{1}(c)$ n'a pas d'importance: on a de toute façon au final la même fonction de $L^{2}(I)$, puisqu'elle est définie comme classe d'equivalence de la relation d'équivalence "égalité presque partout".

Un autre espace de fonctions régulières qui joue un rôle important par la suite est  
$$
\mathcal{C}^{1}(\bar{\Omega})=\left\{\varphi: \Omega \rightarrow \mathbb{R} : \exists \; O \; \text { ouvert contenant } \bar{\Omega}, \exists \psi \in \mathcal{C}^{1}(O), \psi_{\mid \Omega}=\varphi\right\}
$$

Autrement dit, $\mathcal{C}^{1}(\bar{\Omega})$ est l'espace des fonctions $\mathcal{C}^{1}$ sur $\Omega$, prolongeables par continuité sur $\partial \Omega$ et dont le gradient est lui-aussi prolongeable par continuité. 


\begin{theorem}\
	
1. 	 Quand elle existe, la dérivée généralisée est unique.

 2. Quand $u$ appartient  à $\mathcal{C}^{1}(\bar{\Omega})$, les dérivées généralisées sont  égales  aux  dérivées classiques.
\end{theorem}

\subsection{Espaces de Sobolev}

Dans cette sous-section,  on introduit  de nouveaux espaces fonctionnels, sous espaces des espaces  $L^p$, analogue aux espaces $C^p(\Omega)$, $p\geq 1$.   On commence par  $p=2$, le  cas le plus  utilisé. 
\subsubsection{Les espaces $H^{m}$}

\begin{definition}\
	
1. 	L'espace de Sobolev d'ordre $1 $  dans  $L^{2}(\Omega)$  est l'ensemble défini par
$$
H^{1}(\Omega)=\left\{u \in L^{2}(\Omega) : \partial_{i} u \in L^{2}(\Omega), \quad 1 \leq i \leq n\right\},
$$ 

où $\partial_{i} u$ est définie au sens de la dérivée généralisée. 

 2. Pour tout entier $m \geq 1$, le sous ensemble de $L^{2}(\Omega)$
$$
H^{m}(\Omega)=\left\{u \in L^{2}(\Omega): \partial^{\alpha} u \in L^{2}(\Omega),  \forall \alpha=\left(\alpha_{1}, \ldots, \alpha_{n}\right) \in \mathbb{N}^{n}: |\alpha|=\alpha_{1}+\cdots+\alpha_{n} \leq m\right\}
$$

est appelé espace de Sobolev d'ordre $m$.

3. Par extension, on voit aussi que $H^{0}(\Omega)=L^{2}(\Omega)$.

4. Dans le cas de la dimension 1, on écrit plus simplement pour $I$ ouvert de $\mathbb{R}$ :
$$
H^{m}(I)=\left\{u \in L^{2}(I) :  u^{\prime}, \ldots, u^{(m)} \in L^{2}(I)\right\}.
$$

\end{definition}

Exemple :  (exercice)  

Soient $\mathrm{I}=]-1,+1[ $ et  la fonction $u(x)=\frac{1}{2}(|x|+x)$. 

1. Montrer  que $u$  appartient à $H^1(\mathrm{I})$  et que $u^{\prime}=\mathrm{H}$ où
$$
\mathrm{H}(x)=\left\{\begin{array}{clrl}
	+1 & \text { si } & 0<x<1 \\
	0 & \text { si } & -1<x<0.
\end{array}\right.
$$

2. Montrer que $H$ n'appartient pas à $H^1(\mathrm{I})$. 
\begin{theorem}\
	
1. 	$H^{1}(\Omega)$ est un espace de Hilbert pour le produit scalaire
\begin{equation}\label{H1}
	\langle u, v\rangle_{1}=\int_{\Omega} u v+\sum_{i=1}^{n} \int_{\Omega} \partial_{i} u \partial_{i} v=\langle u, v\rangle+\sum_{i=1}^{n}\langle\partial_{i} u, \partial_{i} v\rangle, 
\end{equation}

en notant $\langle\cdot, \cdot \rangle$ le produit scalaire $L^{2}$. On notera $\|\cdot\|_{1}$ la norme associée à $\langle \cdot, \cdot\rangle_{1}$.


2. 
Si $\Omega$ est un ouvert de $\mathbb{R}^{n}$ de frontière $\partial \Omega$ "suffisamment régulière" (par exemple
$\left.\mathcal{C}^{1}\right)$,  l'espace $\mathcal{C}^{1}(\bar{\Omega})$ est dense dans $H^{1}(\Omega)$. 



3. On définit de même un produit scalaire et une norme sur $H^{m}(\Omega)$ par
$$
\langle u, v\rangle_{m}=\sum_{|\alpha| \leq m}\langle \partial^{\alpha} u, \partial^{\alpha}v\rangle  \qquad \text { et } \quad\|u\|_{m}=(u, u)_{m}^{1 / 2}
$$

$H^{m}(\Omega)$ muni du produit scalaire $\langle u, v\rangle_{m}$ est un espace de Hilbert.

4.  Si $\Omega$ est un ouvert de $\mathbb{R}^{n}$ de frontière $\partial \Omega$ "suffisamment régulière" (par exemple
$\left.\mathcal{C}^{1}\right)$, on a l'inclusion: $H^{\mathrm{m}}(\Omega) \subset \mathcal{C}^{k}(\Omega)$ (injection continue) pour $k<m-\frac{n}{2}$.


\end{theorem}
Exemples: En particulier, on voit que pour un intervalle $I$ de $\mathbb{R}$, on a $H^{1}(I) \subset \mathcal{C}^{0}(I)$, c'est à dire que, en 1-D, toute fonction $H^{1}$ est continue.

L'exemple de $u(x)=x \sin \frac{1}{x}$ pour $\left.\left.x \in\right] 0,1\right]$ et $u(0)=0$ montre que la réciproque est fausse.  (Exercice).


L'exemple de $u(x, y)=\left|\ln \left(x^{2}+y^{2}\right)\right|^{k}$ pour $0<k<1 / 2$ montre qu'en dimension supérieure à $1$,  il existe des fonctions $H^{1}$ discontinues. (Exercice).


Les fonctions de $H^{1}$ sont « en gros » des primitives de fonctions de $L^{2}$. Plus précisément on a le résultat suivant :

\begin{theorem}
	
 Soit $u \in H^{1}(I) ;$ alors il existe une fonction $\tilde{u} \in \mathrm{C}(\bar{I})$ telle que et
$$
\begin{aligned}
	\boldsymbol{u} &=\tilde{\boldsymbol{u}}\;\;  \text { p.p. sur } \mathbf{I} \\
	\tilde{\boldsymbol{u}}(\boldsymbol{x})-\tilde{\boldsymbol{u}}(y) &=\int_{y}^{x} \boldsymbol{u}^{\prime}(\boldsymbol{t}) \mathrm{d} t,     \quad \forall x, y \in \overline{\mathrm{I}}.
\end{aligned}
$$

\end{theorem}


Pour $1 \leqslant p \leqslant \infty$,  on peut définir aussi les espaces de Sobolev suivants:


L'espace de Sobolev $\mathbf{W}^{1, p}(\Omega)$ est défini par $\left({ }^{1}\right.$ )
$$
\mathbf{W}^{1, p}(\Omega)=
\left\{u \in \mathrm{L}^{p}(\Omega) : 
	\exists  g_{1}, g_{2}, \dots, g_n\in \mathrm{L}^{p}(\Omega) :
	\int_{\Omega}u \frac{\partial \varphi}{\partial x_i}=-\int_{\Omega} g_{i} \varphi \quad \forall \varphi \in D(\Omega),  \quad \forall i=1,2, \dots,n. 
\right\}
$$

Pour $p=2$, 
$\mathrm{W}^{1,2}(\Omega)= \mathrm{H}^{1}(\Omega)$. 

Pour $u\in \mathrm{W}^{1,p}(\Omega)$, $ g_{i}=\frac{\partial u}{\partial x_{i}}$ est la dérivée généralisée de $u$.  On note 

$$
 \nabla u=\left(\frac{\partial u}{\partial x_{1}}, \frac{\partial u}{\partial x_{2}}, \cdots, \frac{\partial u}{\partial x_{\mathrm{N}}}\right)=\operatorname{grad} u
$$
le gradient généralisé. 

L'espace $\mathbf{W}^{1, p}(\Omega)$,  muni de la norme
$$
\|u\|_{\mathrm{W}^{1, p}}=\|u\|_{\mathrm{L}^p}+\sum_{i=1}^{\mathrm{N}}\left\|\frac{\partial u}{\partial x_{i}}\right\|_{\mathrm{L}^p}
$$

ou parfois de la norme équivalente $\left(\|u\|_{L^{p}}^{p}+\displaystyle \sum_{i=1}^{N}\left\|\frac{\partial u}{\partial x_{i}}\right\|_{L^{P}}^{p}\right)^{1 / p}$ (si $\left.1 \leqslant p<\infty\right)$,

est un Banach. 

Si  $\Omega$ est borné, alors  $C^{1}(\bar{\Omega}) \subset \mathbf{W}^{1, p}(\Omega)$  pour tout  $1 \leqslant p \leqslant \alpha$.


\subsection{Trace d'une fonction}

Pour pouvoir faire les intégrations par parties qui seront utiles par exemple pour la formulation variationnelle, il faut pouvoir définir le prolongement (la trace) d'une fonction sur le bord $\partial \Omega$ de l'ouvert $\Omega$.

Si $n=1$ : on considère un intervalle ouvert $I=] a, b[$ borné. On a vu que $H^{1}(I) \subset$ $\mathcal{C}^{0}(\bar{I})$ . Donc, pour  $u \in H^{1}(I), u$ est continue sur $[a, b]$. La  trace (la valeur )  de $u$,  sur les bords $a$ et $b$ de l'intervalle  $I=] a, b[$, $u(a), u(b)$,  est  bien définie.


$\underline{\text { Si }} n>1$ : on n'a plus $H^{1}(\Omega) \subset \mathcal{C}^{0}(\bar{\Omega})$. Comment alors définir la trace d'une fonction $u\in H^{1}(\Omega)$ sur le bord $\partial \Omega$ ?  La démarche est la suivante :

- Pour une fonction $u\in 
\mathcal{C}^{1}(\bar{\Omega})$,  la trace de $u$ sur $\partial \Omega$, notée $u|_{\partial \Omega}$,  est  bien définie. 

- On peut voir que l'application linéaire  $\gamma_0: \mathcal{C}^{1}(\bar{\Omega})\ni u \longmapsto u|_{\partial \Omega}$ est  continue. 

- Comme, si $\Omega$ est un ouvert borné de frontière $\partial \Omega$ "assez régulière", alors $\mathcal{C}^{1}(\bar{\Omega})$ est dense dans $H^{1}(\Omega)$, alors, on a : le théorème trace: 

$\gamma_0$  se prolonge en une application linéaire continue de $H^{1}(\Omega)$ dans $L^{2}(\partial \Omega)$, notée encore 

$\gamma_{0}$, qu'on appelle opérateur  trace ($\gamma_{0}(u)$ est la trace de $u$ sur $\partial \Omega$). 

Pour une fonction $u$ de $H^{1}(\Omega)$ qui soit en même temps continue sur $\bar{\Omega}$, on a évidemment 

$\gamma_{0}(u)=u_{\mid \partial \Omega} .$ C'est pourquoi on note souvent par abus simplement $u_{\mid \partial \Omega}$ plutôt que $\gamma_{0}(u)$.

%
%On peut de façon analogue définir $\gamma_{1}$, application trace qui permet de prolonger la définition usuelle de la dérivée normale sur $\partial \Omega$. Pour $u \in H^{2}(\Omega)$, on a $\partial_{i} u \in H^{1}(\Omega), \forall i=1, \ldots, n$, et on peut donc définir $\gamma_{0}\left(\partial_{i} u\right)$. La frontière $\partial \Omega$ étant "assez régulière" (par exemple, idéalement, de classe $\left.\mathcal{C}^{1}\right)$, on peut définir la normale $n=\left(\begin{array}{l}n_{1} \\ \vdots \\ n_{n}\end{array}\right)$ en tout point de $\partial \Omega$. On pose alors $\gamma_{1}(u)=\sum_{i=1}^{n} \gamma_{0}\left(\partial_{i} u\right) n_{i}$. Cette application continue $\gamma_{1}$ de $H^{2}(\Omega)$ dans $L^{2}(\partial \Omega)$ permet donc bien de prolonger la définition usuelle de la dérivée normale. Dans le cas où $u$ est une fonction de $H^{2}(\Omega)$ qui soit en même temps dans $\mathcal{C}^{1}(\bar{\Omega})$, la dérivée normale au sens usuel de $u$ existe,

La formule de Green suivante généralise la notion d'intégration par parties. 
\begin{proposition}Formule de Green.\
	
	 On suppose que $\Omega$ est borné et de classe $C^{1}$. Soit $u, v \in H^{1}(\Omega) .$ Alors, 
	 
	  pour tout $i=1, \ldots, d$,  on a
$$
\int_{\Omega} \frac{\partial u}{\partial x_{i}} v d x=-\int_{\Omega} u \frac{\partial v}{\partial x_{i}} d x+\int_{\partial \Omega} u v n_{i} d \sigma
$$
où $n_{i}$ désigne la i-ème composante de la normale extérieure unitaire $\vec{n}$ à $\partial \Omega$.

Cette formule généralise la formule d'intégrations par parties sur $\mathbb{R}$ :

$$
\int_{a}^b u^\prime v d x=-\int_{a}^b u v^\prime d x+u(b)v(b)-u(a)v(a).
$$
\end{proposition}
\subsection{ Espace $\mathrm{H}_{0}^{1}(\Omega)$}

\begin{definition}\
	
 Soit $\Omega$ un ouvert de $\mathbb{R}^{n}$. L'espace $H_{0}^{1}(\Omega)$ est défini comme l'adhérence de $\mathcal{D}(\Omega)$ pour la norme $\|\cdot\|_{1}$ de $H^{1}(\Omega) .$ 

\end{definition}
\begin{theorem}\
	
Par construction $H_{0}^{1}(\Omega)$ est un espace complet. C'est un espace de Hilbert pour le produit scalaire de $H^1(\Omega)$. 
	
Si $n=1$ (cas 1-D) : on considère un intervalle ouvert $I=] a, b[$ borné. Alors
$$
H_{0}^{1}(] a, b[)=\left\{u \in H^{1}(] a, b[), u(a)=u(b)=0\right\}.
$$

Si $n>1:$ Si $\Omega$ est un ouvert borné de frontière "assez régulière" 

(par exemple $\mathcal{C}^{1}$ par morceaux, alors $H_{0}^{1}(\Omega)=\operatorname{ker} \gamma_{0}$.  

D'où, comme dans le cas $n=1$,  pour $u\in H^{1}(\Omega)$, $u\in H_{0}^{1}(\Omega)$ ssi   $\gamma_0u=0$.

\end{theorem}

 Pour toute fonction $u$ de $H^{1}(\Omega)$,  on peut définir l'application:
$$
|u|_{1}=\left(\sum_{i=1}^{n}\left\|\partial_{i} u\right\|_{L^2}^{2}\right)^{1 / 2}=\left(\int_{\Omega} \sum_{i=1}^{n}\left(\partial_{i} u\right)^{2} d x\right)^{1 }=\|\nabla u\|_{L^2} .
$$


\begin{theorem}(Inégalité de Poincaré) \
	
	Si $\Omega$ est borné, alors il existe une constante $C(\Omega)$ telle que 
	$$
	\forall u \in H_{0}^{1}(\Omega),\|u\|_{L^2} \leq C(\Omega)\|\nabla u\|_{L^2} .
	$$
	
On en déduit que $|\cdot|_{1}$ est une norme sur $H_{0}^{1}(\Omega)$, équivalente à la norme $\|\cdot\|_{1}$.


\end{theorem}

Dans la suite, on pourra avoir besoin du dual topologique  de $H_{0}^{1}(\Omega)$ qu'on note $H^{-1}(\Omega)$. On a ce résultat plus concret 


\begin{proposition}
	
	L'espace $H^{-1}(\Omega)$  est caractérisé par 
$$
H^{-1}(\Omega)=\left\{f=v_{0}+\sum_{i=1}^{n} \frac{\partial v_{i}}{\partial x_{i}} \quad \text { with } v_{0}, v_{1}, \ldots, v_{n} \in L^{2}(\Omega)\right\}
$$
Autrement, toute forme linéaire sur  $H_{0}^{1}(\Omega)$,  notée  $L \in H^{-1}(\Omega)$, est écrite  pour tout  $\phi \in H_{0}^{1}(\Omega)$
$$
L(\phi)=\int_{\Omega}\left(v_{0} \phi-\sum_{i=1}^{n} v_{i} \frac{\partial \phi}{\partial x_{i}}\right) d x
$$

avec  $v_{0}, v_{1}, \ldots, v_{n} \in L^{2}(\Omega)$.

Pour  $v \in L^{2}(\Omega) .$ For $1 \leq i \leq N$,  on définit forme linéaire  continue, dit dérivée faible au sens de $L^2$,  $\frac{\partial v}{\partial x_{i}}$ dans  $H^{-1}(\Omega)$ par 
$$
\left\langle\frac{\partial v}{\partial x_{i}}, \phi\right\rangle_{H^{-1}, H_{0}^{1}(\Omega)}=-\int_{\Omega} v \frac{\partial \phi}{\partial x_{i}} d x \quad \forall \phi \in H_{0}^{1}(\Omega).
$$

$\langle\cdot, \cdot \rangle_{H^{-1}, H_{0}^{1}}$ est   le crochet de dualité entre $H^{-1}$ et $H_{0}^{1}$.  On a le résultat : $$H_{0}^{1}(\Omega) \subset L^{2}(\Omega) \equiv\left(L^{2}(\Omega)\right)^{\prime} \subset H^{-1}(\Omega).
$$

\end{proposition}

Pour $1\leq p\leq \infty$, par la même technique, on peut aussi  définir les espaces  $W^{1,p}_0$, les fonctions de $W^{1,p}$ de traces nulles. 

\subsection{Dérivée Normale}

\begin{definition}
	
Pour $u \in H^{2}(\Omega)$, sa dérivée normale sur $\Gamma$ est définie par
$$
\gamma_{0} \frac{\partial u}{\partial \nu}=\sum_{i=1}^{n} \nu_{i} \gamma_{0} \frac{\partial u}{\partial x_{i}}
$$

où $\nu_{i}(x)$ désigne la ieme composante de la fonction $\nu(x)$,  le vecteur normal (perpendiculaire à la tengente au point $x\in \partial \Omega$. On note parfois cette dérivée normale par  $\gamma_1$, ou juste $\partial_n u$, ou $\frac{\partial u}{\partial n}$. 

\end{definition}


Remarquons que $\gamma_{0} \frac{\partial u}{\partial \nu} \in L^{2}(\partial \Omega)$, puisque tous les $\gamma_{0} \frac{\partial u}{\partial x_{i}}$ sont dans $L^{2}(\partial \Omega)$ et que $\left|\nu_{i}\right| \leqslant 1 .$ Ceci prouve aussi que l'application
$$
\gamma_{1} : H^{2}(\Omega) \rightarrow L^{2}(\partial \Omega): u \rightarrow \gamma_{0} \frac{\partial u}{\partial \nu}
$$
est linéaire continue.

Comme corollaire à la Formule de Green, on a cette formule appelée par le même nom.


\begin{corollary}
	
	
	
	
Pour un ouvert $\Omega$ de $\mathbb{R}^n$ "assez régulier"  et 	  $u\in H^2(\Omega)$, on a la formule de Green suivante 

	
$$\int_{\Omega} v\Delta u=\int_{\partial \Omega} v\frac{\partial u}{\partial n}  d \sigma-\int_{\Omega} \nabla u \cdot \nabla v, \quad \forall v \in H^{1}(\Omega),
$$
\end{corollary}

où  $d \sigma$ est la mesure surface sur  $\Gamma$. 

Si de plus, $u, v \in H^{2}(\Omega)$, on a
$$
\int_{\Omega}\{v\Delta u -u \Delta v\} d x=\int_{\partial \Omega}\left\{ v\frac{\partial u}{\partial n}  - u \frac{\partial v}{\partial n}\right\} d \sigma
$$

où on rappelle que $\Delta$ est l'opérateur laplacien défini par
$$
\Delta u=\sum_{i=1}^{n} \frac{\partial^{2} u}{\partial x_{i}^{2}}.
$$

\subsubsection{Autres espaces}

On définit l'espace,  dit l'espace $H$-$div$ et noté $H(div)$.

$$H\left(\operatorname{div} , \Omega\right)=\left\{v \in\left[L^{2}\left(\Omega\right)\right]^{d} ; \nabla \cdot v \in L^{2}\left(\Omega\right)\right\}.
$$

L'espace $H(div)$ est un Hilbert pour le produit scalaire suivant :


$$
\langle\sigma, \tau\rangle=\int_{\Omega}(\sigma(x) \cdot \tau(x)+\operatorname{div} \sigma(x) \operatorname{div} \tau(x)) d x.
$$


On définit aussi l'espace suivant, dit l'espace $H-curl$, et noté $H(curl)$, par 

$H\left(\operatorname{curl} , \Omega\right)=\left\{v \in\left[L^{2}\left(\Omega\right)\right]^{3} ; \nabla \times v \in\left[L^{2}\left(\Omega\right)\right]^{3}\right\}$, où 

$\nabla \times v$ est le rotationnel de la fonction $v:\Omega\subset \mathbb{R}^3\longrightarrow  \mathbb{R}^3,   (x,y,z)\longmapsto (v_x,v_y,v_z)$


défini par la formule

$$
\boldsymbol{\nabla} \times  v=\left(\begin{array}{c}
	\partial v_{z} / \partial y-\partial v_{y} / \partial z \\
	\partial v	_{x} / \partial z-\partial v_{z} / \partial x \\
	\partial v_{y} / \partial x-\partial v_{x} / \partial y
\end{array}\right) =\left(\frac{\partial v_{z}}{\partial y}-\frac{\partial v_{y}}{\partial z}\right) \overrightarrow{e_{x}}+\left(\frac{\partial v_{x}}{\partial z}-\frac{\partial v_{z}}{\partial x}\right) \overrightarrow{e_{y}}+\left(\frac{\partial v_{y}}{\partial x}-\frac{\partial v_{x}}{\partial y}\right) \overrightarrow{e_{z}}.
$$

Ces deux espaces sont très utiles dans la formulation variationnelle de certaines équations.

\section{Problems}

\begin{exercise}
  TODO
\end{exercise}



\chapter{Historical Notes}
\label{ch:cad-historical-notes}

References: \cite{PieglBook1996}, \cite{DeBoor_Book2001}, \cite{farin2002curves, farin1999nurbs, prautzsch2002bezier,rogers2001introduction,cohen2001geometric}

\todo{Add a general introduction}

\todo{Cost evalution of different math functions: ADD, MUL, EXP, SIN, COS, etc}




%----------------------------------------------------------------------------------------
%	PART 2: Finite Elements method 
%----------------------------------------------------------------------------------------
\part{Abstract framework}
\chapter{Introduction}
\section{ Théorie de Lax-Milgram}

Dans cette section, on considère le  problème général $(LMPV)$  dans le cas $W=V$ et 

$V$ est un espce de Hilbert
$$
(LMPV) \text{ trouver } u \in V\;  \text{tel que } a(u, v)=L(v) \;\, \text{pour tout}\;\;  v \in V.
$$



(i) $a$ est une forme bilinéaire  sur  $V\times V$. 

(ii) $L :V\longrightarrow \mathbb{R}$ une application linéaire.




	
Le théorème de Lax-Milgram suivant apporte une réponse à l'existence, l'unicité et la stabilité de la solution du problème ($LMPV$) .

\begin{theorem}\
	
On suppose que  les formes $a$ et $L$ verifient les hypothèses suivantes :

1. Continuité de $L$ : $L\in V'$.

2. Continuité de $a$ : $|a(u, v)| \leq C_{a}\|u\|_V \|v\|_{V}$ pour tout $u, v \in V$.

3. Coercivité de $a$ :  $ a(u, u) \geq \alpha\|u\|_{V}^{2}$,  pour tout $u \in V$,  avec $\alpha>0$.

Alors,  il existe une solution unique $u$ du problème ($LMPV$) qui vérifie l'estimation 

$$
\|u\|_{V} \leq \frac{\|L\|_{V^{\prime}}}{\alpha}.
$$

Ce qui signifie que l'application $L\longmapsto u$ est continue par rapport à $L$.  

On dit alors que le problème ($LMPV$)  est bien-posé au sens de Hadamard.
\end{theorem}

\begin{proof}
L'estimation  s'établit en prenant $v=u$ dans ($LMPV$) puis en appliquant 

la continuité de $L$ et la coercivité de $a$ ce qui donne
$$
\alpha\|u\|_{V}^{2} \leq \|L\|_{V^\prime}\|u\|_{V}.
$$

D'où l'estimation.   Cette estimation nous donne aussi l'unicité de la solution.
\end{proof}


Pour l'existence, considérons d'abord le cas simple où $a$ est une forme symétrique. 

Dans ce cas, la continuité et la coercivité de $a$ montrent que $(u,v)\longmapsto a(u,v)$ est un  produit 

scalaire sur $V \times V$ et que la norme $\|v\|_{a}=\sqrt{a(v, v)}$ est équivalente à la norme $\|\cdot\|_{V}$. 

Puisque $L$ est continue de $V$ dans $\mathbb{R}$, elle l'est aussi par rapport a $\|\cdot\|_{a}$. 

D'où,  par le {\bf théorème de représentation de Riesz}, il existe un unique $u \in V$ tel que 

$L(v)=a(u, v)$ pour tout $v \in V$.



Dans le cas non-symétrique, on remarque que puisque $v \mapsto a(u, v)$ et $v \mapsto L(v)$ sont continues, on peut écrire, par théorème de représentation de Riesz, 
$$
a(u, v)=\langle A u, v\rangle, \qquad L(v)=\langle f, v\rangle, 
$$ 

où $A$ est un opérateur continu sur $X, f \in X$ et $\langle\cdot, \cdot\rangle$ un produit scalaire dans $V$. 

L'équation ($LMPV$) s'écrit donc $A u=f$ dans $V$. L'hypothèse de coercivité fournit  l'estimation
$$
\alpha\|v\|_{V} \leq\|A v\|_{V}
$$
pour tout $v \in V$.  Par une preuve séquentielle,  on montre que $\operatorname{Im}(A)$ est un sous-espace fermé de $V$.  Donc,    $V=\operatorname{Im}(A) \oplus(\operatorname{Im}(A))^{\perp} .$ 

Pour un  $w \in(\operatorname{Im}(A))^{\perp}$,  la coercivité montre que
$$
\alpha\|w\|_{V}^{2} \leq a(w, w)=\langle A w, w\rangle=0.
$$
Par conséquent $(\operatorname{Im}(A))^{\perp}=\{0\} $ et donc $\operatorname{Im}(A)=V$. D'où, l'existence de la solution $u$.

\begin{proposition}\label{lax}\
	
Soit $V$ un espace de Hilbert, soit $a$ une forme bilinéaire continue coercive symétrique sur $V$ et $L \in V^{\prime}$. Alors, $u$ est l'unique  solution du probleme ($LMPV$) ssi $u$ est solution du problème de minimisation suivant:
$$
(MP)\;  \left\{\begin{array}{l}
	u \in V \\
	J(u) \leq J(v), \quad \forall v\in V,
\end{array}\right.
$$

où $J$ est définie de $V$ dans $\mathbb{R}$ par :
$$
J(v)=\frac{1}{2} a(v, v)-L(v),    \quad v\in V.
$$


\end{proposition}

\begin{proof}
Soit $u \in V$ solution unique de ($LMPV$); montrons que $u$ est solution de ($MP$). 

Soit $w \in V$, on va montrer que $J(u+w) \geq J(u)$ :
$$
\begin{aligned}
	J(u+w) &=\frac{1}{2} a(u+w, u+w)-L(u+w) \\
	&=\frac{1}{2} a(u, u)+\frac{1}{2}[a(u, w)+a(w, u)]+\frac{1}{2} a(w, w)-L(u)-L(w) \\
	&=\frac{1}{2} a(u, u)+\frac{1}{2} a(w, w)+a(u, w)-L(u)-L(w) \\
	&=J(u)+\frac{1}{2} a(w, w)\geq J(u)+\frac{\alpha}{2}\|w\|^{2}.
\end{aligned}
$$
Donc $J(u+w)>J(u)$ sauf si $w=0$.
Réciproquement, supposons maintenant que $u$ est solution du problème de minimisation ($MP$)  et montrons que $u$ est solution du problème ($LMPV$).  
\end{proof}

Soit $w \in V$ et $t>0 .$ On a $$ J(u+t w)-J(u) \geq 0\qquad \text{et } \; J(u-t w)-J(u) \geq 0$$ 

car $u$ minimise $J$. On en déduit que :
$$
t a(u, w)+\frac{1}{2} t^{2} a(w, w) -tL(w)\geq 0 \text { et }-t a(u, w)+\frac{1}{2} t^{2} a(w, w) +tL(w)\geq 0
$$
Comme $t$ est strictement positif, on peut diviser ces deux inégalités par $t$ :
$$
a(u, w)-L(w)+\frac{1}{2} t a(w, w) \geq 0 \text { et }-a(u, w)+L(w)+\frac{1}{2} ta(w, w) \geq 0
$$
On fait alors tendre $t$ vers 0 et on obtient $a(u, w)-L(w)=0$ pour tout $w \in V$, ce qui montre que $u$ est bien solution $\mathrm{du}$ problème$LMPV$).  

\section{The Banach-Necas-Babuska (BNB) Theorem : inf-sup conditions}
Dans cette section, on donne un résultat plus général que le théorème de  Lax-Milgram.  Ce résultat connu sous le nom de  Banach-Necas-Babuska Theorem, ou BNB Theorem donne une  condition nécessaire et suffisante pour la résolution du problème variationel $(LMPV)$ dans le cas plus général de $V$ et $W$ des espaces de Banach. 


\begin{theorem}(Banach-Necas-Babuska)\
	
Soient  $W$ un espace de Banach  et $V$ un   Banach réflexif.  Soit  $a \in \mathcal{L}(W \times V ; \mathbb{R})$  et  $f \in V^{\prime}$. Alors, Then, problem (2.1) is well-posed if and only if:
$$
\begin{aligned}
	&(\mathrm{BNB} 1) \quad \exists \alpha>0, \quad \inf _{w \in W} \sup _{v \in V} \frac{a(w, v)}{\|w\|_{W}\|v\|_{V}} \geq \alpha \\
	&(\mathrm{BNB} 2) \quad \forall v \in V, \quad(\forall w \in W, a(w, v)=0) \Longrightarrow(v=0)
\end{aligned}
$$
Moreover, the following a priori estimate holds:
$$
\forall f \in V^{\prime}, \quad\|u\|_{W} \leq \frac{1}{\alpha}\|f\|_{V^{\prime}}.
$$

\end{theorem}


\begin{remark}\
	
	
On définit l'opérateur  $A \in \mathcal{L}\left(W ; V^{\prime}\right)$ par 
	$$
	\forall w \in W, \forall v \in V, \quad\langle A w, v\rangle_{V^{\prime}, V}=a(w, v),
	$$
	

	 
	Alors,  le problème variationnel $(LMPV)$ est équivalent   à chercher  $u \in W$ telle que  
	$$
	A u=L\quad \text{dans }\;\; V^{\prime}.
	$$ 
	
Donc, on peut voir, par des résultats de théorie d'opérateurs, que les conditions du 

Théorème BNB peut être traduite pour l'opéarteur $A$ comme suit :


	$$
	(\mathrm{BNB} 1) \Longleftrightarrow(\operatorname{Ker}(A)=\{0\}   \text{ et } \operatorname{Im}(A)  \; \text{est fermé} ) \Longleftrightarrow A^{*} \text{ est   surjectif }
	$$ 
	
	$$
	(\mathrm{BNB} 2) \Longleftrightarrow \quad\left(\operatorname{Ker}\left(A^{*}\right)=\{0\}\right) \quad \Longleftrightarrow A^{*}  \text{  est  injectif}.
	$$
	
2.  Dans le cas  de  $W=V$, on verra que la condtion de coercivité  du théorème de  Lax-Milgram implique les conditions $(BNB1)$ et $(BNB2)$. En effet, 
	
 Supposons la coercivité de $a$ et soit  $w \in V$.  La condition $(BNB1)$ découle de 
 
$$
\alpha\|w\|_{V} \leq \frac{a(w, w)}{\|w\|_{V}} \leq \sup _{v \in V} \frac{a(w, v)}{\|v\|_{V}}
$$

Soit maintenant $v \in V$.  Pour $w=v$, on a  

$$
\sup _{w \in W} a(w, v) \geq a(v, v) \geq \alpha\|v\|_{V}^{2}. 
$$

Donc, $\displaystyle \sup _{w \in W} a(w, v)=0$ implique  que $v=0$.  D'où  $(BNB2)$ est démontré.
\end{remark}

\subsection{Exemples : }

{\bf The Laplace equation :} 

Considérons l'équation aux dérivées partielles elleptique

\begin{equation}
	\begin{cases}
		-\Delta u=f \;\; \text{ in }\; \Omega\\
		u_{\mid \partial \Omega}=0.
	\end{cases}
\end{equation}

  Ce problème peut être reformulé sous la forme du problème $(LMPV)$ en  posant
$$
\left\{\begin{array}{l}
	W=V=H_{0}^{1}(\Omega) \\
	a(u, v)=\int_{\Omega} \nabla u \cdot \nabla v, 
\end{array}\right.
$$

et  $L(v)=\int_{\Omega} f v$ pour  $f \in L^{2}(\Omega)$.



{\bf The Stokes equation :}  

Considérons l'équation  elleptic Stokes PDEs   
\begin{equation}
\begin{cases}
-\Delta u+\nabla p=h, \quad \nabla \cdot u=g  \;\; \text{ in }\; \Omega\\
u_{\mid \partial \Omega}=0\\
\int_\Omega pdx=0.
\end{cases}
\end{equation}

Ce problème se met sous forme variationnelle en posant
$$
\left\{\begin{array}{l}
	W=V=\left[H_{0}^{1}(\Omega)\right]^{d} \times L_{0}^{2}(\Omega) \\
	a((u, p),(v, q))=\int_{\Omega} \nabla u:\nabla v-\int_{\Omega} p \nabla \cdot v+\int_{\Omega} q \nabla \cdot u, 
\end{array}\right.
$$

$\Omega$ est un ouvert de $\mathbb{R}^d$,   $L(v, q)=\int_{\Omega}(h \cdot v+g q)$ pour  $h\in\left[L^{2}(\Omega)\right]^{d}$ et  $g \in L^{2}(\Omega)$. 

 Ici, $L_{0}^{2}(\Omega)$ est un sous espace de $L^2(\Omega)$ de fonctions de de moyenne zéro sur  $\Omega$. 
 
 

{\bf The advection equation. }

Soit  $\beta \in\left[\mathcal{C}^{1}(\bar{\Omega})\right]^{d}$ un vecteur donné et  notons  $\partial \Omega^{-}=\{x \in \partial \Omega ;(\beta \cdot n)(x)<0\}$, 
$n$ est le vecteur normal sortant à  $\partial \Omega$. 

Considérons  l'équation d'advection  suivante : 

\begin{equation}
	\begin{cases}
	\beta \cdot \nabla u=f  \;\; \text{ in }\; \Omega\\
		u_{\mid \partial \Omega^{-}}=0. 
	\end{cases}
\end{equation}

 Ce problème se met sous forme variationnelle en posant
$$
\left\{\begin{array}{l}
	W=\left\{u \in L^{2}(\Omega) ; \beta \cdot \nabla u \in L^{2}(\Omega) ; u=0 \text { on } \partial \Omega^{-}\right\}, \quad V=L^{2}(\Omega) \\
	a(u, v)=\int_{\Omega} v(\beta \cdot \nabla u)
\end{array}\right.
$$

et  $L(v)=\int_{\Omega} f v$ pour  $f \in L^{2}(\Omega) $ et $v\in V$. Ici l'espace des solutions et l'espace test 

sont différents.

\subsection{Conditions aux bords non homogènes}

Considérons l'équation aux dérivées partielles elleptique, munie de conditions aux bords non homogènes suivante

\begin{equation}
	\begin{cases}
		-\Delta u=f \in  L^{2}(\Omega)\\
		u_{\mid \partial \Omega}=g\in H^{\frac12}(\partial \Omega), 
	\end{cases}
\end{equation}

où $H^{\frac12}(\partial \Omega)$ est l'espace de Sobolev "fractionnaire"  image de l'application trace 

$\gamma_0$ de $H^{1}(\Omega)$.
Cette équation  peut être formulée sous la forme du  problème variationnel 

"non homogène" suivant : trouver $u\in W=H^{1}(\Omega)$ telle  que 
$$
\left\{\begin{array}{l}
	a(u, v)=\int_{\Omega} \nabla u \cdot \nabla v=L(v), \quad \forall v\in V= H_{0}^{1}(\Omega)\\
	\gamma_0(u)=g\in B=H^{\frac12}(\partial \Omega),
\end{array}\right.
$$

Nous avons le résultat d'existence, d'uncité et stabilité d'un problème vartiationnelle non homogène abstrait, suivant :

\begin{proposition}\label{nonH}
Soit  $W, V$, and $B$ trois espaces de Banach , tel que $V$  est reflexif.  

Soient $\gamma_{0} \in \mathcal{L}(W ,B)$ et  $a \in \mathcal{L}(W \times V ; \mathbb{R}) .$ Supposons que  $\gamma_{0}$ est  surjective et que la  restriction de  $a$ à  $W_{0}\times V$, $W_{0} : =\operatorname{Ker}\left(\gamma_{0}\right)$,   satisfait les  conditions  BNB.  

Alors, le problème 

$$
(NHPV)\quad \left\{\begin{array}{l}
	\text { trouver  } u \in W \text { tel que} \\
	a(u, v)=L(v), \quad \forall v\in V\\
	\gamma_0(u)=g\in B,
\end{array}\right.
$$
est bien-posé et il existe  $c>0$ telle que, pour $L \in V^{\prime}$ et  $g \in B$, on a 
	$$
	\|u\|_{W} \leq c\left(\|L\|_{V^{\prime}}+\|g\|_{B}\right).
	$$
	
\end{proposition}

\begin{proof}
Puisque $\gamma_{0}$ est continue et  surjective,  alors, par Théorème de l'application ouverte, 

il  existe $c>0$ telle que,  pour tout  $g \in B$,  
il existe  $u_{g} \in W$  tel  que  

$\gamma_{0} u_{g}=g$  et $\left\|u_{g}\right\|_{W} \leq c\|g\|_{B}$. 
\end{proof}


Le problème $(NHPV)$ est équivalent à poser  $\phi=u-u_{g}$ et considérer le  problème 
$$
\left\{\begin{array}{l}
	\text { trouver  } \phi \in W_{0} \text { tel que} \\
	a(\phi, v)=L(v)-a\left(u_{g}, v\right), \quad \forall v \in V.
\end{array}\right.
$$
On a 
$$
\begin{aligned}
	\left|f(v)-a\left(u_{g}, v\right)\right| & \leq\left(\|f\|_{V^{\prime}}+\|a\|\left\|u_{g}\right\|_{W}\right)\|v\|_{V} \\
	& \leq\left(\|f\|_{V^{\prime}}+c\|a\|\|g\|_{B}\right)\|v\|_{V}.
\end{aligned}
$$

Alors, 
la forme  linéaire $L-a\left(u_{g}, \cdot\right)$ est continue sur $V$.  Le reste se déduit du Théorème BNB.
\section{Approximations de Galerkin}
Dans cette section, nous approchons le problème variationnel abstrait 
$$
(LMPV) \text{ trouver } u \in V\;  \text{tel que } a(u, v)=L(v) \;\, \text{pour tout}\;\;  v \in V
$$

par la méthode de Galerkin.

\subsection{Position du Problème}


L'dée principale des méthodes de Galerkin est de remplacer les espaces $W$ et $V$ par des espaces de dimension finie $W_{h}$ et $V_{h}$. L'espace $W_{h}$ est appelé espace de solutions ou espace d'essais, et l'espace $V_{h}$ est appelé espace de tests. On verra plus tard, comment construire de tels espaces par "les éléments finis". L'indice $h$ se référant à la taille du maillage. 


La méthode de Galerkin  suppose que  la solution  $u$ peut être  representée par une  solution  approchée :
$$
u=u_{0}+\sum_{j=1}^{N} a_{j} \phi_{j}
$$
où  les $\phi_{j}$  sont des  fonctions inconnues, $u_{0}$ est introduite  pour satsfaire les  conditions aux bourds, et les  $a_{i}$  sont des coefficients à determiner.



Ceci revient à  poser  l'espace
$$
W(h)=W+W_{h}.
$$
en supposant  qu'il est muni d'une  $\|\cdot\|_{W(h)}$ vérifiant :


(i) $\left\|w_{h}\right\|_{W(h)}=\left\|w_{h}\right\|_{W_{h}}$ pour tout  $w_{h} \in W_{h}$.

(ii) $\exists c>0: \|w\|_{W(h)} \leq c\|w\|_{W}$ pour tout  $w \in W$ ($W$ s'injecte  continuement dans  $W(h)$).

Dans sa forme générale, la méthode de Galerkin construit   une approximation 

de  la solution $u$ (des problèmes variationnels homogènes et non homogènes) 
en réslovant 

le problème approchée suivant :
$$
(PVA)\quad \left\{\begin{array}{l}
	\text { trouver  } u_{h} \in W_{h} \text { telle que  } \\
	a_{h}\left(u_{h}, v_{h}\right)=L_{h}\left(v_{h}\right), \quad \forall v_{h} \in V_{h}, 
\end{array}\right.
$$

où $a_{h}$ une approximation  de la forme bilinéaire $a$  et  $L_{h}$ une approximation de la forme  linéaire $L$ (ou $L_h-a_h(u_g, \cdot$)).

Un problème particulier de  $(PVA)$)  est,  lorsque  $W_h=V_{h}$,  est le suivant :

$$
\left\{\begin{array}{l}
	\text { Trouver  } u_{h} \in V_{h} \text { telle que  } \\
	a_{h}\left(u_{h}, v_{h}\right)=L_{h}\left(v_{h}\right), \quad \forall v_{h} \in V_{h}. 
\end{array}\right.
$$
Dans ce cas, on dit qu'on a une méthode  standard de Galerkin. Dans le cas $W_h\neq V_{h}$, la méthode est dite  méthode de Petrov-Galerkin, ou méthode  de Galerkin  
 non-standard.

Nous donnons des définitions de différentes approximations,  erreurs et convergence de la méthode.


\begin{definition}\
	
	1. (Conformité). L'approximation est dite  conforme   si  $W_{h} \subset W$  et  $V_{h} \subset V$. 
	
	Dans ce cas $W(h)=W$.   Sinon, elle est dite non-conforme.


2. (Approximabilité). L'approximation admet  la propriété d'approximabilité si 
$$
\forall w \in W, \quad \lim _{h \rightarrow 0}\left(\inf _{w_{h} \in W_{h}}\left\|w-w_{h}\right\|_{W(h)}\right)=0. 
$$

\end{definition}


\begin{definition}[Consistance  et  consistance  asymptotique]  Soit $u$ la solution du problème  $(LMPV)$.
	
(i) L'approximation est dite consistante  si  $a_{h}$ peut être étendue à $W(h) \times V_{h}$ et si la solution exacte $u$ satisfait  le problème approchée $(PVA)$, i.e., si
$$
\forall v_{h} \in V_{h}, \quad a_{h}\left(u, v_{h}\right)=L_{h}\left(v_{h}\right). 
$$

Dans le cas contraire,  l'approximation est dite non-consistante.

(ii) Si  $a_{h}$ est  uniformément  continue ( par rapport à $h$)  sur  $W_{h} \times V_{h}$, 

la méthode d'approximation est dite 
asymptotiquement  consistante s'il existe 

un  opérateur 
$\Pi_{h}: W \rightarrow W_{h}$ tel que : 


$$
\exists c>0: \quad \left\|\Pi_{h} w-w\right\|_{W(h)} \leq c \inf _{w_{h} \in W_{h}}\left\|w-w_{h}\right\|_{W(h)}, \quad \forall w \in W, 
$$ 

et 

$$
\lim _{h \rightarrow 0}\left(\sup _{v_{h} \in V_{h}} \frac{\left|L_{h}\left(v_{h}\right)-a_{h}\left(\Pi_{h} u, v_{h}\right)\right|}{\left\|v_{h}\right\|_{V_{h}}}\right)=0.
$$


L'erreur de consistance  $R_{h}(u)$ est définie par 
$$
R_{h}(u)=\sup _{v_{h} \in V_{h}} \frac{\left|L_{h}\left(v_{h}\right)-a_{h}\left(\Pi_{h} u, v_{h}\right)\right|}{\left\|v_{h}\right\|_{V_{h}}}.
$$

\end{definition}



L'une des propriétés notables des méthodes de Galerkin se trouvent dans le fait que l'erreur commise sur la solution $u$  est orthogonale aux sous-espaces d'approximation. C'est une conséquence immédiate de la consistance.


\begin{proposition}[Orthogonalité]
	
Si l'approximation est 	consistante,  on a la propriété d'orthogonalité 
$$
\forall v_{h} \in V_{h}, \quad a_{h}\left(u-u_{h}, v_{h}\right)=0. 
$$

\end{proposition}
\subsection{Etude du problème approché}

Dans cette section, on va montrer que le problème approché est bien-posé.  On distinguera le cas conforme, consistant et corecive et le cas général.


\subsubsection{Système Linéaire}
Le problème approché $(PVA)$  peut être écris sous la forme d'un système linéaire $Ax=b$. 

En effet, soit 

$$
M=\operatorname{dim} W_{h} \quad \text { et  } \quad N=\operatorname{dim} V_{h}
$$

Soient $\left\{\psi_{1}, \ldots, \psi_{M}\right\}$ une base de $W_{h}$  et  $\left\{\varphi_{1}, \ldots, \varphi_{N}\right\}$ une base  $V_{h}$.  

Ecrivons  $u_{h}$ dans  la base de  $W_{h}$,
$$
u_{h}=\sum_{i=1}^{M} U_{i} \psi_{i}.
$$

En introduisant cette formule dans le problème approché $(PVA)$ et prenant $\varphi_j$ comme fonctions tests, on obtient le système linéaire équivalent  

$$
\mathcal{A} U=b
$$
avec 
$$
\mathcal{A}_{i j}=a_{h}\left(\psi_{j}, \varphi_{i}\right), \quad 1 \leq i \leq N, 1 \leq j \leq M
$$

et  $b \in \mathbb{R}^{N}$ est le vecteur de  compoantes :
$$
b_{i}=L_{h}\left(\varphi_{i}\right), \quad 1 \leq i \leq N.
$$

D'où
$$
u_{h} \text { résout \,  }(PVA)\quad \Longleftrightarrow \quad \mathcal{A} U=b.
$$


\subsubsection{Cas conforme, consistant, corercif}
 Considérons le  of problème approché suivant 
$$
\left\{\begin{array}{l}
	\text {Trouver  } u_{h} \in V_{h} \text { telle que  } \\
	a\left(u_{h}, v_{h}\right)=L\left(v_{h}\right), \quad \forall v_{h} \in V_{h}
\end{array}\right.
$$

avec  $V_{h} \subset V$.  Notez qu'ici on a gardé les formes "continues" de l'application  bilinéaire $a$ et de forme linéaire  $L$, i.e. $a_h=a$ et $L_h=L$; donc, on est dans le cas consistant.  On a le résultat d'existence suivant :

\begin{proposition}\
	
Soient  $V$ un espace de Hilbert,  $a \in \mathcal{L}(V \times V ; \mathbb{R})$, et $L \in V^{\prime}$. 

Soit  $V_{h}$ un espace de dimension finie.

Supposons que : $a$ est  coercive sur   $V$ et  $V_{h} \subset V$.

Alors,   le  problème approché ci-dessus est bien posé. En particulier,  pour tout $L \in$ $V^{\prime}$, on a 
$$
\left\|u_{h}\right\|_{V} \leq \frac{1}{\alpha}\|L\|_{V^{\prime}}.
$$ 

\end{proposition}

\begin{proof}
Puisque  $V_{h} \subset V$, la forme bilinéaire $a$ is coercive sur  $V_{h}$ avec la même constante 

$\alpha$.  On conclut alors par le Théorème de  Lax-Milgram.
\end{proof}


\begin{remark}
Dans le cas  d'une approximation conforme et  consistante d'un  problème coercif 
($W=V$), 
la matrice $\mathcal{A}$ du système linéaire est une matrice {\bf carrée définie  positive} (Exercice).  
\end{remark}

\begin{remark}
Si  $a$ est  symétrique,  la matrice $\mathcal{A} $  est  symétrique.
\end{remark}




\subsection{Cas général : Cas  BNB}

On considère le cas général ou $W\neq V$, et  une approximation qui peut être non-conforme ou non constante.


En s'insipirant du  Théorème BNB,  pour  résoudre le problème approché $(PVA)$, on suppose 

les  conditions  discrètes:
$$
\begin{aligned}
	&\left(\mathrm{BNB} 1_{\mathrm{h}}\right) \quad  \exists \alpha_{h}>0, \quad \inf _{w_{h} \in W_{h}} \sup _{v_{h} \in V_{h}} \frac{a_{h}\left(w_{h}, v_{h}\right)}{\left\|w_{h}\right\|_{W_{h}}\left\|v_{h}\right\|_{V_{h}}} \geq \alpha_{h} \\
	&\left(\mathrm{BNB} 2_{\mathrm{h}}\right) \quad \forall v_{h} \in V_{h}, \quad\left( a_{h}\left(w_{h}, v_{h}\right), \forall w_{h} \in W_{h}=0\right) \Longrightarrow\left(v_{h}=0\right)
\end{aligned}
$$

Donnons une interprétation des conditions $\left(\mathrm{BNB} 1_{\mathrm{h}}\right)$  et  $\left(\mathrm{BNB} 2_{\mathrm{h}}\right)$ en terme de la matrice $\mathcal{A}$ 

du système linéaire.

\begin{proposition}\
	
	(i) $\left(\mathrm{BNB} 1_{\mathrm{h}}\right) \Longleftrightarrow(\operatorname{Ker}(\mathcal{A})=\{0\})$.
	
(ii) $(\mathrm{BNB} 2 \mathrm{~h}) \Longleftrightarrow\left(\operatorname{rank} \mathcal{A}=\operatorname{dim} V_{h}\right)$.

(iii) Si $\operatorname{dim} W_{h}=\operatorname{dim} V_{h},\quad \left(\mathrm{BNB} 1_{\mathrm{h}}\right) \Longleftrightarrow\left(\mathrm{BNB} 2_{\mathrm{h}}\right)$.

\end{proposition}

Par cette proposition, on peut montrer facilement le résultat,  d'existence de solution 

approchées pour le problème $(PVA)$, suivant :

\begin{theorem}\
	
Soient $V_{h}$ et  $W_{h}$  deux espaces de dimension finite munis des normes $\|\cdot\|_{W_{h}}$  et  $\|\cdot\|_{V_{h}}$.  

Supposons 

(i) $a_{h}$ est  bilinéaire continue  sur  $W_{h} \times V_{h}$ et  $L_{h}$ est  continue su  $V_{h}$.

(ii) La condition $\left(\mathrm{BNB} 1_{\mathrm{h}}\right)$ is satisfaite.

(iii) $\dim V_{h}=\dim W_{h}$.

Alors,  le problème approché $(PVA)$ est bien posé, et on a  $\left\|u_{h}\right\|_{W_{h}} \leq \frac{1}{\alpha_{h}}\left\|L_{h}\right\|_{V_{h}^{\prime}}$.
\end{theorem}

\begin{proof}
Par les hypothèses du Théorème et la Proposition précédente, la matrice $\mathcal{A}$ est carrée et inversible donc, le système linéaire admet une solution unique.
\end{proof}
\section{Analyse d'Erreur}

Dans cette section, nous dérivons des estimations de l'erreur d'approximation $u-u_{h}$, où  $u$ résout le problème  $(LMPV)$ et  $u_{h}$ solution du  problème $(PVA)$


\subsubsection{Cas général }

Supposons que :


(i) La condition $\left(\mathrm{BNB} 1_{\mathrm{h}}\right)$ est satisfaite uniformément en  $h$ et 

 $\operatorname{dim}\left(W_{h}\right)=\operatorname{dim}\left(V_{h}\right)$.
 
 
(ii) $a_{h}$ is uniformément continue en  $h$  sur $W_{h} \times V_{h}$.


(iii) L'approximation est   asymptotiquement  consistante.

(iv) L'approximation admet la propriété  d'approximabilité.

Alors, l'erreur de consistance  $R_{h}(u)$ vérifie :
$$
\left\|u-u_{h}\right\|_{W(h)} \leq \frac{1}{\alpha} R_{h}(u)+c \inf _{w_{h} \in W_{h}}\left\|u-w_{h}\right\|_{W(h)}
$$

et  $$
\lim _{h \rightarrow 0}\left\|u-u_{h}\right\|_{W(h)}=0.
$$


\subsubsection{Cas Particuliers }

{\bf Cas non-consistant, non-conforme :}


 On suppose que $a_{h}$ peut être étendue à $W(h) \times V_{h}$ tel que   $a_{h}\left(w, v_{h}\right)$ est définie pour  $w \in W$  et  $v_{h} \in V_{h}$. 
 
 Nous avons le résultat d'estimations de l'erreur suivant :
\begin{proposition} (Strang 2)\
	
	 On suppose : 
	 
(i)  La condition $\left(\mathrm{BNB} 1_{\mathrm{h}}\right)$ et   $\operatorname{dim}\left(W_{h}\right)=\operatorname{dim}\left(V_{h}\right)$.

(ii) $a_{h}$ est coninue  sur $W(h) \times V_{h}$.
Alors, 
$$
\begin{aligned}
	\left\|u-u_{h}\right\|_{W(h)} \leq &\left(1+\frac{\left\|a_{h}\right\|_{W(h), V_{h}}}{\alpha_{h}}\right) \inf _{w_{h} \in W_{h}}\left\|u-w_{h}\right\|_{W(h)} \\
	&+\frac{1}{\alpha_{h}} \sup _{v_{h} \in V_{h}} \frac{\left|f_{h}\left(v_{h}\right)-a_{h}\left(u, v_{h}\right)\right|}{\left\|v_{h}\right\|_{V_{h}}}.
\end{aligned}
$$
\end{proposition}


\begin{proof}
On utilise la formule 
$$
\begin{aligned}
	a_{h}\left(u_{h}-w_{h}, v_{h}\right) &=a_{h}\left(u_{h}-u, v_{h}\right)+a_{h}\left(u-w_{h}, v_{h}\right) \\
	&=L_{h}\left(v_{h}\right)-a_{h}\left(u, v_{h}\right)+a_{h}\left(u-w_{h}, v_{h}\right), \qquad \forall w_{h} \in W_{h},  \forall v_h\in V_h.
\end{aligned}
$$

l'hypothèse (i) et l'inégalité triangulaire des normes.
\end{proof}

{\bf Cas consistant et conforme :}

Pour une approximation  consistante et  conforme avec  $a_{h}=a$ and $L_{h}=L$, nous avons l'estimation suivante:

\begin{proposition}(Lemme de Céa). \
	
	On suppose 
	
	(i)  La condition $\left(\mathrm{BNB} 1_{\mathrm{h}}\right)$.
	
	
	(ii)  $\operatorname{dim}\left(W_{h}\right)=\operatorname{dim}\left(V_{h}\right)$.
	
	 (iii) $V_{h} \subset V$, $W_{h} \subset W, a_{h}=a$,  et  $L_{h}=L.$ 
	 
	 La solution  $u_{h}$ du problème  $(PVA)$ satisfait :
$$
\left\|u-u_{h}\right\|_{W} \leq\left(1+\frac{\|a\|_{W, V}}{\alpha}\right) \inf _{w_{h} \in W_{h}}\left\|u-w_{h}\right\|_{W}
$$

\end{proposition}
\begin{proof}
Pour  $w_{h} \in W_{h},$  l'othogonalité implique 
$$
\forall v_{h} \in V_{h}, \quad a\left(u_{h}-w_{h}, v_{h}\right)=a\left(u-w_{h}, v_{h}\right).
$$

La condition $\left(\mathrm{BNB} 1_{\mathrm{h}}\right)$, la continuité de  $a$ et l'inégalité triangulaire permet de conclure.
\end{proof}

Si en plus de la consistance, la conformité, on suppose la coercivité, alors 

 l'orthogonalité donne
$$
\forall v_{h} \in V_{h}, \quad a\left(u-u_{h}, u-u_{h}\right)=a\left(u-u_{h}, u-v_{h}\right).
$$

Par un raisonnement similaire, on obtient l'estimation plus petite suivante:

$$
\left\|u-u_{h}\right\|_{W} \leq\frac{\|a\|_{W, V}}{\alpha} \inf _{w_{h} \in W_{h}}\left\|u-w_{h}\right\|_{W}.
$$




\section{Problème de Points selles}
Dans cette section, on traitera un cas particulier du problème $(LMPV)$,  venant  de la formulation variationnelle du problème de  Stokes.  On lui  réfère  par "problème de point celle".  On  caractérisera son caratère bien-posé et analysera son  approximation par la méthode de  Galerkin.



\subsection{Position et Résolution du Problème}
On considère deux espaces de Banach réflexifs  $X$ et  $M$, $f \in X^{\prime}, g \in M^{\prime}$, et deux formes  bilinéaires $a \in \mathcal{L}(X \times X ; \mathbb{R})$ et  $b \in \mathcal{L}(X \times M ; \mathbb{R})$.  Le problème abstrait de point selle est :

$$
(PSP)\quad \left\{\begin{array}{l}
	\text {Trouver  } u \in X \text { et  } p \in M \text { tels que  } \\
	a(u, v)+b(v, p)=f(v), \quad \forall v \in X \\
	b(u, q)=g(q), \quad \forall q \in M. 
\end{array}\right.
$$


L'exemple prototype de $(PSP)$ est celui de  Stokes.  Dans ce cas,  $$X=\left[H_{0}^{1}(\Omega)\right]^{d}, \; M=L_{0}^{2}(\Omega),$$
$$
a(u, v)=\int_{\Omega} \nabla u: \nabla v,\quad  b(v, p)=-\int_{\Omega} p \nabla \cdot v, \quad f(v)=\int_{\Omega} f \cdot v, \,   g(q)=-\int_{\Omega} g q dx.
$$

pour un $f\in \left[L^{2}(\Omega)\right]^{d}$ et $g\in L^{2}(\Omega)$. 

On peut écrire le problème de point selle $(PSP)$ comme un problème 

particulier de $(LMPV)$

$$
\left\{\begin{array}{l}
	\text { Trouver  }(u, p) \in V \text { telle que } \\
	c((u, p),(v, q))=k(v, q), \quad \forall(v, q) \in V,
\end{array}\right.
$$

pour 

$$W=V=X \times M, \quad c((u, p),(v, q))=a(u, v)+b(v, p)+b(u, q),
$$
$$L(v, q)=f(v)+g(q).
$$

Il est facile de voir que les deux problèmes $(PSP)$ and $(LMPV)$ sont équivalents. Par conséquent,  une condition  necéssaire et suffisante pour que $(PSP)$ soit bien-posé est les deux conditions $(BNB1)$ et  $(BNB2)$  pour la forme bilinéaire $c$.  Cependant,  vu  sa particularité, il est possible de formuler les conditions  $(BNB1)$  et  $(BNB2)$ en terme des formes bilinéaires $a$ et $b$. Pour cela, on introduit le problème "opérateur associé" équivalent

$$
\left\{\begin{array}{l}
	\text { Trouver } u \in X \text {  et  } p \in M \text { tels que  } \\
	A u+B^{*} p=f \\
	B u=g,
\end{array}\right.
$$



 $A$ and $B$ sont les opérateurs définis par  $A: X \rightarrow X^{\prime},    \quad \langle A u, v\rangle_{X^{\prime}, X}=a(u, v)$,   
 
 $B: X \rightarrow M^{\prime}$, avec $(B v, q\rangle_{M^{\prime}, M}=b(v, q)$  et 
 $B^{*}: M=M^{\prime \prime} \rightarrow X^{\prime}$ son opérateur adjoint. 

Introduisons  le noyau de $B$ par  
$$
\operatorname{Ker}(B)=\{v \in X ; \forall q \in M, b(v, q)=0\}
$$ 

%l'opérateur $$\pi A: \operatorname{Ker}(B) \rightarrow \operatorname{Ker}(B)^{\prime}$$ 
%
% $$\langle\pi A u, v\rangle_{X^{\prime}, X}=\langle A u, v\rangle_{X^{\prime}, X}, \quad \forall u, v \in \operatorname{Ker}(B).
%$$


On a alors le résultat d'existence, unicité et stabilité du point selle suivant :


\begin{theorem}\label{pointselle}
	
	
	 Le Problème $(PSP)$ est bien posé ssi 
\begin{equation}\label{psp}
\left\{\begin{array}{l}
	\exists \alpha>0, \displaystyle \inf _{u \in \operatorname{Ker}(B)} \sup _{v \in \operatorname{Ker}(B)} \frac{a(u, v)}{\|u\|_{X}\|v\|_{X}} \geq \alpha \\
	\forall v \in \operatorname{Ker}(B), \quad( a(u, v)=0, \; \forall u \in \operatorname{Ker}(B)) \Rightarrow \; v=0
\end{array}\right.
\end{equation}

et 

\begin{equation}\label{psp1}
\exists \beta>0, \quad \inf _{q \in M} \sup _{v \in X} \frac{b(v, q)}{\|v\|_{X}\|q\|_{M}} \geq \beta.
\end{equation}

En plus, les estimations suivantes sont satisfaites

$$
\left\{\begin{array}{l}
	\|u\|_{X} \leq c_{1}\|f\|_{X^{\prime}}+c_{2}\|g\|_{M^{\prime}} \\
	\|p\|_{M} \leq c_{3}\|f\|_{X^{\prime}}+c_{4}\|g\|_{M^{\prime}}
\end{array}\right.
$$

avec  $c_{1}=\frac{1}{\alpha}, c_{2}=\frac{1}{\beta}\left(1+\frac{\|a\|}{\alpha}\right), c_{3}=\frac{1}{\beta}\left(1+\frac{\|a\|}{\alpha}\right)$, et  $c_{4}=\frac{\|a\|}{\beta^{2}}\left(1+\frac{\|a\|}{\alpha}\right)$.

\end{theorem}


\begin{remark}\
	
	
	
(i)  Si  $a$ est  coercive sur  $\operatorname{Ker}(B)$,  alors les  conditions  \eqref{psp} sont vérifées, en particulier si  $a$ est  coercive sur l'espace  $X$.




(ii)  On peut voir que les conditions \eqref{psp} et \eqref{psp1} sont satisfaites par $a$ et $b$ ssi  les  conditions  $(\mathrm{BNB} 1)$ et $(\mathrm{BNB} 2)$ sont satisfaites par la forme bilinéaire $c$. 


\end{remark}



On va  définir  maintenant le vrai sense d'un point selle de la  solution du  problème $(PSP)$. 

\begin{definition}\
	
Soient  $X$  et  $M$ deux espaces, et une application  $\mathcal{L}: X \times M \rightarrow$ $\mathbb{R}$. 

Le couple  $(u, p)$ est dit un point selle de  $\mathcal{L}$ si 
$$
\forall(v, q) \in X \times M, \quad \mathcal{L}(u, q) \leq \mathcal{L}(u, p) \leq \mathcal{L}(v, p). 
$$
\end{definition}

Le point selle est caractérisé par :

\begin{lemma}\
	
	
	Le couple  $(u, p)$ est un point selle de  $\mathcal{L}$ ssi
$$
\inf _{v \in X} \sup _{q \in M} \mathcal{L}(v, q)=\sup _{q \in M} \mathcal{L}(u, q)=\mathcal{L}(u, p)=\inf _{v \in X} \mathcal{L}(v, p)=\sup _{q \in M} \inf _{v \in X} \mathcal{L}(v, q).
$$

\end{lemma}

La proposition suivante donne la solution du problème $(PSP)$ comme point selle d'une application. La démonstration se base sur la proposition \ref{lax}. 

\begin{proposition}\
	
	
Supposons que $a$ est  symmetrique et positive.  Alors, le couple  $(u, p)$ est solution du problème  $(PSP)$  ssi $(u, p)$ est un point selle de la fonction
 Lagrangienne 
$$
\mathcal{L}(v, q)=\frac{1}{2} a(v, v)+b(v, q)-f(v)-g(q).
$$
\end{proposition}



\subsection{Approximations du problème de point selle}

Cette subsection étudie les approximations conformes au problème $(PSP)$. Soit $X_h$
un sous-espace de $X$ et soit $M_h$  un sous-espace de $M$ de dimensions finies

Considérons le problème approché :

$$
(PSPA)\quad \left\{\begin{array}{l}
	\text { Trouvet  } u_{h} \in X_{h} \text { et  } p_{h} \in M_{h} \text { tels que  } \\
	a\left(u_{h}, v_{h}\right)+b\left(v_{h}, p_{h}\right)=f\left(v_{h}\right), \quad \forall v_{h} \in X_{h} \\
	b\left(u_{h}, q_{h}\right)=g\left(q_{h}\right), \quad \forall q_{h} \in M_{h}
\end{array}\right.
$$


Soit  $B_{h}: X_{h} \rightarrow M_{h}^{\prime}$ est l'operator induit  par  $b$ tel que  $\left\langle B_{h} v_{h}, q_{h}\right\rangle_{M_{h}^{\prime}, M_{h}}=$ $b\left(v_{h}, q_{h}\right) .$

 Soit  $\operatorname{Ker}\left(B_{h}\right)$ le noyau de  $B_{h}$, i.e.,
$$
\operatorname{Ker}\left(B_{h}\right)=\left\{v_{h} \in X_{h} ; \forall q_{h} \in M_{h}, b\left(v_{h}, q_{h}\right)=0\right\}. 
$$

Etudions tout d'abord le caractère bien-posé de du prblème  approché $(PSPA)$.


\begin{proposition}
	Le Problème $(PSPA)$ est bien posé ssi 
	\begin{equation}\label{psp}
		\left\{\begin{array}{l}
			\exists \alpha_h>0, \displaystyle \inf _{u \in \operatorname{Ker}(B_h)} \sup _{v \in \operatorname{Ker}(B_h)} \frac{a(u, v)}{\|u\|_{X_h}\|v\|_{X_h}} \geq \alpha_h \\
			\forall v \in \operatorname{Ker}(B_h), \quad( a(u, v)=0, \; \forall u \in \operatorname{Ker}(B_h)) \Rightarrow \; v=0
		\end{array}\right.
	\end{equation}
	
	et 
	
	\begin{equation}\label{psp1}
		\exists \beta_h>0, \quad \inf _{q \in M_h} \sup _{v \in X_h} \frac{b(v, q)}{\|v\|_{X_h}\|q\|_{M_h}} \geq \beta_h.
	\end{equation}
	
\end{proposition}


Nous terminons ce chapitre par une estimation d'erreur de la solution du problème du point selle, similaire à celle du lemme de Céa.
  



 
\begin{proposition}\
	


Sous les conditions \eqref{psp}-\eqref{psp1}, la solution $\left(u_{h}, p_{h}\right)$ du problème $(PSP)$   

satisfait les estimations
 $$
 \begin{array}{l}
 	\left\|u-u_{h}\right\|_{X} \leq c_{1 h} \inf _{v_{h} \in X_{h}}\left\|u-v_{h}\right\|_{x}+c_{2 h} \inf _{q_{h} \in M_{h}}\left\|p-q_{h}\right\|_{M} \\
 	\left\|p-p_{h}\right\|_{M} \leq c_{3 h} \inf _{v_{h} \in X_{h}}\left\|u-v_{h}\right\|_{X}+c_{4 h} \inf _{q_{h} \in M_{h}}\left\|p-q_{h}\right\|_{M}
 \end{array}
 $$
 
 avec  $c_{1 h}=\left(1+\frac{\|a\|}{\alpha_{h}}\right)\left(1+\frac{\|b\|}{\beta_{h}}\right)$, \
 
 $c_{2 h}=\frac{\|b\|}{\alpha_{h}}$  si  
 $\operatorname{Ker}\left(B_{h}\right) \not \subset \operatorname{Ker}(B)$ et  $c_{2 h}=0$ sinon, \
 
 $c_{3 h}=c_{1 h} \frac{\|a\|}{\beta_{h}}$, et  $c_{4 h}=1+\frac{\|b\|}{\beta_{h}}+c_{2 h} \frac{\|a\|}{\beta_{h}}$.


\end{proposition}






\section{Problems}

\begin{exercise}
  TODO
\end{exercise}



\chapter{Problèmes Coercifs}
Ce chapitre traite des problèmes dont la formulation faible est dotée d'une propriété de coercitivité. 

Les exemples clés étudiés désormais sont des EDP elliptiques scalaires, i.e.,  la solution $u$ est une fonction à valeurs réelles. 
L'objectif est double : Premièrement, mettre en place
un cadre mathématique pour le caractère bien-posé de ces équations. Deuxièment, étudier les approximations  par éléments finis,  conformes et non-conformes,  basées sur les méthodes de Galerkin.

Les estimations d'erreurs sont dérivées des résultats théoriques des chapitres  précédents 
et sont illustrés numériquement. La dernière section de ce chapitre concerne la 
 perte de la  coercitivité. 
 
\section{Equations aux dérivées Partielles Scalaires}


Soit  $\Omega$ un ouvert de  $\mathbb{R}^{d}$. Considérons  l'opérateur différential   $\mathcal{L}$ 
$$
\mathcal{L} u=-\nabla \cdot(\sigma \cdot \nabla u)+\beta \cdot \nabla u+\mu u
$$

où  $\sigma$ est une fonction définie sur $\Omega$ à valeurs  dans l'ensemble des matrices  carrées d'ordre $d$,  $\beta$ et $\mu$ des   fonctions, sur $\Omega$,  à valeurs dans $\mathbb{R}^{d}$ et  $\mathbb{R}$, respectivement. 

Donnons une fonction $f: \Omega \rightarrow \mathbb{R}$, et considérons le  problème  de trouver une  fonction $u: \Omega \rightarrow \mathbb{R}$ telle que 

\begin{equation}\label{eleq}
\left\{\begin{array}{ll}
	\mathcal{L} u=f & \text { in } \Omega \\
	\mathcal{B} u=g & \text { on } \partial \Omega, 
\end{array}\right.
\end{equation}

avec  $\mathcal{B}$  un opérateur qui donne les conditions aux bords.   Ce  modèle  apparait dans plusieurs  applications :


(i) Transfert de Chaleur: 

$u$ est la temérature, $\sigma=\kappa \mathcal{I}$,  le coefficient de  conductivité de la température, $\beta$ est le champ du flux de chaleur, $\mu=0$, and $f$ est une sourche de chaleur externe exercée sur $\Omega$.



(ii) Advection-diffusion: 

$u$ est  la concentration d'une sunstance  transportée dans un champ d'écoulemen t$\beta$. La matrice $\sigma$ modélise la diffusion de la substance.  La production ou la  destruction de la substance pour une réaction chimique est modilisée par le terme  $\mu u$, et  $f$ désigne une source fixée.

Dans toute la suite,  les  hypothèses suivantes sont faites sur les données : 

$$f \in L^{2}(\Omega), \sigma \in\left[L^{\infty}(\Omega)\right]^{d, d}, \beta \in\left[L^{\infty}(\Omega)\right]^{d}, \nabla \cdot \beta \in L^{\infty}(\Omega), \;\;
\mu \in L^{\infty}(\Omega) .
$$ 


En plus, l'opérateur $\mathcal{L}$ est supposé "uniformément" elliptic dans le sense suivant : 

\begin{definition}
	
L'opérateur $\mathcal{L}$ ci-dessus est dit elliptic s'il existe $\sigma_{0}>0$ telle que 

\begin{equation}\label{elip}
\forall \xi \in \mathbb{R}^{d}, \quad \sum_{i, j=1}^{d} \sigma_{i j} \xi_{i} \xi_{j} \geq \sigma_{0}\|\xi\|_{d}^{2} \quad \text { p.p. dans  } \;  \Omega.
\end{equation}

L'équation \eqref{eleq} est dit une EDP elleptic.
\end{definition}


Exemple : Un exemple fondamental d'opérateur elleptic est :  

$\mathcal{L}=-\Delta$, obtenu pour $\sigma=\mathcal{I}, \beta=0$, et  $\mu=0$.


Dans ce cas, l'équation est dite  un problème de  Poisson.

\subsection{Formulation Variationnelle et Solutions Fortes}

Nous donnons une formulation variationnelle du probléme 

\begin{equation}\label{eleq1}
	\left\{\begin{array}{ll}
		\mathcal{L} u=f & \text { in } \Omega \\
		\mathcal{B} u=g & \text { on } \partial \Omega, 
	\end{array}\right.
\end{equation}

{\bf Condtions aux bords de Dirichlet homogènes}: 

Considérons le cas des conditions aux bords de Dirichlet homogènes, i.e.,   

$$\mathcal{B} u=u_{|\partial \Omega}=g=0.
$$ 


Multiplions  l'équation $\mathcal{L} u=f$ par une fonction test  $v\in H^1_0(\Omega)$, et intégrons  sur  $\Omega$. 

Par la formule de  Green, 

\begin{equation}\label{green}
\int_{\Omega}-\nabla \cdot(\sigma \cdot \nabla u) v=\int_{\Omega} \nabla v \cdot \sigma \cdot \nabla u-\int_{\partial \Omega} v(n \cdot \sigma \cdot \nabla u),
\end{equation}

on a  

$$
\int_{\Omega} \nabla v \cdot \sigma \cdot \nabla u+v(\beta \cdot \nabla u)+\mu u v=\int_{\Omega} f v.
$$

On obtient alors le problème variationnel suivant 

\begin{equation}\label{eqV1}
\left\{\begin{array}{l}
\text{Trouver} \; u\in H_{0}^{1}(\Omega) : \\
	a_{\sigma,\beta,\mu}(u, v):=\int_{\Omega} \nabla v \cdot \sigma \cdot \nabla u+v(\beta \cdot \nabla u)+\mu u v=\int_{\Omega} f v=L(v), \forall v\in  H_{0}^{1}(\Omega).
\end{array}\right.
\end{equation}

Une solution de ce  problème variationnel est une solution forte de l'EDP  \eqref{eleq1} :

\begin{proposition}
	
 Si $u$ est solution de  \eqref{eqV1}, alors  $\mathcal{L} u=f$ p.p. dans  $\Omega$ et  $u=0$ p.p. sur  $\partial \Omega$.
 
\end{proposition}

\begin{proof}
Soit    $u$  solution to  \eqref{eqV1}.  Donc,
$$
\begin{aligned}
	&\int_{\Omega} \sigma \cdot \nabla u \cdot \nabla \varphi =\int_{\Omega}(f-\beta \cdot \nabla u-\mu u) \varphi, \quad \forall \varphi \in \mathcal{D}(\Omega).
\end{aligned}
$$

Par définition de $H^1(\Omega)$, on a $ \sigma \cdot \nabla u\in  (H^1(\Omega))^d$ et 

$$
\nabla \cdot( \sigma \cdot \nabla u)=f-\beta \cdot \nabla u-\mu u\in L^2(\Omega),
 $$
i.e., $\mathcal{L} u=f$ dans  $L^{2}(\Omega) .$ D'où, $\mathcal{L} u=f$ p.p. dans  $\Omega$. En plus,  $u=0$ p.p. sur  $\partial \Omega$ par définition de  $H_{0}^{1}(\Omega)$.  D'où le résultat.
\end{proof}

{\bf Conditions aux bords de Dirichlet non-homgènes }:  

Considérons maintenant le cas où  $u=g$ sur  $\partial \Omega$, pour une fonction  $g\in H^{\frac12}( \partial \Omega )$ donnée. Donc, il existe  $u_{g}\in H^{1}(\Omega)$ tel que  $u_{g}=g$ sur  $\partial \Omega$.    Comme ci-dessus, nous obtenons alors la formulation variationnelle :

\begin{equation}\label{eqV2}
\left\{\begin{array}{l}
	\text { Trouver  } u \in H^{1}(\Omega) \text { telle que } \\
	u=u_{g}+\phi, \quad \phi \in H_{0}^{1}(\Omega) \\
	a_{\sigma, \beta, \mu}(\phi, v)=\int_{\Omega} f v-a_{\sigma, \beta, \mu}\left(u_{g}, v\right), \quad \forall v \in H_{0}^{1}(\Omega).
\end{array}\right.
\end{equation}

Par un raisonnement similaire, nous montrons le résultat suivant :

\begin{proposition}
	
Soit $g \in H^{\frac{1}{2}}(\partial \Omega) .$	Si $u$ est solution de  \eqref{eqV2}, alors  $\mathcal{L} u=f$ p.p. dans  $\Omega$ et  $u=g$ p.p. sur  $\partial \Omega$.
	
\end{proposition}



{\bf Conditions aux bords de Neumann }: 

Soit  $g: \partial \Omega \rightarrow \mathbb{R}$, et considérons le cas  $n \cdot \sigma \cdot \nabla u=g$ sur  $\partial \Omega$. Dans le cas  $\sigma=\mathcal{I}$, tla condition de  Neumann  fait intervenir  la dérivée normale de  $u$ puisque $n \cdot \nabla u=\partial_{n} u$. Dans le cas général, $n \cdot \sigma \cdot \nabla u$ est la dérivée conormale de $u$ par rapport à $\mathcal{L}$.  

Par  la formule de Green \eqref{green},  et la condition de Neumann, $n \cdot \sigma \cdot \nabla u=g$,  nous obtenons le problème variationnel 



\begin{equation}\label{eqv3}
	\left\{\begin{array}{l}
	\text { Trouver  } u \in H^{1}(\Omega) \text { telle que  } \\
	a_{\sigma, \beta, \mu}(u, v)=\int_{\Omega} f v+\int_{\partial \Omega} g v, \quad \forall v \in H^{1}(\Omega).
\end{array}\right.
\end{equation}

Nous avons le résultat de solution forte suivant :


\begin{proposition}
	Soit  $g \in L^{2}(\partial \Omega) .$ Si $u$ est solution de  \eqref{eqv3}, alors  $\mathcal{L} u=f$ p.p. dans  $\Omega$ et  $n \cdot \sigma \cdot \nabla u=g$ p.p. sur  $\partial \Omega$.
\end{proposition} 

\begin{proof}
Comme pour le cas Dirichlet,  considérer les fonctions tests dans  $\mathcal{D}(\Omega)$  implique que  $\mathcal{L} u=f$ p.e.,  dans  $\Omega$.
Par suite, $-\nabla \cdot(\sigma \cdot \nabla u) \in L^{2}(\Omega) .$

Par la formule de Green et le faite, par le problème variationnel et $\mathcal{L} u=f$, on trouve 

$$
\forall \phi \in H^{\frac12}(\partial \Omega), \quad\int_{\partial \Omega} (n \cdot \sigma \cdot \nabla u)\phi =\int_{\partial \Omega} g \phi.
$$

D'où, 
$n \cdot \sigma \cdot \nabla u=g$ dansd $L^{2}(\partial \Omega)$.  
\end{proof}



\textbf{Robin boundary condition :}

Soeint  $g, \gamma: \partial \Omega \rightarrow \mathbb{R}$ deux fonctions données.  On considère maintenant la condition 

aux bords  

$$\gamma u+n \cdot \sigma \cdot \nabla u=g    \text{ sur }\;\; \partial \Omega, 
$$ 

dite conditions aux bords de Robin. 

Comme au paravant, par la formule de Green, une solution de $\mathcal{L}u=f$ dans $L^2(\Omega)$ satisfait 

le problème variationnel suivant :
\begin{equation}\label{eqv4}
\left\{\begin{array}{l}
	 u \in H^{1}(\Omega) \\
	a_{\sigma, \beta, \mu}(u, v)+\int_{\partial \Omega} \gamma u v=\int_{\Omega} f v+\int_{\partial \Omega} g v, \quad \forall v \in H^{1}(\Omega).
\end{array}\right.
\end{equation}

Le résultat suivant se démontre de la même façon  que le précédent.


\begin{proposition}
	Soient  $g \in L^{2}(\partial \Omega) $ et $g \in L^{\infty}(\partial \Omega)$.  Si $u$ est solution de  \eqref{eqv4}, alors  $\mathcal{L} u=f$ p.p. dans  $\Omega$ et  $\gamma u+n \cdot \sigma \cdot \nabla u=g$ p.p. sur  $\partial \Omega$.
\end{proposition} 


\begin{remark}
	
	On peut considérer aussi  des problèmes elleptiues  avec conditions aux bords mixtes Dirichlet et Neumann, e.g.  si $\partial \Omega =\Gamma_1\cup \Gamma2$, $u=0$ sur $\Gamma_1$ et $n \cdot \sigma \cdot \nabla u=0$ sur  $\Gamma_2$? Cette EDP peut formulé comme un problème variationnelle similaire aux prcédents sur l'espace $V=\{v\in H^1(\Omega ):  u=0  \;  \text{ sur }  \; \;  \Gamma_1\}$. 
\end{remark} 
\subsection{Existence de solutions faible}

Par  solutions faibles, on désigne les solutions des problèmes variationnels de la sous-section précédente. Pour montrer leur  existence, on appliquera le Théorème de Lax-Milgram. On vérfie facilement que les applications bilinéaires associées sont continues et que les seconds membres sont des forrmes linéaires coninues. Il restera à montrer la coercivité.

Dans le théorème  suivant, nous rassemblons les résulats d'existence pour les différentes conditions aux bords.

\begin{theorem}\label{thm12}\
	
Soient $f \in L^{2}(\Omega)$,  $\sigma \in\left[L^{\infty}(\Omega)\right]^{d, d}$  tel que  \eqref{elip} est satisfaites, et  $\beta \in\left[L^{\infty}(\Omega)\right]^{d}$ 

avec  $\nabla \cdot \beta \in L^{\infty}(\Omega)$, et $\mu \in L^{\infty}(\Omega) .$ Posons 


$p=\operatorname{infess}_{x \in \Omega}\left(\mu-\frac{1}{2} \nabla \cdot \beta\right)$ et  $c_{\Omega}$ la constante de l'inégalité de  Poincaré.



(i) Les problèmes variationnels avec conditions aux bords  Dirichlet  homogènes  et  

non-homogènes sont bien posés si
$$
\sigma_{0}+\min \left(0, \frac{p}{c_{\Omega}}\right)>0
$$

(ii) Le problèmes variationnel avec conditions aux bords  Neumann est bien-posé  si

$$
p>0 \quad \text { and } \quad \operatorname{infess}_{x \in \partial \Omega}(\beta \cdot n) \geq 0.
$$

(iii) Pour $q=\operatorname{infess}_{x \in \partial \Omega}\left(\gamma+\frac{1}{2} \beta \cdot n\right)$,   le problèmes variationnel avec 

conditions aux bords  Robin est bien posé si
$$
p> 0, \quad q \geq 0.
$$

\end{theorem} 

\begin{proof}
Montrons  $(i)$.  Par la Formule de Green, voir Proposition 11, on obtient l'identité 
\begin{equation}\label{ident}
\int_{\Omega} u(\beta \cdot \nabla u)=-\frac{1}{2} \int_{\Omega}(\nabla \cdot \beta) u^{2}+\frac{1}{2} \int_{\partial \Omega}(\beta \cdot n) u^{2}
\end{equation}

pour $u\in H^1_0(\Omega)$.   Donc, par l'ellipticité  de  $\mathcal{L}$, on a 

$$
\forall u \in H_{0}^{1}(\Omega), \quad a_{\sigma, \beta, \mu}(u, u) \geq \sigma_{0}|u|_{1, \Omega}^{2}+p\|u\|_{0, \Omega}^{2}.
$$

Pour $\delta=\min \left(0, \frac{p}{c_{\Omega}}\right)$,  l'inégalité de Poincaré montre que 
$$
\forall u \in H_{0}^{1}(\Omega), \quad a_{\sigma, \beta, \mu}(u, u) \geq\left(\sigma_{0}+\frac{\delta}{c_{\Omega}}\right)|u|_{1, \Omega}^{2} \geq \alpha\|u\|_{1, \Omega}^{2}
$$

avec 
$\alpha=\sigma_{0}+\frac{\delta}{c_{\Omega}}>0$, par hypothèse.  Le résultat découle par  Lax-Milgram. 
\end{proof}


Le problème non-homogène se déduit aussi de  la Proposition  \ref{nonH}.


Pour le cas des conditions aux bords de  Neumann, on utilise aussi l'identité  \eqref{ident}, et un 

raisonnement similaire.


Pour le cas Robin, posons  $a(u, v)=a_{\sigma, \beta, \mu}(u, v)+\int_{\partial \Omega} \gamma u v .$ 

On montre que  

$$
\forall u \in H^{1}(\Omega), \quad a(u, u) \geq \sigma_{0}|u|_{1, \Omega}^{2}+p\|u\|_{0, \Omega}^{2}+q\|u\|_{0, \partial \Omega}^{2}. 
$$

Si $p>0$ et   $q \geq 0$, la forme bilinéaire est  clairement  coercive sur $H^{1}(\Omega)$, avec la constante  $\alpha=\min \left(\sigma_{0}, p\right)$. 

D'où le problème est bien posé par  le théorème de Lax-Milgram.

\subsection{Approximation}


\section{Perte de Coercivité}

\subsection{Motivation}

Soit le système linéaire 

$$
\left(\begin{matrix}
\epsilon&\beta\\
0&\epsilon
\end{matrix}
\right)\left( \begin{matrix}
u_1\\u_2
\end{matrix}\right)= \left( \begin{matrix}
b_1\\b_2
\end{matrix}\right)
$$

Pour $\epsilon=0$, la matrice n'est pas bijective et donc il n'y pas uncité de la solution. 

Mais, pour $\epsilon>0$, elle est bijective.  

Analytiquement, la solution  est unique  et elle est donnée par 
$$
\left( \begin{matrix}
	u_1\\u_2
\end{matrix}\right)= \left(\begin{matrix}
\frac1{\epsilon}&-\frac{\beta}{\epsilon}\\
0&\frac1{\epsilon}
\end{matrix}
\right) \left( \begin{matrix}
	b_1\\b_2
\end{matrix}\right).
$$

Mais, numériquement si $\epsilon$ est trop petit, la solution devient très grande.  La  matrice est mal conditionnée.  

\subsection{Position du Problème}

Considérons le problème  variationnel 

$$
\left\{\begin{array}{l}
	\text { Trouver } u \in V \text { tel que } \\
	a_{\eta}(u, v)=f(v), \quad \forall v \in V,
\end{array}\right.
$$

où  $V$ est un  Hilbert, $f \in V^{\prime}$,  et  $a_{\eta}$ est continue, coercive, sur  $V \times V$. La forme $a_{\eta}$ dépend d'un parametre    $\eta$ qui peut prendre des petites valeurs. 

Posons $\left\|a_{\eta}\right\|:=\left\|a_{\eta}\right\|_{V, V}$ et par  $\alpha_{\eta}$ la constante de  coercivité de  $a_{\eta}$, i.e.,
$$
\alpha_{\eta}=\inf _{u \in V} \frac{a_{\eta}(u, u)}{\|u\|_{V}^{2}}.
$$
\begin{definition}

On a une perte de coercivité  si

$$
\lim _{\eta \rightarrow 0} \frac{\left\|a_{\eta}\right\|}{\alpha_{\eta}}=\infty. 
$$

\end{definition}
Par analogie avec la motivation,   la perte de  coercivité ici désigne que  $a_\eta$ est mal-conditionée.

Soit  $V_{h}$ une approximation conforme de  $V$ et supposons qu'on a l'estimation, vérifiée dans le cas des éléments finis,  
$$
\forall u \in W, \quad \inf _{v_{h} \in V_{h}}\left\|u-v_{h}\right\|_{V} \leq c_{i} h^{k}\|u\|_{W}, 
$$

où  $W$ est un sous-ensemble dense dans  $V$  et  $c_i$  est une constante d'interpolation.  

Soit  $u_{h}$  la solution du problème approché:
$$
\left\{\begin{array}{l}
	\text { Trouver  } u_{h} \in V_{h} : \\
	a_{\eta}\left(u_{h}, v_{h}\right)=f\left(v_{h}\right), \quad \forall v_{h} \in V_{h}.
\end{array}\right.
$$

Si la solution exacte $u$  est dans  $W$ et vérifie cette estimation, voir Chapitre 2, 
$$
\left\|u-u_{h}\right\|_{V} \leq \frac{\left\|a_{\eta}\right\|}{\alpha_{\eta}} c_{i} h^{k}\|u\|_{W}.
$$

Si le problème présente une perte de coercivité, cette estimation  ne donne aucun contrôle pratique pour  l'erreur. Évidemment, en gardant $\eta$ fixe, quand  $h \rightarrow 0$, la convergence est atteinte.  Mais, si $\eta$ est trop petite, le $h$ ne peut pas compenser la valeur très grande de $\frac{\left\|a_{\eta}\right\|}{\alpha_{\eta}}$. 


Considérons l'exemple de l'équation  d'advection-diffusion :
$$
\begin{aligned}
	-\nabla \cdot(\epsilon \nabla u)+b\cdot \nabla u &=f \quad \text { dans  } \Omega \\
	u &=0\quad \text { sur   }  \partial\Omega, 
\end{aligned}
$$
où  $\epsilon$ est  le  coefficient de diffusion, $b$ est le champ de  vitesse,  et $f$ est le terme  source.   

On suppose $\nabla \cdot b=0$. 

Nous avons montré dans le chapitre 2 (Théorème \ref{thm12}), que la forme bilinéaire $a_\epsilon$ associée 

à ce problème est coercive avec $\alpha=\epsilon$.  

Mais si $\epsilon =0$, $a_\epsilon$ n'est pas coercive.  On peut aussi montrer que 

$\frac{\left\|a_{\epsilon}\right\|}{\alpha_{\epsilon}}\longrightarrow \infty$ quand  $\epsilon\rightarrow 0$. 










\section{Problems}

\begin{exercise}
  TODO
\end{exercise}



\chapter{ Problèmes mixtes}

Dans ce chapitre, on considère un problème modèle qui s'exprime sous la forme d'un système d'équations aux dérivées partielles où interviennent plusieurs fonctions inconnues qui ne jouent pas le même rôle mathématique et physique. 

Par exemple, dans le problème de Stokes,
$$
\begin{aligned}
-\Delta u+\nabla p=f & \text { dans } \Omega \\
\nabla \cdot u=g & \text { dans } \Omega
\end{aligned}
$$
le champ $u: \Omega \rightarrow \mathbb{R}^{d}$ représente une vitesse et la fonction $p: \Omega \rightarrow \mathbb{R}$ une pression. 

Dans le problème de Darcy,
$$
\begin{aligned}
\sigma+\nabla u &=f & & \text { dans } \Omega \\
\nabla \cdot \sigma &=g & & \text { dans } \Omega
\end{aligned}
$$
le champ $\sigma: \Omega \rightarrow \mathbb{R}^{d}$ représente une vitesse et la fonction $u: \Omega \rightarrow \mathbb{R}$ une charge hydraulique. 

On établira la   formulation variationnelle de l'équation de Stokes et on étudiera son caractère bien posé.
En général, le caractère bien posé de ces problèmes résulte d'une condition inf-sup qui, comme on l'a vu dans le Chapitre 2, n'est pas automatiquement transférée au niveau discret. En pratique, cela veut dire que pour que l'approximation par éléments finis soit bien posée, il faut que les espaces d'approximation pour les deux fonctions inconnues satisfassent une condition de compatibilité donnant lieu à une condition inf-sup discrète. Dans ce cas, on parle d'éléments finis mixtes.

\section{Equation de Stokes}


Soient $\Omega\subset \mathbb{R}^n$, $u,f :  \Omega\longrightarrow \mathbb{R}^n$ et $g :  \Omega\longrightarrow \mathbb{R}$.   On considère l'équation de Stokes suivante:
$$
\begin{aligned}
-\Delta u+\nabla p=f & \text { dans } \Omega \\
\nabla \cdot u=g & \text { dans } \Omega\\
u=0& \text { sur  } \partial \Omega.
\end{aligned}
$$

La première équation est dite l'équation de la quantité de mouvement et la deuxième est dite l'équation de la conservation de la masse.
\subsection{Formulation Variationnelle}




On multiplie la première équation par $v\in \left[H_{0}^{1}(\Omega)\right]^{d}$. En utilsant la formule de Green, on trouve

\begin{eqnarray*}
-\int_{\Omega} v \cdot \Delta u+\int_{\Omega} v \cdot \nabla p&=\sum_{i=1}^{d}-\int_{\Omega} v_{i} \Delta u_{i}-\int_{\Omega} p \nabla \cdot v\\
&=\sum_{i=1}^{d} \int_{\Omega} \nabla u_{i} \cdot \nabla v_{i}-\int_{\Omega} p \nabla \cdot v\\
&=\int_{\Omega} \nabla u: \nabla v-\int_{\Omega} p \nabla \cdot v.
\end{eqnarray*}


puisque  $v=0$  sur $\partial\Omega$.  D'où, on obtient 
$$
\int_{\Omega} \nabla u: \nabla v-\int_{\Omega} p \nabla \cdot v=\int_{\Omega} f \cdot v
$$


Les trois  intégrales sont bien-définies si  $u$  et $v$ sont  $\left[H_{0}^{1}(\Omega)\right]^{d}, p$  dans  $L^{2}(\Omega)$,  et $f \in\left[L^{2}(\Omega)\right]^{d} .$

De même de la deuxième équation, on obtient

$$
-\int_{\Omega} q \nabla \cdot u 
=-\int_{\Omega} g q,  \qquad \forall q\in L^2(\Omega)
$$

et pour $q=c$, on a 

$$
\int_{\Omega} gc=\int_{\Omega} c\nabla \cdot u =\int_{\partial \Omega} u \cdot n=0.
$$

La condition $\displaystyle \int_{\Omega} g=0$ est donc une condition nécessaire à l'existence d'une solution $(u, p)$ pour l'équation  de Stokes avec  conditions aux limites de Dirchlet homogènes.

Donc,  il est inutile de tester la deuxième équation par les constantes. Ceci exige l'introduction de l'espace  fonctionnel
$$
L_{0}^{2}(\Omega)=\left\{q \in L^{2}(\Omega) ; \int_{\Omega} q=0\right\}.
$$

On observera aussi que dans l'équation  de Stokes, la pression n'est déterminée qu'à une constante additive près,  donc pour assurer l'unicité de la solution, on se restreint au cas de  champ de pression de moyenne nulle sur $\Omega $, i.e.,  $p\in L_{0}^{2}(\Omega)$. 


On obtient alors la formulation variationnelle suivante : 

$$
\left\{\begin{array}{c}
\text { Chercher }(u, p) \in\left[H_{0}^{1}(\Omega)\right]^{d} \times L_{0}^{2}(\Omega) \text { tel que } \\
\int_{\Omega} \nabla u: \nabla v-\int_{\Omega} p \nabla \cdot v=\int_{\Omega} f \cdot v, \quad \forall v \in\left[H_{0}^{1}(\Omega)\right]^{d} \\
-\int_{\Omega} q \nabla \cdot u 
=-\int_{\Omega} g q, \quad \forall q \in L_{0}^{2}(\Omega)
\end{array}\right.
$$

Nous obtenons alors un problème de point selle abstrait 

$$
(FVS)\quad \left\{\begin{array}{l}
\text {Trouver  } u \in X \text { et  } p \in M \text { tels que  } \\
a(u, v)+b(v, p)=f(v), \quad \forall v \in X \\
b(u, q)=g(q), \quad \forall q \in M, 
\end{array}\right.
$$
dans les espaces 
$X=\left[H_{0}^{1}(\Omega)\right]^{d}$ et $M=L_{0}^{2}(\Omega)$, et pour les formes bilinéaires

$$
a(u, v)=\int_{\Omega} \nabla u: \nabla v \quad \text { et } \quad b(v, p)=-\int_{\Omega} p \nabla \cdot v
$$
et les formes linéaires $f(v)=\int_{\Omega} f \cdot v$ et $g(q)=-\int_{\Omega} g q$.

\subsection{Solutions fortes et faibles} 


Avant de montrer l'existence d'une solution faible du problème variationnel $(FVS)$, on montre que toute solution faible est forte:


\begin{proposition}\
	
 Si $f$ et $g$ sont dans $\left[L^{2}(\Omega)\right]^{d}$ et $L_{0}^{2}(\Omega)$, respectivement, la solution faible $(u, p)$  satisfait
$$
\left\{\begin{aligned}
-\Delta u+\nabla p=f & \text { p.p. dans  } \Omega \\
\nabla \cdot u=g & \text { p.p.  dans  } \Omega \\
u=0 & \text { p.p.  sur  } \partial \Omega.
\end{aligned}\right.
$$

\end{proposition}

\begin{proof}
Considérons les  fonctions  tests dans  $[\mathcal{D}(\Omega)]^{d}$ pour la première équation. Intégration par parties  montre que 
$$
\forall v \in[\mathcal{D}(\Omega)]^{d}, \quad\langle-\Delta u+\nabla p, v\rangle_{\mathcal{D}^{\prime}, \mathcal{D}}=\int_{\Omega} f \cdot v
$$
puisque $f \in\left[L^{2}(\Omega)\right]^{d}$. Par densité de  $[\mathcal{D}(\Omega)]^{d}$ dans  $\left[L^{2}(\Omega)\right]^{d}$, l'égalité  $-\Delta u+$ $\nabla p=f$ est satisfaite dans  $\left[L^{2}(\Omega)\right]^{d}$,  qui implique une égalité p.p.  dans  $\Omega$. 
\end{proof}


De même, pour pour la deuxième équations.  La condition aux bords sur  $u$ vient du fait que  $u\in \left[H_{0}^{1}(\Omega)\right]^{d}$.

Le caractère bien posé du problème de Stokes  repose de manière fondamentale sur le résultat suivant: 

\begin{lemma}\label{lemmeAO}\


Soit $\Omega$ un domaine de $\mathbb{R}^{d}$ avec $d \geqslant 2 .$
 Alors, 
 
 
 1. l'opérateur
$$
B:=\nabla \cdot:\left[H_{0}^{1}(\Omega)\right]^{d} \longrightarrow L_{0}^{2}(\Omega)
$$
est surjectif.

2.  Il existe $\beta>0: $ pour tout $q\in L_{0}^{2}(\Omega)$, il existe $v\in  \left[H_{0}^{1}(\Omega)\right]^{d}  : \nabla \cdot v=q$ et $\|v\|_{1}\leq \frac1\beta \|q\|_0$.
\end{lemma}	

La 2ème assertion repose sur le Théorème de l'application ouverte.


\begin{theorem}
	
Le problème variationnel $(FVS)$ est bien posé et  il existe $c_{1}$ et  $c_{2}$ telles que, pour tout  $f \in\left[L^{2}(\Omega)\right]^{d}$ et  $g \in L_{0}^{2}(\Omega)$
$$
\|u\|_{1}+\|p\|_{0} \leq\left. c_{1}\|f\|_{0}\right|_{\Omega}+c_{2}\|g\|_{0}. 
$$

\end{theorem}


\begin{proof}
Nous appliquons le Théorème \ref{pointselle}. 

On a  $\operatorname{Ker}(B)=V_{0}=\left\{v \in\left[H_{0}^{1}(\Omega)\right]^{d} : \nabla \cdot v=0\right\} $ est un Hilbert puisque c'est un sous-espace fermé de  $\left[H_{0}^{1}(\Omega)\right]^{d}$.

On peut voir que la forme bilinéaire $a$ est  coercive sur  $\left[H_{0}^{1}(\Omega)\right]^{d}$, par l'inégalité de  Poincaré et donc  coercive sur  $V_{0}$.  Comme conséquence, les deux conditions \eqref{psp} sont satisfaites. 
\end{proof}

En plus,  par le lemme ci-dessus et par l'Appendice A.42, dans livre Alexander Earn, l'inégalité suivante 

$$
\inf _{q \in L_{0}^{2}(\Omega)} \sup _{v \in\left[H_{0}^{1}(\Omega)\right]^{d}} \frac{\int_{\Omega} q \nabla \cdot v}{\|v\|_{1, \Omega}\|q\|_{0, \Omega}} \geq \beta ???
$$


est satisfaite. C'est exactement  l'inégalité \eqref{psp1}.  D'où, le Théorème   \ref{pointselle} permet de conclure. 
\subsection{Formulation variationnelle alternative}


Dans cette sous-section, nous donnons une  formulation variationnellle alternative  de 

l'équation de Stokes, qui fait inclure la contrainte sur la divergence  dans l'espace 

des solutions.   D'après le lemme \ref{lemmeAO},   nous avons  :

Il existe $\beta>0: $ pour tout $g\in L_{0}^{2}(\Omega)$, il existe $u_g\in  \left[H_{0}^{1}(\Omega)\right]^{d} : $ 


$$
 \nabla \cdot u_g=g, \qquad \|u_g\|_{1}\leq \frac1\beta \|g\|_0.
 $$


Définissons l'espace de Hilbert suivant

$$
V_{0}=\left\{v \in\left[H_{0}^{1}(\Omega)\right]^{d} : \nabla \cdot v=0\right\} .
$$

Si $u$ vérifie la condition $\nabla \cdot u=g$, la fonction $u'=u-u_g$ vérifie $\nabla \cdot u'=0$ et 

donc appartient à $V_0$. 

Si on se restreint aux fonctions testes $v\in V\in V_0$, dans le problème variationnel

$$
(FVS)\quad \left\{\begin{array}{l}
\text {Trouver  } u \in X=\left[H_{0}^{1}(\Omega)\right]^{d} \text { et  } p \in M =L_{*}^{2}(\Omega)\text { tels que  } \\
a(u, v)+b(v, p)=f(v), \quad \forall v \in X \\
b(u, q)=g(q), \quad \forall q \in M=L_{0}^{2}(\Omega), 
\end{array}\right.
$$

avec  $b(v, p)=-\int_{\Omega} p \nabla \cdot v$, on aura $b(v, p)=0$ et donc  

$a(u, v)+b(v, p)=f(v)$ devient  $a(u, v)=f(v)$.  

L'équation $b(u, q)=g(q)$ s'écrit 

$$
-\int_{\Omega} q (\nabla \cdot u-g)=0=-\int_{\Omega} q (\nabla \cdot u-\nabla \cdot u_g)=-\int_{\Omega} q \nabla \cdot( u- u_g).
$$ 

 Or  $u'=u-u_g\in V_0$, cette dernière équation est toujours vérifiée. 

Donc, on obtient la formulation  variationnelle suivante 

$$
(FVS)'\quad \left\{\begin{array}{l}
\text {Trouver  } u' \in V_0\text { tels que  } \\
a(u', v)=f(v)-a(u_g, v), \quad \forall v \in V_0.
\end{array}\right.
$$


Cette formulation ne fait pas intervenir la pression $p$.


Il est facile de voir que : 

(i) la forme  bilinéaire $a$ est  continue et coercive sur  $V_{0} \times V_{0}$.

(ii)  La forme linéaire  $f(\cdot)-a\left(u_{g}, \cdot\right)$ est  continue  sur  $V_{0}$. En effet, 
$$
\forall v \in V_{0}, \quad\left|f(v)-a\left(u_{g}, v\right)\right| \leq\left(\|f\|_{0}+c\|g\|_{0}\right)\|v\|_{1}.
$$


Donc, par Théorème de Lax-Milgram, on a 

\begin{proposition}
	
	Le problème $(FVS)'$ est bien posé.
	
\end{proposition}

La  relation entre la formulation mixte $(FVS)$  et la formulation  $(FVS)'$ est donnée par : 

\begin{proposition}
	
Soit $u$  une  fonction  dans  $\left[H_{0}^{1}(\Omega)\right]^{d}$ et soit $u^{\prime}=u-u_{g}$.  Alors, les assertions suivantes sont équivalentes.

(a)  Il existe  $p\in L_{0}^{2}(\Omega)$ telle que $(u, p)$ est solution de $(FVS)$.

(b) $u^{\prime}$ est solution de $(FVS)'$.

\end{proposition} 


\begin{proof}
L'implication (a) $\Longrightarrow$ (b) est déja faite en haut. 

Montrons l'implication inverse  (b) $\Longrightarrow$ (a).  Supposons  que  $a(u', v)=f(v)-a(u_g, v), \quad \forall v \in V_0$.  Donc l'application  linéaire  $v\longmapsto a(u, v)-f(v)$ est  nulle sur $V_0.$  Il est aussi linéaire continue sur  $\left[H_{0}^{1}(\Omega)\right]^{d}$.  Par le Théorème de  Rham's,  voir Theorem B. 73 dans livre de Alexander Earn, il existe  $p\in L_{0}^{2}(\Omega)$ tel que  

$$a(u, v)-f(v)=\langle\nabla p, v\rangle_{H^{-1}, H_{0}^{1}}, \qquad \forall v \in \left[H_{0}^{1}(\Omega)\right]^{d}.
$$

 D'où, le résultat. 
\end{proof}

 
 
%\section{Approximation par éléments finis mixtes}
%
%
%\end{document}
%On s'intéresse maintenant à une approximation conforme du problème (6.21) par éléments finis mixtes dans le cadre de la méthode de Galerkin standard; voir la section 6.1.2. Étant donné deux espaces d'approximation $X_{h} \subset\left[H_{0}^{1}(\Omega)\right]^{d}$ et $M_{h} \subset L_{*}^{2}(\Omega)$, on considère le problème approché suivant :
%$$
%\left\{\begin{aligned}
%\text { Chercher }\left(u_{h}, p_{h}\right) & \in X_{h} \times M_{h} & \text { tel que } & \\
%\int_{\Omega} \nabla u_{h}: \nabla v_{h}-\int_{\Omega} p_{h} \nabla \cdot v_{h} &=\int_{\Omega} f \cdot v_{h}, & \forall v_{h} \in X_{h} & \\
%-\int_{\Omega} q_{h} \nabla \cdot u_{h} & &=-\int_{\Omega} g q_{b}, & & \forall q_{h} \in M_{b}
%\end{aligned}\right.
%$$
%D'après le théorème $6.5$, ce problème est bien posé si et seulement si les espaces $X_{h}$ et $M_{b}$ sont tels qu'il existe une constante $\beta_{b}>0$ telle que
%$$
%\inf _{q_{b} \in M_{b}} \sup _{v_{h} \in X_{b}} \frac{\int_{\Omega} q_{b} \nabla \cdot v_{h}}{\left\|q_{b}\right\|_{0, \Omega}\left\|v_{h}\right\|_{1, \Omega}} \geqslant \beta_{h}
%$$
%
%Lorsque cette condition est satisfaite avec une constante $\beta$ indépendante de $h$, on dit que les espaces $X_{h}$ et $M_{b}$ satisfont (6.26) uniformément en $h$.
%Lorsque la condition de compatibilité $(6.26)$ est satisfaite, le lemme $6.6$ permet d'obtenir une estimation d'erreur pour la vitesse en norme $H^{1}$ et une estimation d'erreur pour la pression en norme $L^{2}$. De plus, afin d'obtenir des estimations d'erreur pour la vitesse en norme $L^{2}$, on utilise la notion suivante.

\section{Problems}

\begin{exercise}
  TODO
\end{exercise}



\input{tex/part_2_historical_notes.tex}

%----------------------------------------------------------------------------------------
%	PART 3: Finite Elements method 
%----------------------------------------------------------------------------------------
\part{Finite Elements method (TO REPLACE/UPDATE)}
% ...................................................................
\chapter{Functional Analysis}
\label{ch:fem-functional-analysis}
\section{Notations and Preliminaries}

In the course of the lecture we shall work with the Sobolev spaces $ H^m(\Omega)$, $H(\textrm{div}, \Omega)$ and $H(\textrm{div}, \Omega)$ and recall here there basic properties without proof. For a more detailed presentation with proofs we refer to Section 2.1. of \cite{BoffiBook2013}.
%
\begin{itemize}
	\item we use blodface notation for spaces of vector functions. For instance in $3D$, $\Hgradv$ denotes the space $\left( H^1(\Omega) \right)^3$. 
        \item for an operator $d \in \{\Grad, \Curl, \Div \}$, we denote the kernel space $\mathcal{N}(d)$ while the range will be $\mathcal{R}(d)$.
%        \item we restrict our study to the unit logical domain. We therefor consider $\Omega = [0,1]^d$, with $d \in \{1,2,3\}$
\end{itemize}
%
% --- Sobolev Spaces
We shall need the following Sobolev spaces, given first without boundary conditions,
%
\begin{align}
    \Hgrad  &= \left\{ \varphi \in L^2(\Omega), ~ \nabla \varphi \in \Ltwov \right\} 
    \\
    \Hgradv &= \left\{ \PsiPsi \in L^2(\Omega), ~ \nabla \PsiPsi \in \Ltwov  \right\} 
    \\
    \Hcurl  &= \left\{ \PsiPsi \in L^2(\Omega), ~ \Curl \PsiPsi \in \Ltwov  \right\} 
    \\
    \Hdiv   &= \left\{ \PsiPsi \in L^2(\Omega), ~ \Div \PsiPsi \in \Ltwo  \right\} 
%  \label{}
\end{align}
%
and using the correspondant boundary conditions,
%
\begin{align}
    \Hgradzero &= \left\{ \varphi \in \Hgrad, ~ \varphi = 0 ~ \mbox{on} ~ \partial \Omega \right\}
    \\
    \Hgradvzero &= \left\{ \PsiPsi \in \Hgradv, ~ \PsiPsi = 0 ~ \mbox{on} ~ \partial \Omega \right\}
    \\
    \Hcurlzero &= \left\{ \PsiPsi \in \Hcurl, ~ \PsiPsi \times \nn = 0 ~ \mbox{on} ~ \partial \Omega \right\} 
    \\
    \Hdivzero &= \left\{ \PsiPsi \in \Hdiv, ~ \PsiPsi \cdot \nn = 0 ~ \mbox{on} ~ \partial \Omega \right\} 
    \\
    \Ltwozero  &= \left\{ \varphi \in L^2(\Omega); ~ \int_{\Omega} \varphi = 0  \right\}
%  \label{}
\end{align}
%
Scalar-valued test functions will be denoted by $\varphi$, while $\varphi_i$ will denote a scalar basis function, after reordering all the basis functions of a given discrete space.
\\
Vector-valued functions will be written in bold, like $\uu, \vv$. $\boldsymbol{\Psi}$ will denote a vector-valued test function, while$\boldsymbol{\Psi}_i$ will denote a vector-valued basis function (after reordering the basis functions). 
\\
Even if most of what follows holds for both the $2D$ and $3D$ cases, we will restrict our studies to the $2D$ one. We recall that in $2D$, there are two \textit{curl} operators, one acting on scalars $\Rotv \phi = \left( \partial_y \phi, - \partial_x \phi \right)$ and one acting on vectors $\Rots \PsiPsi = \partial_x \PsiPsi^y - \partial_y \PsiPsi^x$. Differential operators that return a vector (grad, curl) will be written in bold ($\Grad, \Rots$). 
%
%
We also recall the Green formula for the \textit{divergence} and \textit{curl/rotational} operators
%
\begin{align}
  \boxed{
    \int_{\Omega} \left( \Div \FF \right) G = - \int_{\Omega} \FF \cdot \Grad G 
    + \int_{\partial \Omega} \left( \FF \cdot \nn \right) G, \quad \forall \FF \in \Hdiv, \forall G \in H^1(\Omega)
  }
  \label{eq:green_div}
\end{align}
%
\begin{align}
  \boxed{
    \int_{\Omega} \left( \Rotv G \right) \cdot \FF = \int_{\Omega} G \Rots \FF  
    - \int_{\partial \Omega} \left( G \times \nn \right) \cdot \FF
    , \quad \forall \FF \in \Hcurl, \forall G \in H^1(\Omega)
  }
  \label{eq:green_rot}
\end{align}
%
If $\Omega \subset \mathbb{R}^d$, $\uu = \left( u_1, u_2, \ldots, u_d \right)$ and $\PsiPsi = \left( \Psi_1, \Psi_2, \ldots, \Psi_d \right)$ , we recall the notation 
%
\begin{align}
  \int_{\Omega} \Grad \uu : \Grad \PsiPsi = \sum_{i=1}^d \int_{\Omega} \Grad u_i \cdot \Grad \Psi_i  
%  \label{}
\end{align}

\section{Sobolev spaces}
We shall denote by $\mathcal{D}(\Omega)$ the space of distribution.
\subsection{The Sobolev space $W^{s,m}$}
We start by recalling the definition of Sobolev spaces:
\begin{definition}
  Let $s$ and $m$ be two integers with $s \geq 0$ and $1 \leq m \leq \infty$. The Sobolev space $W^{s,m}(\Omega)$ is defined as
  \begin{align}
    W^{s,m}(\Omega) = \{ u \in \mathcal{D}^\prime(\Omega), ~ D^{\alpha} u \in L^m(\Omega), | \alpha | \leq s \}
    \label{eq:sobolev-space}
  \end{align}
\end{definition}
The space $W^{s,m}(\Omega)$ can be equipped with norm
\begin{align}
  \| u \|_{W^{s,m}(\Omega)} := \sum\limits_{|\alpha| \leq s} \| D^{\alpha} u \|_{L^p(\Omega)}
  \label{eq:sobolev-space-norm}
\end{align}
%
\subsection{The Sobolev space $H^m$}

For any integer $m\geq 1$, one can define
\begin{equation}\label{eq:hm}
H^m(\Omega) = W^{m,2}(\Omega) := \{  v \in L^2( \Omega) \;|\; D^\alpha v \in L^2( \Omega), ~|\alpha|\leq m  \} 
\end{equation}
$H^m(\Omega)$ is a Hilbert space equipped with the scalar product 
\begin{align}
  \left( u,v \right)_{m, \Omega} := \sum\limits_{|\alpha| \leq m} \int_{\Omega} D^\alpha u D^\alpha v 
  \label{eq:sobolev-space-scalar-product}
\end{align}
the associated norm will be denoted $\| \cdot \|_{s,\Omega}$.

The most classical  second order operator is the Laplace operator, which reads in an arbitrary dimension $d$ (generally $d=1,2$ or $3$), $$\Delta u = \sum_{i=1}^d \frac{\partial^2 u}{\partial x_d^2}.$$
The classical Green formula for the Laplace operator reads: 
for $u\in H^1(\Omega)$ and $v\in H^1(\Omega)$ 
\begin{equation}\label{eq:greenlap}
  -\int_\Omega\Delta u\, v\dd x=\int_\Omega\nabla u\cdot\nabla v\, dx -\int_{\partial\Omega} \fracp{u}{n}
v\dd \sigma.
\end{equation}


For essential boundary conditions related with this Green formula we shall define the space
$$H^1_0(\Omega) = \{ v \in H^1(\Omega) \;|\; v_{|\partial\Omega} =0 \}.$$

Another classical operator which comes from elasticity is the bilaplacian operator $\Delta^2 = \Delta \Delta$, which is a fourth order operator. The Green formula needed for variational formulations of PDEs based on the bilaplacian reads
\begin{equation}\label{eq:greenbilap}
\int_\Omega \Delta^2 u\, v \dd \mathbf{x} = \int_\Omega u\,  \Delta^2 v \dd \mathbf{x}
+ \int_{\partial\Omega} \left(   u \fracp{\Delta v}{n} - v \fracp{\Delta u}{n} + \Delta u \fracp{v}{n} -\Delta v \fracp{u}{n}  \right)\dd\sigma 
\end{equation}

$$H^2_0(\Omega) = \{ v \in H^1(\Omega) \;|\; v_{|\partial\Omega} =0, \; \fracp{v}{n}\bigl|_{\partial\Omega}=0 \}.$$

\subsection{Inequalities}
\begin{lemma}[Poincar\'e]
  Let $1 \leq p \leq \infty$ and $\Omega$ be a bounded open set. Then, there exists a constant $C=C(p, \Omega)$, such that
  \begin{align}
    \forall v \in W_0^{1,p}(\Omega), \quad C \| v \|_{L^p(\Omega)} \leq  \| \nabla v \|_{L^p(\Omega)} 
    \label{eq:poincare-inequality}
  \end{align}
%  \label{}
\end{lemma}



\section{The Sobolev space $ H(\textrm{curl}, \Omega)$}

%The Green's formula that will be useful for variational problems involving the $\textrm{curl}$ operator reads
%\begin{equation}\label{eq:greencurl}
%\int_\Omega \mathbf{u}\cdot \nabla\times \mathbf{v} = \int_\Omega \nabla\times\mathbf{u}\cdot  \mathbf{v} 
%+ \int_{\partial\Omega} (\mathbf{u}\times \mathbf{n})\cdot \mathbf{v}\dd s.
%\end{equation}

\section{The Sobolev space $ H(\textrm{div}, \Omega)$}

%\begin{equation}\label{eq:greendiv}
%\int_\Omega\nabla\cdot \mathbf{u}\, v\dd x = -\int_\Omega \mathbf{u}\cdot\nabla v\dd x + \int_{\partial\Omega} \mathbf{u}\cdot \un \, v \dd \sigma
%\end{equation}

\section{DeRham sequences}
For any function $u \in \Hgrad$ we have $\Curl \Grad u = 0$. On the other hand, we have for any function $\uu \in \Hcurl$, $\Div \Curl \uu = 0$. We just have shown that $\Grad(\Hgrad) \subset \mathcal{N}(\Curl)$ and $\Curl(\Hcurl) \subset \mathcal{N}(\Div)$. This is summarized in the following diagram, known as DeRham sequence, without boundary conditions in this case,
%
\begin{align}
  \mathbb{R} \hookrightarrow \Hgrad  \xrightarrow{\quad \Grad \quad}  \Hcurl  \xrightarrow{\quad \Curl \quad}   \Hdiv  \xrightarrow{\quad \Div \quad}  \Ltwo  \xrightarrow{\quad} 0  
  \label{eq:derham-nobc}
\end{align}
and using the correspondant boundary conditions,
\begin{align}
  \Hgradzero  \xrightarrow{\quad \Grad \quad}  \Hcurlzero  \xrightarrow{\quad \Curl \quad}   \Hdivzero  \xrightarrow{\quad \Div \quad}  \Ltwozero  \xrightarrow{\quad} 0  
  \label{eq:derham-bc}
\end{align}
%
In fact, DeRham complexes are sequences of spaces $V_i$ and operators $\diff_i$ such that $\diff_{i+1} \circ \diff_i = 0$. It leads to a sepcific algebraic structure that has been subject to active research in Analysis and Algebraic Geometry.
or in a differential forms setting
%
\begin{align}
\mathbb{R} \hookrightarrow \Lambda^0  \xrightarrow{\quad d \quad}  \Lambda^1  \xrightarrow{\quad d \quad} \Lambda^2  \xrightarrow{\quad d \quad}  \Lambda^3 \xrightarrow{\quad} 0 
\end{align}
%
where $d$ stands for the exterior derivative, while $\Lambda^k$ is the space of $k$-forms.

\begin{equation*}
\begin{array}{ccccccc}
            & \Grad &                 & \Curl &                    & \Div   &             \\
\Hgrad & \longrightarrow & \Hcurl & \longrightarrow  & \Hdiv & \longrightarrow & \Ltwo    \\
            &       &                 &       &                    &        &             \\
\Pigrad \Big\downarrow   &     & \Picurl \Big\downarrow  &   & \Pidiv \Big\downarrow &  & \Piltwo \Big\downarrow       \\
            & \Grad &                 & \Curl &                    & \Div   &             \\
\Vgrad &\longrightarrow  & \Vcurl &\longrightarrow   & \Vdiv & \longrightarrow & \Vltwo   \\
%
\\
\end{array}
\end{equation*}

\subsection{Exact discrete DeRham sequence}


%***********************************************************************


\subsection{Space decompositions}
%
More details can be found in \cite{monk_book, girault1986a}.
%
\begin{align}
	\mathcal{N}(\Curl) = \nabla \Hgradzero \oplus \left( \nabla \Hgradzero \right)^{\perp} 
%	\label{}
\end{align}
where $ \left( \nabla \Hgradzero \right)^{\perp}$ is the orthogonal of $\Hgradzero$ in $\Hcurlzero$ with respect to its inner product. We denote this space $K_N(\Omega)$, which is refered as \textit{the normal cohomology space}.  
\begin{align}
	K_N(\Omega) := \left( \nabla \Hgradzero \right)^{\perp} 
%	\label{}
\end{align}
$K_N(\Omega)$ is the space of harmonic functions which vanish on the exterior and are constant in the interior connected components of $\partial\Omega$. Note that $K_N(\Omega) \subset \Hgradvzero$.
%
\begin{theorem}
	If $\uu \in \Hcurlzero$, such that $\Curl \uu = 0$, then there exists a unique scalar potential $p \in \Hgradzero$ and a function $\ff_N \in K_N(\Omega)$ such that
	\begin{align}
		\uu = \nabla p + \ff_N
%		\label{}
	\end{align}
%	\label{}
\end{theorem}
%
\begin{theorem}[Helmoltz Decomposition]
	For every $\uu \in \Ltwov$ there exist a unique
	\begin{itemize}
		\item $p \in \Hgradzero$ 
		\item $\ff_N \in K_N(\Omega)$ 
		\item $\mathbf{A} \in \{ \ww \in \Hcurl, ~ \Div \ww = 0, ~ \nn \cdot \ww = 0 ~ \partial\Omega, ~ < \ww \cdot \nn, 1 >_{\Gamma_l} = 0  \}$ 
	\end{itemize}
	ensuring the following decomposition
	\begin{align}
		\uu = \nabla p + \Curl \mathbf{A} + \ff_N 
%		\label{}
	\end{align}
%	\label{}
\end{theorem}
%
\begin{remark}
	If $\Omega$ is homotopy equivalent to a ball, then
	\begin{align}
		\mathcal{N}(\Curl) = \nabla \Hgradzero
%		\label{}
	\end{align}
	Therefor, the Helmoltz decomposition writes
	\begin{align}
		\uu = \nabla p + \Curl \mathbf{A}
%		\label{}
	\end{align}
%	or using the function spaces
%	\begin{align}
%		\Ltwov = \nabla \Hgradzero \oplus \Curl \AA
%%		\label{}
%	\end{align}
\end{remark}
%
\begin{theorem}[Regular Decomposition of $\Hcurl$]
%
    For any $\uu \in \Hcurl$ there exists a $\vv \in \Hgradv$ such that
    \begin{enumerate}
            \item $\Curl \vv = \Curl \uu$ 
            \item $\| \vv \|_{\Ltwov} \lesssim \| \uu \|_{\Ltwov}$
            \item $\| \vv \|_{\Hgradv} \lesssim \| \Curl \uu \|_{\Ltwov}$
    \end{enumerate}
%	\label{}
\end{theorem}
%
The following result can be found in \cite{Pasciak2002a} and \cite{Zhao2002a}.
%
\begin{theorem}[Regular Decomposition of $\Hcurlzero$]
    For any $\uu \in \Hcurlzero$ there exists a $\vv \in \Hgradvzero$ such that
    \begin{enumerate}
            \item $\Curl \vv = \Curl \uu$ 
            \item $\| \vv \|_{\Ltwov} \lesssim \| \uu \|_{\Ltwov}$
            \item $\| \vv \|_{\Hgradvzero} \lesssim \| \Curl \uu \|_{\Ltwov}$
    \end{enumerate}
    \label{th:regular-decomposition-bc}
\end{theorem}
%
%%\subsection{Edge Space decompositions}
%%%
%%%
%%%\subsubsection*{The HX decomposition}
%%%
%%%
%%%\subsubsection*{The scalar HX decomposition}
%%%
%%%
%%\subsection{Hitpmair-Xu decomposition}
%%%
%%%
%%\begin{theorem}
%%%\hspace{-3cm}$\clubsuit$\hspace{3cm} 
%%  Every $\uu_h \in \Vcurl$, has a, non unique, decomposition
%%  \begin{align}
%%      \uu_h = \ww_h + \Picurl \vv_h + \nabla p_h 
%%%    \label{}
%%  \end{align}
%%  where
%%  \begin{itemize}
%%    \item $\ww_h \in \Vcurl$ 
%%    \item $\vv_h \in \Vgradv$ 
%%    \item $p_h \in \Vgrad \oplus K_{N,h}(\Omega)$ 
%%    \item $\left( h^{-2} + \mu \right) \|\ww_h\|_{\Ltwo}^2 + \| \vv_h \|_{\mu}^2 + \mu \| \nabla p_h\|_{\Ltwo}^2 \lesssim \left( \| \Curl \uu_h\|_{\Ltwo}^2 + \mu \| \uu_h\|_{\Ltwo}^2 \right) $
%%  \end{itemize}
%%  with
%%  \begin{align}
%%    \|\vv_h \|_{\mu}^2 := \| \vv_h\|_{\Hgradvzero}^2 + \mu \|\vv_h \|_{\Ltwo}^2  \quad \mbox{(HX-1)} 
%%%    \label{}
%%  \end{align}
%%  or
%%  \begin{align}
%%    \|\vv_h \|_{\mu}^2 := \| \Curl \Picurl \vv_h\|_{\Ltwo}^2 + \mu \|\Picurl \vv_h \|_{\Ltwo}^2  \quad \mbox{(HX-2)} 
%%%    \label{}
%%  \end{align}
%%%  \label{}
%%\end{theorem}
%%%
%%\begin{proof}
%%Using (Corollary \ref{cor:discrete-decomposition}), we can write $\uu_h = \Picurl \vv + \nabla p_h$. Thanks to (Lemma \ref{lemma:stable-component}), there exists a stable component $\vv_h \in \Vgrad$ of $\vv$. Now let's define $\ww_h := \Picurl \left( \vv - \vv_h \right)$, we have
%%%
%%\begin{align}
%%  \uu_h &= \Picurl \vv + \nabla p_h = \Picurl \left( \vv - \vv_h \right) + \Picurl \vv_h + \nabla p_h  
%%  \\
%%        &= \ww_h + \Picurl \vv_h + \nabla p_h  
%%  \label{}
%%\end{align}
%%\end{proof}
%%%
%%\subsection{Scalar Hitpmair-Xu decomposition}
%%%
%%Let us first introduce the component-wise operators
%%%
%%\begin{align}
%%\Picurlx v := \Picurl (v, 0, 0)  
%%\\
%%\Picurly v := \Picurl (0, v, 0)  
%%\\
%%\Picurlz v := \Picurl (0, 0, v)  
%%%  \label{}
%%\end{align}
%%%
%%The aim of this section, is to provide another version of the HX decomposition, based only on scalar projectors. We the above notations we can write
%%%
%%\begin{align}
%%  \Picurl \vv = \sum_{k=1}^{3} \Picurlk v^k, \quad \forall \vv = \left( v^1, v^2, v^3 \right)
%%%  \label{}
%%\end{align}
%%%
%%\begin{theorem}
%%\hspace{-3cm}$\clubsuit$\hspace{3cm} 
%%  Every $\uu_h \in \Vcurl$, has a, non unique, decomposition
%%  \begin{align}
%%      \uu_h = \ww_h + \sum_{k=1}^{3} \Picurlk v_h^k + \nabla p_h 
%%%    \label{}
%%  \end{align}
%%  where
%%  \begin{itemize}
%%    \item $\ww_h \in \Vcurl$ 
%%    \item $\vv_h \in \Vgradv$ 
%%    \item $p_h \in \Vgrad \oplus K_{N,h}(\Omega)$ 
%%    \item $\left( h^{-2} + \mu \right) \|\ww_h\|_{\Ltwo}^2 + \sum_{k=1}^{3} \| v_h^k \|_{k,\mu}^2 + \mu \| \nabla p_h\|_{\Ltwo}^2 \lesssim \left( \| \Curl \uu_h\|_{\Ltwo}^2 + \mu \| \uu_h\|_{\Ltwo}^2 \right) $
%%  \end{itemize}
%%  with
%%  \begin{align}
%%    \|v_h^k \|_{k,\mu}^2 := \| \Curl \Picurlk v_h^k\|_{\Ltwo}^2 + \mu \|\Picurlk v_h^k \|_{\Ltwo}^2  
%%%    \label{}
%%  \end{align}
%%%  \label{}
%%\end{theorem}
%%%
%%\begin{proof}
%%  TODO
%%\end{proof}

% ...................................................................
% ...................................................................
\section{Problems}
\label{sec:fem-functional-analysis-problems}
TODO




\chapter{Galerkin methods}
\label{ch:fem-abstract}
\section{Abstract framework}
% TODO M is the continuity coef => to be added
We consider $a$ and $L$ to be continuous bilinear and linear forms, respectively, on a Hilbert space $V$.
We want to find a computable approximation for the solution $u \in V$ of the variational problem

\begin{align}
  a(u,v) = \langle L, v \rangle, \quad \forall v \in V
  \label{eq:variational-problem}
\end{align}
where $\langle ~\cdot, ~\cdot \rangle$ denotes the duality product between $V^\prime$ and $V$.
\\
The idea of Galerkin approximation, is to find the solution in a family of subspaces of finite dimension, then prove that the constructed solutions converge to the solution of the variational problem Eq. \eqref{eq:variational-problem}. There are two major strategies, the first one is based on the coercivity and the other one on the \textit{inf-sup} conditions. While the coercivity is easy to use, unfortunately, most of problems in CFD do not fullfill it.
%
\begin{definition}[V-ellipcity or Coercivity]
 $a$ is said to be coercive, if there exists a constant $\alpha > 0$ such that
 \begin{align}
   a(v,v) \geq \alpha \| v \|^2_V, \quad \forall v \in V
   \label{eq:coercivity}
 \end{align}
\end{definition}
%
\begin{definition}[\textit{inf-sup} conditions]
  $a$ is said to statisfy the \textit{inf-sup} conditions, if there exists a constant $\alpha > 0$ such that
  \begin{enumerate}
    \item 
     \begin{align}
       \sup\limits_{v \in V} \frac{a(u,v)}{\| v \|_V} \geq \alpha \| u \|_V, \quad \forall u \in V
       \label{eq:infsup-1}
     \end{align}
    \item 
     \begin{align}
       \sup\limits_{u \in V} \frac{a(u,v)}{\| u \|_V} \geq \alpha \| v \|_V, \quad \forall v \in V
       \label{eq:infsup-2}
     \end{align}
  \end{enumerate}
\end{definition}
\begin{remark}
  Notice that when $a$ satisfies the \textit{inf-sup} conditions and it is symmetric, then both conditions are the same. In general, the two conditions can be written as
  \begin{align*}
         \inf\limits_{u \in V} \sup\limits_{v \in V} \frac{a(u,v)}{\| u \|_V \| v \|_V} > 0 
  \end{align*}
  and
  \begin{align*}
         \inf\limits_{v \in V} \sup\limits_{u \in V} \frac{a(u,v)}{\| u \|_V \| v \|_V} > 0 
  \end{align*}
\end{remark}
Finally, let us notice that the coercivity implies the \textit{inf-sup} conditions.
\begin{lemma}
  If $a$ is coercive then it satisfies the \textit{inf-sup} conditions. 
  \label{lemma:coer-to-infsup}
\end{lemma}
\begin{proof}
  We have 
  \begin{align*}
       \sup\limits_{v \in V} \frac{a(u,v)}{\| v \|_V} \geq \frac{a(u,u)}{\| u \|_V}
  \end{align*}
  then we conclude using the coercivity of $a$.
\end{proof}

% TODO add the part on the operators and their adjoints

\section{Galerkin Approximation}
We consider a family of finite dimensional subspaces of $V$, denoted by $\left( V_h \right)_{h > 0}$.
The Galerkin approximation $u_h \in V_h$ is defined as the solution of the varional problem Eq. \eqref{eq:variational-problem} by restricting the test functions on $V_h$, \textit{i.e.}
\begin{align}
  a(u_h,v) = \langle L, v \rangle, \quad \forall v \in V_h
  \label{eq:galerkin-approx}
\end{align}
It is important to notice that while coercivity is inherited on subspaces, the \textit{inf-sup} conditions are not. It is therefor important to have an additional \textit{inf-sup} condition on the subspace, known as \textbf{stability condition}.

\subsection{Convergence under coercivity}
We recall Cea's lemma, which states that the Galerkin approximation is bounded by the best approximation of $u$ from the subspace.
\begin{lemma}[Cea]
  If $a$ is a continuous and coercive bilinear form, then
  \begin{align}
    \| u-u_h \|_V \leq \frac{M}{\alpha} \inf\limits_{v \in V_h} \| u-v \|_V
    \label{eq:cea-ineq}
  \end{align}
  \label{lemma:cea}
\end{lemma}
\begin{proof}
  Since both $u$ and $u_h$ are solutions to the variational problem \eqref{eq:variational-problem} respectively on $V$ and $V_h$, we have 
  \begin{align*}
    a(u-u_h, v) = 0, \quad \forall v \in V_h
  \end{align*}
  therefor, for $v \in V_h$, we have
  \begin{align*}
    a(u-u_h, u-v) = a(u-u_h, u-u_h) + a(u-u_h, u_h-v) = a(u-u_h, u-u_h) 
  \end{align*}
  Using the coercivity, we have
  \begin{align*}
    \alpha \| u-u_h \|_V \leq a(u-u_h, u-u_h) = a(u-u_h, u-v) 
  \end{align*}
  finaly, the continuity of $a$ gives
  \begin{align*}
   a(u-u_h, u-v) \leq M \|u-u_h \|_V \|u-v \|_V 
  \end{align*}
  by combining the two previous inequalities we get the desired result.
\end{proof}
\begin{theorem}
  If $a$ is a continuous and coercive bilinear form and the subspaces $V_h$ are such that 
  \begin{align}
    \lim\limits_{h \rightarrow 0} d(u,V_h) = 0
    \label{eq:cvg-space-cond}
  \end{align}
  with $d(u,V_h) := \inf\limits_{v \in V_h} \| u-v \|_V$.
  Then 
  \begin{align*}
    \lim\limits_{h \rightarrow 0} u_h = u 
  \end{align*}
\end{theorem}
\begin{proof}
  Follows immedialty from Cea's lemma.
\end{proof}

\subsection{Convergence under \textit{inf-sup} conditions}
As mentioned before, the \textit{inf-sup} conditions are not inherited on the subspaces $V_h$. In general, we must prove that there exists $\beta > 0$, such that the \textit{inf-sup} holds on $V_h$, \textit{i.e.}
\begin{align}
   \sup\limits_{v \in V_h} \frac{a(u,v)}{\| v \|_V} \geq \beta \| u \|_V, \quad \forall u \in V_h
   \label{eq:infsup-stability}
\end{align}
Idealy, $\beta$ should be independent of $N$, in order to get the convergence.
\begin{exercise}
  Prove that the second part of the \textit{inf-sup} conditions is a consequence of the inequality \eqref{eq:infsup-stability}.
\end{exercise}
As in the coercive case, we first state a result that compares the Galerkin approximation with the distance to the space of approximation. This result is a generalization of Cea's lemma, and is due to Babuska.
% TODO add references
\begin{lemma}[Babuska]
  If $a$ is a continuous bilinear form and satisfies the \textit{inf-sup}+\textit{stability} conditions, then
  \begin{align}
    \| u-u_h \|_V \leq \left( 1 + \frac{M}{\beta} \right) \inf\limits_{v \in V_h} \| u-v \|_V
    \label{eq:babuska-ineq}
  \end{align}
  \label{lemma:babuska}
\end{lemma}
\begin{proof}
  Let $v \in V_h$. 
  since $v-u_h \in V_h$, the stability condition gives
  \begin{align*}
     \sup\limits_{\phi \in V_h} \frac{a(v-u_h,\phi)}{\| \phi \|_V} \geq \beta \| v-u_h \|_V
  \end{align*}
  but $a(u-u_h, v) = 0$, meaning $a(v-u_h, \phi) = a(v-u, \phi)$ for all $\phi \in V_h$. Now using the continuity of $a$ we get
  \begin{align*}
     M \| v-u \|_V \geq \sup\limits_{\phi \in V_h} \frac{a(v-u,\phi)}{\| \phi \|_V} \geq \beta \| v-u_h \|_V
  \end{align*}
  we conclude the proof by using the triangle inequality
  \begin{align*}
    \| u-u_h \|_V \leq \| u-v \|_V + \| v-u_h \|_V  \leq  \| u-v \|_V + \frac{M}{\beta}\| v-u \|_V 
  \end{align*}

\end{proof}
\begin{theorem}
  If $a$ is a continuous bilinear form and satisfies the \textit{inf-sup}+\textit{stability} conditions. If the subspaces $V_h$ are such that condition \eqref{eq:cvg-space-cond} holds, then 
  \begin{align*}
    \lim\limits_{h \rightarrow 0} u_h = u 
  \end{align*}
\end{theorem}
\begin{proof}
  Follows immedialty from Babuska's lemma.
\end{proof}
\begin{remark}
  Under the additional stability condition, we have $\| u_h \|_V \leq \frac{1}{\beta} \| L \|_{V^\prime}$, which is valid uniformuly in $N$ if $\beta$ is independent of $N$.
\end{remark}

\subsection{The three basic aspects of the Finite Elements method}
% TODO add some definitions
Let $\Omega \subset \mathbb{R}^d$, with $d \geq 1$, be a bounded domain.
\\
In order to apply the Galerkin method, we face, by definition the problem of constructing the family of finite dimensional subspaces $V_h \subset V$, such that $V$ is $H^1(\Omega)$, $H^1_0(\Omega)$, $H^2(\Omega)$, $H(\mbox{curl}, \Omega)$, \ldots
As stating by P. Ciarlet, the Finite Elements Method is in its simplest form, a specific process of constructing the family $\left( V_h \right)_{h \geq 0}$. This construction is characterized by three basic aspects and are described below.
\subsubsection*{First basic aspect: Triangulation}
A triangulation $\mathcal{T}_h$ is estabilshed over $\bar{\Omega}$, \textit{i.e.} $\bar{\Omega}$ is subdivided into a finite number of subsets $K$, called \textbf{finite elements},  such that
\begin{enumerate}
  \item  $\bar{\Omega} = \bigcup\limits_{K \in \mathcal{T}_h} K$ 
  \item  for all $K \in \mathcal{T}_h$, $K$ is closed and its interior is not empty 
  \item  for all $K_1 \neq K_2 \in \mathcal{T}_h$ we have $\mathring{K_1} \bigcap \mathring{K_2} = \emptyset$
  \item  for all $K \in \mathcal{T}_h$, $\partial K$ is Lipschitz-continuous  
\end{enumerate}
\subsubsection*{Second basic aspect: power approximation}
On every $K \in \mathcal{T}_h$, a space of functions $P_K$ is constructed. $P_K$ should contain polynomials or functions which are close to polynomials.
\begin{itemize}
  \item this is the key to all convergence results 
  \item it is also important for having simple and fast computations of the coefficients of the resulting linear system 
\end{itemize}
\subsubsection*{Third basic aspect: basis functions}
There exists at least one \textbf{canonical basis} in the space $V_h$ whose corresponding basis functions have a \textit{local support} property, are as small as possible and can be easily described.
This aspect leads to sparsity in the resulting matrix.

\newpage
\subsection{Examples}
\subsubsection*{Scalar linear elliptic equations of second order}
For $\Omega \subset \mathbb{R}^d$, we consider the following problem
\begin{align}
  \left\{ 
  \begin{array}{rl}
    - \sum\limits_{1 \leq i,j \leq d} \partial_{x_i} \left( a_{ij} \partial_{x_j} u \right) &= f, \quad \Omega 
    \\
    u &= 0, \quad \partial\Omega
  \end{array} \right.
  \label{eq:elliptic_pde}
\end{align}
where the coefficients functions $a_{ij} : \Omega \rightarrow \mathbb{R}$ are bounded and there exists $\gamma > 0$ (ellipticity condition) such that
\begin{align}
  \gamma | y |^2 \leq  \sum\limits_{1 \leq i,j \leq d} a_{ij}(x) y_i y_j, \quad \forall x \in \Omega, ~\forall y \in \mathbb{R}^d   
%  \label{}
\end{align}
Before writing the variational formulation associated to \eqref{eq:elliptic_pde}, we need to define the space of function $V$ which is in this case given by $V := H^1_0(\Omega)$ where
\begin{align}
  H^1_0(\Omega) = \{ u\in H^1(\Omega) , \;   u=0 \mbox{ on } \partial \Omega \}
  \label{eq:h1_0_space}
\end{align}
and
\begin{align}
  H^1(\Omega) = \{ u\in L^2(\Omega), \;  \nabla u\in (L^2(\Omega))^d \}
  \label{eq:h1_space}
\end{align}
$H^1_0(\Omega)$ is a Hilbert space under the norm $\| \cdot \|_{H^1_0(\Omega)}$ with
\begin{align}
  \| u \|_{H^1_0(\Omega)}^2 := \| u \|_{L^2(\Omega)}^2 + \| \nabla u \|_{L^2(\Omega)}^2 
  \label{eq:h1_0_norm}
\end{align}
the bilinear and linear forms are given by
\begin{align*}
  a(u,v) := \sum\limits_{1 \leq i,j \leq d} \int_{\Omega} a_{ij} \partial_{x_j} u \partial_{x_i} v ~ dx 
\end{align*}
and
\begin{align*}
  \langle L, v \rangle = \int_{\Omega} f v ~ dx
\end{align*}
Using the ellipticity condition, the boundness of the coefficients and the Poincar\'e inequality, we show that $a$ is coercive and continuous. Moreover, if $f \in L^2(\Omega)$ then $L$ is countinuous.

\subsubsection*{Biharmonic problem}
We consider the following problem
\begin{align}
  \left\{ 
  \begin{array}{rl}
    \triangle^2 u &= f, \quad \Omega 
    \\
    u &= 0, \quad \partial\Omega
    \\
    \nabla u \cdot \nn &= 0, \quad \partial\Omega
  \end{array} \right.
  \label{eq:biharmonic_pde}
\end{align}
which is known as the \textit{homogeneous Dirichlet problem for the biharmonic operator $\triangle$}.
The space of functions considered in this example $V$ is given by $V := H^2_0(\Omega)$ where
\begin{align}
  H^2_0(\Omega) = \{ u\in H^2(\Omega) , \;   u=\nabla u \cdot \nn=0 \mbox{ on } \partial \Omega \}
  \label{eq:h2_0_space}
\end{align}
and
\begin{align} % TODO
  H^2(\Omega) = \{ u\in L^2(\Omega), \;  \nabla u\in (L^2(\Omega))^d \}
  \label{eq:h2_space}
\end{align}
$H^2_0(\Omega)$ is a Hilbert space under the norm $v \mapsto | \triangle v |_{L^2(\Omega)}$ with
%\begin{align}
%  \| u \|_{H^1_0(\Omega)}^2 := \| u \|_{L^2(\Omega)}^2 + \| \nabla u \|_{L^2(\Omega)}^2 
%  \label{eq:h1_0_norm}
%\end{align}
the bilinear and linear forms are given by
\begin{align*}
  a(u,v) := \int_{\Omega} \triangle u \triangle v ~ dx 
\end{align*}
and
\begin{align*}
  \langle L, v \rangle = \int_{\Omega} f v ~ dx
\end{align*}

\subsubsection*{$\Hcurl$-elliptic problem}
Let $\Omega \subset \mathbb{R}^d$ be an open Liptschitz bounded set, and we look for the solution of the following problem
\begin{align}
  \left\{ 
  \begin{array}{rl}
    \Curl \Curl \uu + \mu \uu &= \ff, \quad \Omega 
    \\
    \uu \times \nn &= 0, \quad \partial\Omega
  \end{array} \right.
  \label{eq:elliptic_hcurl}
\end{align}
where $\ff \in \mathbf{L}^2(\Omega)$,  $\mu \in L^\infty(\Omega)$ and there exists $\mu_0 > 0$ such that $\mu \geq \mu_0$ almost everywhere.
We take the Hilbert space $V := \Hcurlzero$, in which case the variational formulation corresponding to \eqref{eq:elliptic_hcurl} writes 
\begin{tcolorbox}
  {\em Find $\uu \in V$ such that}
  \begin{align}
      a(\uu,\vv) = l(\vv) \quad \forall \vv \in V 
    \label{eq:abs_var_elliptic_hcurl}
  \end{align}
  where 
  \begin{align}
    \left\{ 
    \begin{array}{rll}
    a(\uu, \vv) &:= \int_{\Omega} \Curl \uu \cdot \Curl \vv + \int_{\Omega} \mu \uu \cdot \vv, & \forall \uu, \vv \in V  \\
    l(\vv) &:= \int_{\Omega} \vv \cdot \ff, & \forall \vv \in V  
    \end{array} \right.
  \end{align}
  \label{tcb:elliptic_hcurl}
\end{tcolorbox}

We recall that in $\Hcurlzero$, the bilinear form $a$ is equivalent to the inner product and is therefor continuous and coercive. Hence, our abstract theory applies and there exists a unique solution to the problem \eqref{eq:abs_var_elliptic_hcurl}.

\subsubsection*{$\Hdiv$-elliptic problem}
Let $\Omega \subset \mathbb{R}^d$ be an open Liptschitz bounded set, and we look for the solution of the following problem
\begin{align}
  \left\{ 
  \begin{array}{rl}
    - \Grad \Div \uu + \mu \uu &= \ff, \quad \Omega 
    \\
    \uu \times \nn &= 0, \quad \partial\Omega
  \end{array} \right.
  \label{eq:elliptic_hdiv}
\end{align}
where $\ff \in \mathbf{L}^2(\Omega)$,  $\mu \in L^\infty(\Omega)$ and there exists $\mu_0 > 0$ such that $\mu \geq \mu_0$ almost everywhere.
We take the Hilbert space $V := \Hdivzero$, in which case the variational formulation corresponding to \eqref{eq:elliptic_hdiv} writes 
\begin{tcolorbox}
  {\em Find $\uu \in V$ such that}
  \begin{align}
      a(\uu,\vv) = l(\vv) \quad \forall \vv \in V 
    \label{eq:abs_var_elliptic_hdiv}
  \end{align}
  where 
  \begin{align}
    \left\{ 
    \begin{array}{rll}
    a(\uu, \vv) &:= \int_{\Omega} \Div \uu ~ \Div \vv + \int_{\Omega} \mu \uu \cdot \vv, & \forall \uu, \vv \in V  \\
    l(\vv) &:= \int_{\Omega} \vv \cdot \ff, & \forall \vv \in V  
    \end{array} \right.
  \end{align}
\end{tcolorbox}
We recall that in $\Hdivzero$, the bilinear form $a$ is equivalent to the inner product and is therefor continuous and coercive. Hence, our abstract theory applies and there exists a unique solution to the problem \eqref{eq:abs_var_elliptic_hdiv}.

\newpage
\section{Saddle-point problems}

In the sequel, we consider a special case of the problem \eqref{eq:variational-problem}. 
\\
\noindent
Consider two Hilbert spaces $V$ and $W$, two continuous bilinear forms
$a\in \mathcal{L}(V\times V, \mathbb{R})$ and $b\in \mathcal{L}(V\times W, \mathbb{R})$
and two continuous linear forms $l_V\in \mathcal{L}(V, \mathbb{R})$ and
$l_W\in \mathcal{L}(W, \mathbb{R})$. We denote $M_a$ and $M_b$ the continuity constants for the bilinear forms $a$ and $b$ respectively.
Then we define the abstract mixed variational problem as {\em Find $(u,p)\in V\times W$ such that}
\begin{align}
  \left\{ 
  \begin{array}{cccccc}
    a(u,v) & + & b(v,p) &=& l_V(v) \quad \forall v\in V \\
           &   & b(u,q) &=& l_W(q) \quad \forall q\in W
  \end{array} \right.
  \label{eq:abs_var_mixed}
\end{align}

%\begin{align}
%a(u,v) + b(v,p) &= l_V(v) \quad \forall v\in V, \label{eq:abs_var_mixed1}\\
%b(u,q) &= l_W(q) \quad \forall q\in W.  \label{eq:abs_var_mixed2}
%\end{align}
Many problems arising in CFD fit into this abstract framework, such as the Stokes equation. For saddle point problems the Lax-Milgram framework cannot be applied. The alternative solution is then to use the \textbf{inf-sup} conditions, known in this case as  Banach-Ne\v{c}as-Babu\v{s}ka (BNB) theorem. 
\\
\noindent
The link with the previous section is achieved by using the bilinear form
$c \in \mathcal{L}(X\times X, \mathbb{R})$
\begin{align}
  c((u, p), (v, q)) := a(u,v) + b(v,p) + b(u,q) 
  \label{eq:abs_var_mixed_c}
\end{align}
and the linear form
$l_X \in \mathcal{L}(X, \mathbb{R})$
\begin{align}
  l_X(v, q) := l_V(v) + l_W(q)
  \label{eq:abs_var_mixed_l}
\end{align}
with $X := V \times W$ endowed with the norm $\| (u,p) \|_X := \| u \|_V + \| p \|_W$.
\\
\noindent
Let us introduce the operators $A: V \rightarrow V^\prime$ and $B: V \rightarrow W^\prime$ such that
\begin{align}
  \langle Au, v \rangle_{V^\prime, V} := a(u,v) \quad \forall (u,v) \in V \times V
%  \label{}
\end{align}
and
\begin{align}
  \langle Bu, p \rangle_{W^\prime, V} := b(u,p) \quad \forall (u,p) \in V \times W
%  \label{}
\end{align}
Since all Hilbert spaces are reflexive Banach spaces, we have $W^{\prime\prime} = W$. Hence we can define the following operator $B^T: W \rightarrow V^\prime$ such that 
\begin{align}
  \langle B^T p, u \rangle_{V^\prime, W} := b(u,p) \quad \forall (u,p) \in V \times W
%  \label{}
\end{align}
Therefor, the problem \eqref{eq:abs_var_mixed} is equivalent to 
{\em Find $(u,p)\in V\times W$ such that}
\begin{align}
  \left\{ 
  \begin{array}{cccccc}
    A u & + & B^T p &=& l_V \\
        &   & B u   &=& l_W
  \end{array} \right.
  \label{eq:abs_var_mixed_op}
\end{align}
Now, let us introduce the nullspace of $B$
\begin{align}
  \Ker{B} := \{ v \in V, ~\forall q \in W \quad b(v,q) = 0 \}  
%  \label{}
\end{align}
The following theorem gives shows under which conditions the saddle problem \eqref{eq:abs_var_mixed} has a solution.
\begin{theorem}
  The variational problem \eqref{eq:abs_var_mixed} admits a unique solution if and only if
\begin{itemize}
\item[1)] there exists $\alpha > 0$, such that 
  \begin{align}
    \begin{cases}
      \underset{u \in \Ker{B}}{\inf} ~ \underset{v \in \Ker{B}}{\sup}
      \frac{a(u,v)}{\| u \|_V \| v \|_V} \geq \alpha
      \\
      \forall v \in \Ker{B}, \quad \left( \forall u \in \Ker{B}, ~ a(u,v) = 0 \right) \Rightarrow (v = 0)
    \end{cases}
%    \label{}
  \end{align}

\item[2)] The Babuska-Brezzi, or inf-sup condition, is verified:  there exists $\beta>0$ such that
  \begin{align}
      \underset{q \in W}{\inf} ~ \underset{v \in V}{\sup}
      \frac{b(u,q)}{\| v \|_V \| q \|_W} \geq \beta
%    \label{}
  \end{align}
\end{itemize}
In addition, the following a priori estimates hold
\begin{align}
  \left\{ 
  \begin{array}{cl}
    \|u\|_V  \leq & \frac{1}{\alpha} \|l_V\|_{V'} + \frac{1}{\beta}(1+ \frac{M_a}{\alpha})  \|l_W\|_{W'}
    \\
    \|p\|_W  \leq & \frac{1}{\beta}(1+ \frac{M_a}{\alpha})  \|l_V\|_{V'} + \frac{M_a}{\beta^2}(1+ \frac{M_a}{\alpha})  \|l_W\|_{W'}
  \end{array} \right.
%  \label{}
\end{align}

%  \label{}
\end{theorem}
A special case is when the bilinear form $a$ is coercive. In this case, the first conditions can be replaced by a coercivity on $\Ker{B}$.

% In this case the appropriate theoretical tool, called in Ern and Guermond, which is more general than Lax-Milgram. We will specialise it to our type of mixed problems assuming that the bilinear form $a$ is symmetric and coercive on $K=\{v\in V\; | \; b(q,v)=0,~\forall q\in W\}$. This corresponds to Theorem 2.34 p.~100 of \cite{ern2013theory}, modified according to Remark 2.35.

\begin{theorem} Let $V$ and $W$ be Hilbert space. Assume $a$ is a continuous bilinear form on
$V\times V$ and that b is a continuous linear form on $V\times W$, that  $l_V$ and $L_W$ are continuous linear forms on $V$  and $W$ respectively and that the following two hypotheses are verified
\begin{itemize}
\item[1)]  $a$ is coercive on $K=\{v\in V\; |\; b(q,v)=0,~\forall q\in W\}$, \textit{i.e.} there exists $\alpha>0$ such that
$$a(v,v) \geq \alpha \|v\|^2_V \quad\forall v\in K.$$
\item[2)] The Babuska-Brezzi, or inf-sup condition, is verified:  there exists $\beta>0$ such that
$$ \inf_{q\in W} \sup_{v\in V} \frac{b(v,q)}{\|q\|_W\|v\|_V} \geq \beta.$$
\end{itemize}
Then the variational problem admits a unique solution and the solution satisfies the a priori estimate
\begin{align}
  \left\{ 
  \begin{array}{cl}
    \|u\|_V  \leq & \frac{1}{\alpha} \|l_V\|_{V'} + \frac{1}{\beta}(1+ \frac{M_a}{\alpha})  \|l_W\|_{W'}
    \\
    \|p\|_W  \leq & \frac{1}{\beta}(1+ \frac{M_a}{\alpha})  \|l_V\|_{V'} + \frac{M_a}{\beta^2}(1+ \frac{M_a}{\alpha})  \|l_W\|_{W'}
  \end{array} \right.
%  \label{}
\end{align}
\end{theorem}


The inf-sup conditions  plays an essential role, as it is only satisfied if the spaces $V$ and $W$ are compatible in some sense. This condition being satisfied at the discrete level  with a constant $\beta$ that does not depend on the mesh size being essential for a well behaved Finite Element method.  It can be written equivalently
\begin{equation}\label{inf-sup}
\beta \|q\|_W \leq \sup_{v\in V} \frac{b(v,q)}{\|v\|_V} \quad \forall q\in W.
\end{equation}
And often, a simple way to verify it is, given any  $q\in W$, to find a specific $v=v(q)$ depending on $q$ such that
$$\beta \|q\|_W \leq \frac{b(v(q),q)}{\|v(q)\|_V} \leq \sup_{v\in V} \frac{b(v,q)}{\|v\|_V}$$ with a constant $\beta$ independent of $w$.

\subsection{Examples}

\subsubsection*{First mixed formulation of the Poisson problem}
Let $\Omega \subset \mathbb{R}^3$ and consider the Poisson problem
\begin{align}
  \left\{ 
  \begin{array}{clr}
    -\Delta p & =f & ,~\Omega    \\
    p         & =0 & ,~\partial \Omega
  \end{array} \right.
  \label{eq:abs_poisson_mixed}
\end{align}
Using that $\Delta p = \nabla\cdot\nabla p$, we set $ \mathbf{u}=\nabla p$, then the Poisson equation \eqref{eq:abs_poisson_mixed} can be written equivalently
$$ \mathbf{u}=-\nabla p, ~~~ \nabla\cdot \mathbf{u}= f.$$
Instead of having one unknown, we now have two, along with the above two equations.
In order to get a mixed variational formulation, we first take the dot product of the first one by $ \mathbf{v}$ and integrate by parts
$$\int_{\Omega} \mathbf{u}\cdot \mathbf{v}\dd \mathbf{x} -\int_{\Omega} p\,\nabla\cdot \mathbf{v}\dd \mathbf{x} +
\int_{\partial\Omega} p \,  \mathbf{v}\cdot \mathbf{n}\dd \sigma=
\int_{\Omega} \mathbf{u}\cdot \mathbf{v}\dd \mathbf{x} -\int_{\Omega} p\,\nabla\cdot \mathbf{v}\dd \mathbf{x}=0,$$
using $p=0$ as a natural boundary condition. Then multiplying the second equation by $q$ and integrating yields
$$\int_{\Omega} \nabla\cdot\mathbf{u} \, q \dd \mathbf{x} = \int_{\Omega} f q \dd \mathbf{x}.
$$
No integration by parts is necessary here. And we thus get the following mixed variational formulation:
\begin{tcolorbox}
  {\em Find $(\mathbf{u},p) \in H(\operatorname{div},\Omega)\times L^2(\Omega)$ such that}
  \begin{equation}
    \left\{ 
    \begin{array}{llll}
      \int_{\Omega} \mathbf{u}\cdot \mathbf{v}\dd \mathbf{x} &- \int_{\Omega} p\,\nabla\cdot \mathbf{v}\dd \mathbf{x} &=0, & \forall \mathbf{v}\in H(\operatorname{div},\Omega) \\
      - \int_{\Omega} \nabla\cdot\mathbf{u} \, q \dd \mathbf{x} &  &= - \int_{\Omega} f q \dd \mathbf{x}, & \forall q\in L^2(\Omega)
    \end{array} \right.
    \label{eq:abs_var_mixed_poisson_1}
  \end{equation}
\end{tcolorbox}
%
\subsubsection*{Second mixed formulation of the Poisson problem}
Here, we get an alternative formulation by not integrating by parts, the mixed term in the first formulation but in the second. The first formulation simply becomes
$$\int_{\Omega} \mathbf{u}\cdot \mathbf{v}\dd \mathbf{x} +\int_{\Omega} \nabla p \cdot \mathbf{v}\dd \mathbf{x}=0,$$
and the second, removing immediately the boundary term due to the essential boundary condition $q=0$
$$\int_{\Omega}\nabla \cdot\mathbf{u}  \, q \dd \mathbf{x} =
 -\int_{\Omega}  \mathbf{u} \cdot \nabla q  \dd \mathbf{x} =
\int_{\Omega} f q \dd \mathbf{x},$$
which leads to the variational formulation
\begin{tcolorbox}
  {\em Find $(\mathbf{u},p) \in L^2(\Omega)^3 \times H^1_0(\Omega)$ such that}
  \begin{align}
    \left\{ 
    \begin{array}{llll}
      \int_{\Omega} \mathbf{u}\cdot \mathbf{v}\dd \mathbf{x} &+ \int_{\Omega} \nabla p \cdot \mathbf{v}\dd \mathbf{x} &=0, & \forall \mathbf{v}\in L^2(\Omega)^3 \\
      \int_{\Omega}  \mathbf{u} \cdot \nabla q  \dd \mathbf{x} & & = -\int_{\Omega} f q \dd \mathbf{x}, & \forall q\in H^1_0(\Omega)
    \end{array} \right.
    \label{eq:abs_var_mixed_poisson_2}
  \end{align}
\end{tcolorbox}
Note that this formulation actually contains the classical variational formulation for the Poisson equation. Indeed for $q\in H^1_0(\Omega)$, $\nabla q \in L^2(\Omega)^3$ can be used as a test function in the first equation. And plugging this into the second we get
$$\int_{\Omega}  \nabla p \cdot \nabla q  \dd \mathbf{x}  = \int_{\Omega} f q \dd \mathbf{x}, \quad \forall q\in H^1_0(\Omega).$$

%coercivity of $a$ trivial
%Inf-sup: take $p\in H_0^1$, $v=\nabla p\in L^2$, inf sup condition follows from Poincar\'e inequality.


\subsubsection*{First mixed formulation of the Stokes problem}
We consider now the Stokes problem for the steady-state modelling of an incompressible fluid
\begin{align}
  \left\{
    \begin{array}{rl}
      - \nabla^2 \uu + \nabla p = \ff & \mbox{in} ~ \Omega,  \\
      \Div \uu          = 0   & \mbox{in} ~ \Omega,  \\
      \uu               = 0   & \mbox{on} ~ \partial \Omega,
    \end{array}
    \right.
    \label{eq:stokes}
\end{align}
For the variational formulation, we take the dot product of the first equation with $v$ and integrate over the whole domain
$$\int_{\Omega} (-\Delta \mathbf{u} + \nabla p)\cdot \mathbf{v} \dd \mathbf{x} 
=\int_{\Omega}\nabla \mathbf{u}:\nabla \mathbf{v} \dd \mathbf{x} + \int_{\Omega} \nabla p \cdot \mathbf{v} \dd \mathbf{x}
= \int_{\Omega} \mathbf{f}\cdot \mathbf{v} \dd \mathbf{x}$$
The integration by parts is performed component by component. We impose the essential boundary condition $ \mathbf{v}=0$ on $\partial\Omega$, and we denote by
$$\int_{\Omega}\nabla \mathbf{u}:\nabla \mathbf{v} \dd \mathbf{x} =\sum_{i=1}^3 \int_{\Omega}\nabla u_i\cdot\nabla v_i \dd \mathbf{x} =\sum_{i,j=1}^3 \int_{\Omega}\partial_j u_i\partial_j v_i \dd \mathbf{x} .$$
We now need to deal with the constraint $\nabla\cdot \mathbf{u}=0$. The theoretical framework for saddle point problems requires that the corresponding bilinear form is the same as the second one appearing in the first part of the variational formulation. To this aim we multiply  $\nabla\cdot \mathbf{u}=0$ by a scalar test function (which will be associated to $p$) and integrate on the full domain, with an integration by parts in order to get the same bilinear form as in the first equation
$$\int_{\Omega} \nabla\cdot \mathbf{u} \,q \dd \mathbf{x}= - \int_{\Omega} \mathbf{u}\cdot \nabla q\dd \mathbf{x}=0,$$
using that $q=0$ on the boundary as an essential boundary condition.
We finally obtain the mixed variational formulation: 
\begin{tcolorbox}
  {\em Find $( \mathbf{u},p)\in H^1_0(\Omega)^3\times H^1_0(\Omega)$ such that }
  \begin{align}
    \left\{
      \begin{array}{llll}
        \int_{\Omega}\nabla \mathbf{u}:\nabla \mathbf{v} \dd \mathbf{x} &+ \int_{\Omega} \nabla p \cdot \mathbf{v} \dd \mathbf{x}
        &= \int_{\Omega} \mathbf{f}\cdot \mathbf{v} \dd \mathbf{x}, & \forall \mathbf{v}\in H_0^1(\Omega)^3 \\
        \int_{\Omega} \mathbf{u}\cdot \nabla q\dd \mathbf{x} & &=0, & \forall p\in H^1_0(\Omega)
      \end{array}
      \right.
  \label{eq:abs_var_mixed_stokes_1}
  \end{align}
\end{tcolorbox}
%
\subsubsection*{Second mixed formulation of the Stokes problem}
Another possibility to obtained a well posed variational formulation, is to integrate by parts the
$ \int_{\Omega} \nabla p \cdot \mathbf{v} \dd \mathbf{x}$ term in the first formulation:
$$ \int_{\Omega} \nabla p \cdot \mathbf{v} \dd \mathbf{x} = - \int_{\Omega} p \nabla \cdot \mathbf{v} \dd \mathbf{x} 
+ \int_{\partial\Omega} p \mathbf{v} \cdot \mathbf{n}\dd \sigma=
 -\int_{\Omega} p \,\nabla \cdot \mathbf{v} \dd \mathbf{x} ,$$
 using here $p=0$ as a natural boundary condition. Note that in the other variational formulation the same boundary condition was essential. In this case, for the second variational formulation, we just multiply $\nabla\cdot \mathbf{u}=0$ by $q$ and integrate. No integration by parts is needed in this case.
$$\int_{\Omega} \nabla \cdot \mathbf{u}\, q \dd \mathbf{x} =0.$$
This then leads to the following variational formulation:
\begin{tcolorbox}
  {\em Find $( \mathbf{u},p)\in H^1(\Omega)^3\times L^2(\Omega)$ such that }
  \begin{align}
    \left\{
      \begin{array}{llll}
        \int_{\Omega}\nabla \mathbf{u}:\nabla \mathbf{v} \dd \mathbf{x} &- \int_{\Omega}  p \, \nabla\cdot \mathbf{v} \dd \mathbf{x}
        &= \int_{\Omega} \mathbf{f}\cdot \mathbf{v} \dd \mathbf{x}, &\forall \mathbf{v}\in H^1(\Omega)^3
        \\
        \int_{\Omega}  \nabla\cdot\mathbf{u}\, q\dd \mathbf{x} & &=0,  &\forall q\in L^2(\Omega)
      \end{array}
      \right.
  \label{eq:abs_var_mixed_stokes_2}
  \end{align}
\end{tcolorbox}


\subsection{Galerkin approximation}
Let us now come to the Galerkin discretisation. The principle is simply to construct finite dimensional subspaces $W_h \subset W$ and $V_h\subset V$ and to write the variational formulation \eqref{eq:abs_var_mixed} replacing $W$ by $W_h$ and $V$ by $V_h$. 
The variational formulations are the same as in the continuous case, like for conforming finite elements. This automatically yields the consistency of the discrete formulation.
In order to get the stability property needed for convergence, we need that the coercivity constant $\alpha$ and the inf-sup constant $\beta$ are independent of $h$.

Because $V_h\subset V$ the coercivity property is automatically verified in the discrete case, with a coercivity constant that is the same as in the continuous case and hence does not depend on the discretisation parameter $h$.

Here, however, there is an additional difficulty, linked to the inf-sup conditions, which is completely dependent on the two spaces $V_h$ and $W_h$. By far not any conforming approximation of the two spaces will verify the discrete inf-sup condition with a constant $\beta$ that is independent on $h$. Finding compatible discrete spaces for a given mixed variational formulation, has been an active area of research. 

The variational problem for the Galerkin approximation is {\em Find $(u_h,p_h)\in V_h\times W_h$ such that}
\begin{align}
  \left\{ 
  \begin{array}{cccccc}
    a(u_h,v_h) & + & b(v_h,p_h) &=& l_V(v_h) \quad \forall v_h \in V_h \\
               &   & b(u_h,q_h) &=& l_W(q_h) \quad \forall q_h \in W_h 
  \end{array} \right.
  \label{eq:abs_var_mixed_galerkin}
\end{align}


%A nice framework to have this compatibility condition verified is the exact sequence of discrete spaces, thanks to which it is always possible to find for any $q$ a $v$ in the previous space such that $Av=q$.

Let us introduce the operator $B_h: V_h \rightarrow W_h^\prime$ such that
\begin{align}
  \langle B_h u_h, p_h \rangle_{W_h^\prime, V_h} := b(u_h,p_h) \quad \forall (u_h,p_h) \in V_h \times W_h 
%  \label{}
\end{align}
and its nullspace
\begin{align}
  \Ker{B_h} := \{ v_h \in V_h, ~\forall q_h \in W_h \quad b(v_h,q_h) = 0 \}  
%  \label{}
\end{align}

The following proposition states the conditions under which the Galerkin approximation of the problem \eqref{eq:abs_var_mixed_galerkin} admits a solution
\begin{proposition}
  The variational problem \eqref{eq:abs_var_mixed_galerkin} admits a unique solution if and only if
\begin{itemize}
\item[1)] there exists $\alpha_h > 0$, such that 
  \begin{align}
      \underset{u_h \in \Ker{B_h}}{\inf} ~ \underset{v_h \in \Ker{B_h}}{\sup}
      \frac{a(u_h,v_h)}{\| u_h \|_V \| v_h \|_V} \geq \alpha_h 
%    \label{}
  \end{align}

\item[2)] there exists $\beta_h>0$ such that
  \begin{align}
      \underset{q_h \in W_h}{\inf} ~ \underset{v_h \in V_h}{\sup}
      \frac{b(u_h,q_h)}{\| v_h \|_V \| q_h \|_W} \geq \beta_h 
%    \label{}
  \end{align}
\end{itemize}
  \label{prop:abs_var_mixed_galerkin_existence}
\end{proposition}
\begin{remark}
  The second condition is equivalent to assuming $B_h$ is surjective.
\end{remark}
Finally, we state the following lemma which is equivalent to Cea's lemma.
\begin{lemma}
  Under the assumptions of theorem \ref{prop:abs_var_mixed_galerkin_existence}, we have
  \begin{itemize}
    \item[a)] if $\Ker{B_h} \subset \Ker{B}$,
      \begin{align}
        \left\{ 
        \begin{array}{cl}
          \|u -u_h\|_V  \leq & \left( 1+\frac{ M_a}{\alpha_h} \right) \left( 1+\frac{ M_b}{\beta_h} \right) \inf\limits_{v_h \in V_h} \| u-v_h \|_{V}
          \\
          \|p -p_h\|_W  \leq & \frac{ M_a}{\beta_h} \left( 1+\frac{ M_a}{\alpha_h} \right) \left( 1+\frac{ M_b}{\beta_h} \right) \inf\limits_{v_h \in V_h} \| u-v_h \|_{V}
          \\
          & + \left( 1+\frac{ M_b}{\beta_h} \right) \inf\limits_{q_h \in W_h} \| p-q_h \|_{W}
        \end{array} \right.
      %  \label{}
      \end{align}

    \item[b)] otherwise,
      \begin{align}
        \left\{ 
        \begin{array}{cl}
          \|u -u_h\|_V  \leq & \left( 1+\frac{ M_a}{\alpha_h} \right) \left( 1+\frac{ M_b}{\beta_h} \right) \inf\limits_{v_h \in V_h} \| u-v_h \|_{V}
          \\
          & + \frac{ M_b}{\alpha_h} \inf\limits_{q_h \in W_h} \| p-q_h \|_{W}
          \\
          \|p -p_h\|_W  \leq & \frac{ M_a}{\beta_h} \left( 1+\frac{ M_a}{\alpha_h} \right) \left( 1+\frac{ M_b}{\beta_h} \right) \inf\limits_{v_h \in V_h} \| u-v_h \|_{V}
          \\
          & + \left( 1+\frac{ M_b}{\beta_h} + \frac{ M_a}{\alpha_h}\frac{ M_b}{\beta_h}  \right) \inf\limits_{q_h \in W_h} \| p-q_h \|_{W}
        \end{array} \right.
      %  \label{}
      \end{align}

  \end{itemize}
  
%  \label{}
\end{lemma}


\subsection{Examples}

\subsubsection*{Matrix form of the first mixed formulation of the Poisson problem}
%
Let $V_h$ and $W_h$ be subspaces of finite dimensions of $\Hdiv$ and $\Ltwo$ respectively, leading to a stable discretization of the variational problem \eqref{eq:abs_var_mixed_poisson_1}.
We shall assume that 
$$V_h = \mathrm{span}\{ \PsiPsi_{i}, ~ 1 \leq i \leq N_{V_h} \}$$ 
and
$$W_h = \mathrm{span}\{ \phi_{i}, ~ 1 \leq i \leq N_{W_h} \}$$ 
where $N_{V_h}$ and $N_{W_h}$ is the dimension of $V_h$ and $W_h$ respectively.
For $\uu_h \in V_h$ and $p_h \in W_h$, we can write
\begin{align*}
  \uu_h = \sum\limits_{j=1}^{N_{V_h}} u_{j} \PsiPsi_{j}, \quad 
  p_h = \sum\limits_{j=1}^{N_{W_h}} p_{j} \phi_{j} 
\end{align*}
By taking $\vv = \PsiPsi_{i}$ in the first equation of the variational formulation, we get
\begin{align*}
 \sum\limits_{j=1}^{N_{V_h}} u_{j} \int_{\Omega} \PsiPsi_{j}\cdot \PsiPsi_{i} \dd \mathbf{x} 
 - \sum\limits_{j=1}^{N_{W_h}} p_{j} \int_{\Omega} \phi_{j} \, \nabla \cdot \PsiPsi_{i} \dd \mathbf{x} =0, \quad \forall ~ 1 \leq i \leq N_{V_h}
\end{align*}
By taking $q = \phi_{i}$ in the second equation of the variational formulation, we get
\begin{align*}
 \sum\limits_{j=1}^{N_{\mathrm{div}}} u_{j} \int_{\Omega} \nabla\cdot\PsiPsi_{j} \, \phi_{i} \dd \mathbf{x} = \int_{\Omega} f \phi_{i} \dd \mathbf{x}, \quad \forall ~ 1 \leq i \leq N_{W_h} 
\end{align*}
\begin{tcolorbox}
  {\em Find $(U,P) \in \mathbb{R}^{N_{V_h}} \times \mathbb{R}^{N_{W_h}}$ such that}
  \begin{align}
    \begin{pmatrix}
      A   & B \\
      B^T & 0 
    \end{pmatrix}
    \begin{pmatrix} U \\ P \end{pmatrix}
    = \begin{pmatrix} 0 \\ F \end{pmatrix}
  %  \label{}
  \end{align}
  where the matrices $A$ and $B$ are given by
  \begin{align*}
    A_{i, j} &:=  \int_{\Omega} \PsiPsi_{j}\cdot \PsiPsi_{i} \dd \mathbf{x}  
    ,  \quad 1 \leq i, j \leq N_{V_h}
    \\
    B_{i, j} &:=  - \int_{\Omega} \phi_{j} \, \nabla \cdot \PsiPsi_{i} \dd \mathbf{x}  
    ,  \quad 1 \leq j \leq N_{V_h} 
    , \quad 1 \leq i \leq N_{W_h} 
    \\
    F_{i} &:= - \int_{\Omega} f \phi_{i} \dd \mathbf{x} 
    , \quad 1 \leq i \leq N_{W_h} 
  %  \label{}
  \end{align*}
  \label{tcb:mixed_poisson_1}
\end{tcolorbox}


\subsubsection*{Matrix form for the second mixed formulation of the Poisson problem}
%
Let $V_h$ and $W_h$ be subspaces of finite dimensions of $L^2(\Omega)^3$ and $H^1_0(\Omega)$ respectively, leading to a stable discretization of the variational problem \eqref{eq:abs_var_mixed_poisson_2}.
We shall assume that 
$$V_h = \mathrm{span}\{ \PsiPsi_{i}, ~ 1 \leq i \leq N_{V_h} \}$$ 
and
$$W_h = \mathrm{span}\{ \phi_{i}, ~ 1 \leq i \leq N_{W_h} \}$$ 
where $N_{V_h}$ and $N_{W_h}$ is the dimension of $V_h$ and $W_h$ respectively.
Following the same approach as before, we get the matrix form of our variational formulation
\begin{tcolorbox}
  {\em Find $(U,P) \in \mathbb{R}^{N_{V_h}} \times \mathbb{R}^{N_{W_h}}$ such that}
  \begin{align}
    \begin{pmatrix}
      A   & B \\
      B^T & 0 
    \end{pmatrix}
    \begin{pmatrix} U \\ P \end{pmatrix}
    = \begin{pmatrix} 0 \\ F \end{pmatrix}
  %  \label{}
  \end{align}
  where the matrices $A$ and $B$ are given by
  \begin{align*}
    A_{i, j} &:= \int_{\Omega} \PsiPsi_{j}\cdot \PsiPsi_{i} \dd \mathbf{x}  
    ,  \quad 1 \leq i, j \leq N_{V_h}
    \\
    B_{i, j} &:= \int_{\Omega} \nabla \phi_{j} \cdot \PsiPsi_{i} \dd \mathbf{x}  
    ,  \quad 1 \leq j \leq N_{V_h} 
    , \quad 1 \leq i \leq N_{W_h} 
    \\
    F_{i} &:= - \int_{\Omega} f \phi_{i} \dd \mathbf{x} 
    , \quad 1 \leq i \leq N_{W_h} 
  %  \label{}
  \end{align*}
  \label{tcb:mixed_poisson_2}
\end{tcolorbox}

\subsubsection*{First mixed formulation of the Stokes problem}
%
Let $V_h$ and $W_h$ be subspaces of finite dimensions of $H^1_0(\Omega)^3$ and $H^1_0(\Omega)$ respectively, leading to a stable discretization of the variational problem \eqref{eq:abs_var_mixed_stokes_1}.
We shall assume that 
$$V_h = \mathrm{span}\{ \PsiPsi_{i}, ~ 1 \leq i \leq N_{V_h} \}$$ 
and
$$W_h = \mathrm{span}\{ \phi_{i}, ~ 1 \leq i \leq N_{W_h} \}$$ 
where $N_{V_h}$ and $N_{W_h}$ is the dimension of $V_h$ and $W_h$ respectively.
Following the same approach as before, we get the matrix form of our variational formulation
\begin{tcolorbox}
  {\em Find $(U,P) \in \mathbb{R}^{N_{V_h}} \times \mathbb{R}^{N_{W_h}}$ such that}
  \begin{align}
    \begin{pmatrix}
      A   & B \\
      B^T & 0 
    \end{pmatrix}
    \begin{pmatrix} U \\ P \end{pmatrix}
    = \begin{pmatrix} F \\ 0 \end{pmatrix}
  %  \label{}
  \end{align}
  where the matrices $A$ and $B$ are given by
  \begin{align*}
    A_{i, j} &:= \int_{\Omega} \PsiPsi_{j} : \PsiPsi_{i} \dd \mathbf{x}  
    ,  \quad 1 \leq i, j \leq N_{V_h}
    \\
    B_{i, j} &:= \int_{\Omega} \nabla \phi_{j} \cdot \PsiPsi_{i} \dd \mathbf{x}  
    ,  \quad 1 \leq j \leq N_{V_h} 
    , \quad 1 \leq i \leq N_{W_h} 
    \\
    F_{i} &:= \int_{\Omega} \ff \cdot \PsiPsi_{i} \dd \mathbf{x} 
    , \quad 1 \leq i \leq N_{W_h} 
  %  \label{}
  \end{align*}
  \label{tcb:mixed_stokes_1}
\end{tcolorbox}

\subsubsection*{Second mixed formulation of the Stokes problem}
%
Let $V_h$ and $W_h$ be subspaces of finite dimensions of $H^1(\Omega)^3$ and $L^2(\Omega)$ respectively, leading to a stable discretization of the variational problem \eqref{eq:abs_var_mixed_stokes_2}.
We shall assume that 
$$V_h = \mathrm{span}\{ \PsiPsi_{i}, ~ 1 \leq i \leq N_{V_h} \}$$ 
and
$$W_h = \mathrm{span}\{ \phi_{i}, ~ 1 \leq i \leq N_{W_h} \}$$ 
where $N_{V_h}$ and $N_{W_h}$ is the dimension of $V_h$ and $W_h$ respectively.
Following the same approach as before, we get the matrix form of our variational formulation
\begin{tcolorbox}
  {\em Find $(U,P) \in \mathbb{R}^{N_{V_h}} \times \mathbb{R}^{N_{W_h}}$ such that}
  \begin{align}
    \begin{pmatrix}
      A   & B \\
      B^T & 0 
    \end{pmatrix}
    \begin{pmatrix} U \\ P \end{pmatrix}
    = \begin{pmatrix} F \\ 0 \end{pmatrix}
  %  \label{}
  \end{align}
  where the matrices $A$ and $B$ are given by
  \begin{align*}
    A_{i, j} &:= \int_{\Omega} \PsiPsi_{j} : \PsiPsi_{i} \dd \mathbf{x}  
    ,  \quad 1 \leq i, j \leq N_{V_h}
    \\
    B_{i, j} &:= - \int_{\Omega} \phi_{j} \nabla \cdot \PsiPsi_{i} \dd \mathbf{x}  
    ,  \quad 1 \leq j \leq N_{V_h} 
    , \quad 1 \leq i \leq N_{W_h} 
    \\
    F_{i} &:= \int_{\Omega} \ff \cdot \PsiPsi_{i} \dd \mathbf{x} 
    , \quad 1 \leq i \leq N_{W_h} 
  %  \label{}
  \end{align*}
  \label{tcb:mixed_stokes_2}
\end{tcolorbox}

% ...................................................................
\section{Problems}
\label{sec:fem-abstract-problems}
TODO



\input{tex/part_3_historical_notes.tex}

%----------------------------------------------------------------------------------------
%	PART 4: Finite Elements method 
%----------------------------------------------------------------------------------------
\part{Finite Elements method (TO REPLACE/UPDATE)}
% ...................................................................
\chapter{Finite Elements Method}
\section{Description of a Finite Element}

\subsection{Formal definition of a Finite Element}

Let $(K,P,\Sigma)$ be a triple such that

(i) $K$ is a closed subset of $\ \mathbb{R}^n$ of non empty interior,

(ii) $P$ is a finite dimensional vector space of functions defined on $K$,

(iii) $\Sigma$ is a set of linear forms on $P$ of finite cardinal
 $N$, $\Sigma=\{\sigma_1,\dots,\sigma_N\}$.

\begin{definition}
The elements of $\Sigma$ are called \emph{degrees of freedom} of the Finite Element.
\end{definition}
The degrees of freedom characterise the basis of $P$ associated to the Finite Element.

\begin{definition} 
  $\Sigma$ is said to be \textbf{$P$-unisolvent} if for any $N$-tuple
$(\alpha_1,\dots,\alpha_N)$, there  exists a unique element $p\in P$ such that
$\sigma_i(p)=\alpha_i$ pour $i=1,\dots, N$.
\end{definition} 
\begin{definition}
  The triple $(K,P,\Sigma)$ of $ \mathbb{R}^n$ is called \textbf{Finite Element} of $ \mathbb{R}^n$ if it satisfies
     (i), (ii) and (iii) and if $\Sigma$
  is $P$-unisolvent.
\end{definition}

In the definition of a Finite Element, $K$ is the domain on which the Finite Element is defined, $P$ is the finite dimensional approximation space, and $\Sigma$ uniquely defines a basis of $P$, which is needed to build the Finite Element matrices associated to the variational formulation applied to functions in the Finite Dimensional approximation space. This basis can be either defined through the degrees of freedom or directly by exhibiting an explicit formula for the basis functions. In this last case the degrees of freedom are not explicitly needed. 

The unisolvence property is needed to establish that the elements of $P$ characterised by the degrees fo freedom actually form a basis of $P$. In practice, this is done by first verifying that the dimension is right, \textit{i.e.} that the number of degrees of freedom is equal to $\dim P=N$ and then by checking either the injectivity or the surjectivity, which both imply the bijectivity if the dimension is right, of the mapping
\begin{align*}
P &\rightarrow \mathbb{R}^N\\
p &\mapsto (\sigma_1(p), \dots,  \sigma_N(p))
\end{align*}
This can be formalised in the two following lemmas:
\begin{lemma}\label{lemma:unisolvInj}
The set $\Sigma$ is $P$-unisolvent if and only if the two following properties are satisfied
\begin{itemize}
\item[(i)] $\dim P = |\Sigma|$ (where $|\Sigma|$ denotes the number of elements in the set $\Sigma$).
\item[(ii)] $\sigma_j(p)=0$ for $j=1, \dots, N$ $\Rightarrow p=0$.
\end{itemize}
\end{lemma}

\begin{lemma}\label{lemma:unisolvSurj}
The set $\Sigma$ is $P$-unisolvent if and only if the two following properties are satisfied
\begin{itemize}
\item[(i)] $\dim P = |\Sigma|=N$.
\item[(ii)] There exist $N$ linearly independent functions $p_i\in P, \,i=1, \dots, N$ such that $\sigma_j(p_i)=\delta_{ij}$.
\end{itemize}
\end{lemma}

In addition to its local definition, independent of its neighbours, the degrees of freedom of a Finite Element need to be chosen to allow a natural embedding into the function spaces in which the variational formulation is defined.
In practice for Finite Element spaces embedded in $L^2$ no continuity is required,   for Finite Element spaces embedded in $H^1$ $C^0$ continuity is required, for Finite Element spaces embedded in $H^2$ $C^1$ continuity is required, for Finite Element spaces embedded in $ H(\textrm{curl}, \Omega) $ $C^0$ continuity of the tangential component is required and for Finite Element spaces embedded in $ H(\textrm{div}, \Omega) $ $C^0$ continuity of the normal component is required. 

We will thus present examples of Finite Elements according to their conformity, \textit{i.e.} to the space in which they can be naturally embedded. Note that the continuity requirement is enforced by sharing the degrees of freedom on the interface between two elements and by making sure that these define uniquely the trace of $P$ on the interface.

\begin{remark}
Note that not all degrees of freedom on the interface need to be shared between the elements sharing the interface. This needs to be decided when constructing the global Finite Element space from the reference element.
\end{remark}

\subsection{$C^0$ Lagrange Finite Elements}

These are continuous Finite Elements, and the continuity will be enforced by sharing the degrees of freedom on the interface.   Thanks to this continuity property the Finite Element constructed from those will be included in $H^1$. These elements are called \emph{$H^1$ conforming}.

\paragraph{1D Lagrange $\mathbb{P}_k$ Element } Let $a,b\in \mathbb{R}$, $a<b$. Let $K=[a,b]$, 
$P= \mathbb{P}_k$ the set of polynomials of degree $k$ on $[a,b]$, $\Sigma=\{\sigma_0, \dots,
\sigma_k\}$, where $a= x_0 < x_1 < \dots < x_k= b$ are distinct points and
$$
\begin{array}{rcl}
\sigma_k:P&\rightarrow & \mathbb{R},\\
p&\mapsto &p(x_i).
\end{array}$$
Moreover $\Sigma$ is $P$-unisolvant, using Lemma \ref{lemma:unisolvSurj}. Indeed the Lagrange polynomials
at the interpolation points $x_0,x_1, \dots, x_k$ which read
\begin{equation}\label{eq:LagrangePol}
l_{k,i}(x)=\frac{\displaystyle \prod_{\substack{0\leq j\leq k \\ j\neq i}} (x -x_j)}{
\displaystyle \prod_{\substack{0\leq j\leq k \\ j\neq i}} (x_i -x_j)}
\end{equation}
satisfy 
 $l_{k,i}(x_j)=\delta_{i,j}$, $0\leq i,j \leq k$.
 
The fact that the  points  are all distinct makes the Lagrange interpolation problem well posed. The first and last point being on the boundary, will correspond to shared degrees of freedom. The corresponding basis function will be shared with the neighbouring element.

\includegraphics[width=.3\textwidth]{figures/part_4/linear_lagrange}
\includegraphics[width=.3\textwidth]{figures/part_4/quadratic_lagrange}
\includegraphics[width=.3\textwidth]{figures/part_4/cubic_lagrange}
 
For low degree polynomials, typically up to three, the degrees of freedom can be chosen uniformly space. For higher degree, the resulting matrices have better properties in particular conditioning if Gauss-Lobatto points are chosen. Note that high order Lagrange Finite Elements based on Gauss-Lobatto points are often called \emph{spectral elements}.

The extension from one to multiple dimensions is generally done in two ways. Either one builds finite elements based on simplices, or on tensor products of 1D elements. There are other possibilities but they are less classical. 

\paragraph{Simplex based $ \mathbb{P}_k$ Finite Elements.}
A simplex in $n$ dimensions is the convex hull of $n+1$ affine independent points. In 2D this is a non degenerate triangle in 3D a tetrahedron. In simplices basis functions are most easily expressed using the barycentric coordinates associated to the vertices of the simplex, that we shall denote by $ \mathbf{a}_0, \dots, \mathbf{a}_n \in \mathbb{R}^n$. The barycentric coordinate, classically denoted by $(\lambda_i)_{0\leq i\leq n}$, associated to the vertex $ \mathbf{a}_i$ is the affine function with value 1 at $ \mathbf{a}_i$ and 0 at the $n$ other vertices. More precisely, for any point $ \mathbf{x}=(x_1,\dots,x_n)^\top \in \mathbb{R}^{n}$,
\begin{equation}\label{eq:barcoord}
\lambda_i( \mathbf{x}) = \alpha_{i,0} + \sum_{l=1}^n \alpha_{i,l} x_l, ~~~ i=0,\dots,n.
\end{equation}
The coefficients $(\alpha_{i,l})_{0\leq l \leq n}$ being determined by the relations  $\lambda_i( \mathbf{a}_j)  =\delta_{ij}$.

\textit{Example: $\mathbb{P}_k$ in 2D.} For 3 non aligned points $ \mathbf{a}_1$, $ \mathbf{a}_2$,  and $ \mathbf{a}_3$ in $\mathbb{R}^2$ let $K$ be the triangle defined by $a_1$, $a_2$ and $a_3$. Let $P=\mathbb{P}_k$ be the vector space of polynomials of degree $k$ in 2D
$$ \mathbb{P}_k=\{\mathrm{Span}(x^\alpha y^\beta), ~~~~ (\alpha,\beta)\in \mathbb{N}^2, ~~~ 0\leq \alpha+\beta \leq k   \}.$$
The dimension of $ \mathbb{P}_k$ is $\frac{(k+1)(k+2)}{2}$. Putting the Lagrange interpolation points on a lattice with $k+1$ points on the edges and then one point less at each level starting from one edge, we get
$(k+1) + k + (k-1) + \dots +1 = (k+1)(k+2)/2$ interpolation points and so also degrees of freedom, which is the same as the dimension of the space (see Figure \ref{fig:FE} for an example with $k=3$). 
The basis functions can be expressed using the barycentric coordinates. In our example with $k=3$.
The three basis functions associated to the three vertices are
$(9/2)\lambda_i( \lambda_i -1/3)(\lambda_i-2/3)$,
the basis function associated  associated to the point closest to $a_i$ on the edge $[a_i,a_j]$ is
$(27/2)\lambda_i\lambda_j(\lambda_i-1/3)$. There are six basis functions of this form.
Finally the basis function associated to the center point is $27\lambda_1\lambda_2\lambda_3$.
We thus find 10 basis functions for $ \mathbb{P}_3$, whose dimension is 10.

\begin{figure}[ht]
\centerline{
\includegraphics[width=.3\textwidth]{figures/part_4/P3FE.pdf}~~~~~~~~ 
\includegraphics[width=.3\textwidth]{figures/part_4/Q2FE.pdf}}
\caption{\label{fig:FE} (Left) $\mathbb{P}_3$ Finite Element, (Right) $\mathbb{Q}_2$ Finite Element}
\end{figure}

\paragraph{Hypercube based $ \mathbb{Q}_k$ Finite Element.}
Let $K$ be the hypercube of dimension $n$ $K=\prod_{1\leq i \leq n} [a_i,b_i]$.
Let $P=\mathbb{Q}_k$ be the vector space of polynomials of degree $k$ in each direction:
$$ \mathbb{Q}_k=\{\mathrm{Span}(x_1^{\alpha_1} \dots x_n^{\alpha_n}), ~~~~ (\alpha_1,\dots,\alpha_n)\in \mathbb{N}^n, ~~~ 0\leq \alpha_1,\dots, \alpha_n \leq k   \}.$$
The degrees of freedom for this element is obtained by choosing $k+1$ distinct points $a_i=x_{i,0}< \dots < x_{i,k}=b_i$ in each direction.   The corresponding basis functions are the products in all directions of all the 1D Lagrange basis functions.
The Lagrange basis for $ \mathbb{Q}_k$ is $(l_{k,i_1}(x_1) \dots l_{k,i_n}(x_n))_{0\leq i_1,\dots,i_n \leq k}$, where the $l_{k,i_j}(x_j)$ denote the 1D Lagrange basis functions at the interpolation points $x_{j,0}, \dots, x_{j,k}$.
The dimension of $ \mathbb{Q}_k$ in $n$ dimensions is $(k+1)^n$, which is also the number of the basis functions.
An example with $k=2$ in 2D is given in Figure \ref{fig:FE}. Here the dimension of $ \mathbb{Q}_2$ is nine and the
nine basis functions are $l_{k,i}(x)l_{k,j}(y)$, $0\leq i,j\leq 2$. 


\subsection{$C^1$ Hermite Finite Elements}

Such elements are $H^2$ conforming and are needed when second order derivatives appear in the variational formulation. 

\paragraph{1D Hermite $ \mathbb{P}_{2k+1}$ Finite Element.} Hermite interpolation on an interval $[a,b]$ consists in finding a polynomial $p$ of degree $2k+1$ for $k \geq 1$, such that $p^{l}(a)$ and $p^l(b)$ are given for $0\leq l\leq k$. This will enforce $C^l$ continuity if all the interface degrees of freedom are shared.
To get $C^1$ continuity $k=1$ is enough:

\textit{Example $\mathbb{P}_3$ Hermite element in 1D.} This is the simplest possible Hermite Finite Element. 
Let $a,b\in \mathbb{R}$, $a<b$. Let $K=[a,b]$, 
$P= \mathbb{P}_3$ the linear space of cubic polynomials on $[a,b]$, $\Sigma=\{\sigma_1,
\sigma_2,\sigma_3, \sigma_4\}$, with
$$
\begin{array}{rcl}
\sigma_1:P&\rightarrow & \mathbb{R},\\
p&\mapsto &p(a),
\end{array}\qquad
\begin{array}{rcl}
\sigma_2:P&\rightarrow & \mathbb{R},\\
p&\mapsto &p(b),
\end{array}\qquad
\begin{array}{rcl}
\sigma_3:P&\rightarrow & \mathbb{R},\\
p&\mapsto &p'(a),
\end{array}
\begin{array}{rcl}
\sigma_4:P&\rightarrow & \mathbb{R},\\
p&\mapsto &p'(b).
\end{array}
$$

$C^1$ Finite Elements  are much more involved in higher dimensions at least for triangles and higher dimensional simplices than $C^0$ finite elements.  Tensor product constructions can be obtained from 1D Hermite Finite Elements, however they often have the constraint that the mapping also needs to be $C^1$, which is not the case for general affine mappings as are standardly used for $C^0$ Lagrange Finite Elements. Let us introduce the simplex examples on simplest triangle based and quad based $C^1$ finite elements.

\paragraph{Triangle $\mathbb{P}_3$ Hermite element.}

$K$ is a non degenerate triangle of vertices $ (\mathbf{a}_1, \mathbf{a}_2, \mathbf{a}_3)$, $P= \mathbf{P}_3$.
So $\dim P= 10$ and so ten degrees of freedom are needed. To enforce $C^1$ continuity, the value and the two components  of the gradient need to be fixed on each of the three vertices. This amounts to nine degrees of freedom, the tenth can be taken to be the value at the barycenter of the triangle.   This is represented on the left of Figure \ref{fig:C1FE}.

\paragraph{The Bogner-Fox-Schmitt rectangle $\mathbb{Q}_3$ Finite element.}
Taking the tensor product of two 1D $ \mathbf{P}_3$ Hermite Finite element. One obtains the Bogner-Fox-Schmitt element, which in addition to the point and two derivative values at each of the four vertices has also the cross derivative $\frac{\partial^2 p}{\partial x\partial y}$ as a degree of freedom. The corresponding basis consists of the products of all possible 1D cubic Hermite basis functions. Its dimension is $4\times 4=16.$
This is represented on the right of Figure \ref{fig:C1FE}.


\begin{figure}[ht]
\centerline{
\includegraphics[width=.25\textwidth]{figures/part_4/HermiteTriangle.pdf}~~~~~~~~ 
\includegraphics[width=.4\textwidth]{figures/part_4/BognerFoxSchmitt.pdf}}
\caption{\label{fig:C1FE} $C^1$ Finite Elements: (Left) $\mathbb{P}_3$ Hermite Finite Element, (Right) Bogner-Fox-Schmitt Finite Element}
\end{figure}


\subsubsection{Discontinuous finite elements}


Functions that are only in $L^2$ can be defined piecewise on each cell without any continuity requirements. 
Then there is no restriction on the degrees of freedom. 

Classical discontinuous Finite Elements are  \emph{nodal} Lagrange elements based on Gauss 
points as the quadrature points. Even though their approximation order is lower for the same number of points it is sometimes convenient to place the degrees of freedom at Gauss-Lobatto points, where one point on each side is on the edge. This makes easier the computation of edge integrals. However in this case, as the Finite Element is discontinuous, degrees of freedom on the interface are not shared by neighbouring elements. 

Another class of classical discontinuous Finite Elements are \emph{modal} elements where the basis functions are generally taken to be the orthonormal Legendre polynomials $(L_i)_{0\leq i \leq k}$, which have the advantage of  yielding the identity as a mass matrix. The corresponding degrees of freedom are
$\sigma_i(p) = \int p(x) L_i(x) \dd x.$ 
Such constructions, where each element of the basis has a deferent degree, are called hierarchical finite elements and facilitate what is called the $p$ refinement consisting in obtaining a more accurate approximation by taking a higher order polynomial rather that refining the grid. Indeed in this case when refining only one basis function needs to be added to the existing one rather than replacing all the basis functions as would be necessary for the previously seem Finite Elements where all basis functions have the same degree. 


\subsection{ $H(\textrm{div}, \Omega)$ conforming Finite Elements}

The $H(\textrm{div}, \Omega)$ space consists of vector fields. Each element as $n$ components in  $n$ dimensions. Moreover, as we have seen before $H(\textrm{div}, \Omega)$ vector fields have a continuous normal component and a discontinuous tangential component. Let us explain how such elements can be constructed in two dimensions. This can be generalised to higher dimensions. 

\paragraph{Tensor product construction.} Due to the given continuity requirements a $H(\textrm{div}, \Omega)$ conforming Finite Element can be constructed, by defining separately an approximation of the tangential component and of the normal component. 
In this case $K=[-1,1]\times [-1,1]$, 
$$ P = \{ \mathbf{p}=(p_x,p_y)^\top \,|\, p_x\in \mathbb{Q}_{k-1,k}, p_y\in \mathbb{Q}_{k,k-1}\},$$
where 
$$ \mathbb{Q}_{m,n}= \mathrm{Span} ( x^\alpha y^\beta,  ~ 0\leq \alpha\leq m, 0\leq \beta\leq n).$$
For $p_x$ the degrees of freedom are the Gauss points in $x$ and the Gauss-Lobatto points in $y$, such that $p_x$ is discontinuous in $x$ and continuous in $y$. It is the other way for $p_y$.
This enables arbitrarily high order   $H(\textrm{div}, \Omega)$ conforming elements.

\paragraph{The Raviart-Thomas (RT) Finite Element.}

$K$ is a non degenerate triangle of vertices $ (\mathbf{a}_1, \mathbf{a}_2, \mathbf{a}_3)$. For the element of order $k+1$, $k\geq 0$, denoting by $\bar{ \mathbb{P}}_k$ the space of homogeneous polynomials of degree $k$ (\textit{i.e.} all the monomials are exactly of degree $k$)
$$P = RT_k = (\mathbb{P}_k)^2 + 
\begin{pmatrix} x\\ y \end{pmatrix} \bar{ \mathbb{P}}_k.
$$
In two dimensions $\dim \mathbb{P}_k = (k+1)(k+2)/2$ and $\dim \bar{\mathbb{P}}_k = k+1$, so that
$\dim RT_k = (k+1)(k+3)$. This yields in particular $\dim RT_0= 3$, $\dim RT_1= 8$, $\dim RT_2= 15$.
The degrees of freedom are constructed starting from the edges in order to enforce the continuity requirements. This is our first example of moment based degrees of freedom, defined by 1D moments along the edges to enforce continuity and then 2D moments in the triangle to complete the missing degrees of freedom. 
$$\Sigma=\left\{ \mathbf{p}\mapsto\int_{e_i} \mathbf{p}\cdot \un s^l\dd s, 0\leq l\leq k,~~~  
\mathbf{p}\mapsto\int_K \mathbf{p}(x_1,x_2) \cdot x_i^l \dd x_i, ~~  0\leq l\leq k-1, ~i=1,2\right\}.  $$
We count here $k+1$ degrees of freedom for each of the three edges, and  $2k$ inner degrees of freedom,
which are in total $3(k+1)+ 2k(k+1)/2 = (k+1)(k+3)$ which is precisely the dimension of $RT_k$. To prove the unisolvence it is thus enough to prove that a polynomial which vanishes on all degrees of freedom is necessarily zero. Note that because the monomials in our degrees of freedom are a basis of the polynomial spaces
$ \mathbb{P}_k(e_i), ~i=1,2,3$ and  $ \mathbb{P}_{k-1}(K)$ respectively this is given by the following lemma:
\begin{lemma} Let $k\geq 0$, and $ \mathbf{p}\in RT_k(K)$ then
\begin{align}
\int_{e_i} \mathbf{p}\cdot \un \,r \dd s &=0, \quad \forall r\in \mathbb{P}_k(e_i), \label{eq:RTe}\\
\int_K \mathbf{p}  \cdot \mathbf{q} \dd \mathbf{x} &=0 \quad\forall \mathbf{q} \in (\mathbb{P}_{k-1})^2
\label{eq:RTf}
\end{align}
implies that $ \mathbf{p}=0$.
\end{lemma}

\begin{remark}
Note that instead of the monomials one can use \eqref{eq:RTe} and  \eqref{eq:RTf} applied to arbitrary bases of 
$\mathbb{P}_k(e_i)$ and $(\mathbb{P}_{k-1})^2$ respectively.
\end{remark}


\begin{proof}
Let $ \mathbf{p} \in RT_k$. Then $ \mathbf{p} = \mathbf{q} + \begin{pmatrix} x\\ y \end{pmatrix} \psi$
with $\mathbf{q}\in (\mathbb{P}_k)^2$ and $\psi\in \bar{ \mathbb{P}}_k$.
Let us first observe that on any of the three edges $e_i$ we have
  $ \mathbf{p}\cdot \un\in \mathbb{P}_k(e_i)$, this is obvioulsy the case for
$ \mathbf{p}\cdot \un$. On the other hand if $(x(s),y(s))$ defines a parametrisation of $e_i$, then
by definition of the normal $n_x (x(s)-x(s_0)) + n_y (y(s)-y(s_0)) = 0$ so that
$n_x x(s) + n_y y(s)$ is constant on $e_i$. It follows that on the edge $e_i$
$$ \un\begin{pmatrix} x\\ y \end{pmatrix} \psi= (n_x x + n_yy)\psi.$$
Then as $(n_x x + n_yy)$ is a constant, $n_x x + n_yy)\psi$ is a polynomial of degree $k$ like $\psi$.

Now  computing the divergence of  $\mathbf{p}$ we find
$$\nabla\cdot \mathbf{p} = \nabla\cdot \mathbf{q} + 2\psi + \begin{pmatrix} x\\ y \end{pmatrix} \cdot \nabla\psi
= \nabla\cdot \mathbf{q}  + (k+2)\psi. $$
Hence $ \nabla\cdot \mathbf{p}$ is a polynomial of degree $k$ orthogonal to all polynomials of degree k. Thus it is 0. Then because $\nabla\cdot \mathbf{q} \in \mathbb{P}_{k-1}$ it also follows that $\psi=0$ and hence
$\mathbf{p}\in (\mathbb{P}_k)^2$.


Let us take $\varphi\in \mathbb{P}_k$. Then $\nabla\varphi\in (\mathbb{P}_{k-1})^2$ and hence using
\eqref{eq:RTf} we get using a Green's formula
$$\int_K \mathbf{p}  \cdot \nabla\varphi \dd \mathbf{x} =0 = \int_{\partial K}  \mathbf{p}\cdot \un\varphi \, \dd s + \int_K \nabla\cdot \mathbf{p} \, \varphi \dd \mathbf{x}.$$
As the boundary term vanishes due to \eqref{eq:RTe}, it follows that $\int_K \nabla\cdot \mathbf{p} \, \varphi \dd \mathbf{x}=0$. This for all $\varphi\in \mathbb{P}_k$.


Now in order to conclude we assume that the reference triangle has one edge on $x=0$ and one edge on $y=0$ and is in the positive quarter plane.
The the condition $ \mathbf{p}\cdot \un=0$ yields that $p_x=0$ for $x=0$, hence $p_x= x\psi_1$ with
$\psi_1\in \mathbb{P}_{k-1}$ and $p_y=0$ for $y=0$, hence $p_y= y\psi_2$ with
$\psi_2\in \mathbb{P}_{k-1}$. Then taking $\mathbf{q}=(\psi_1,\psi_2)^\top$ in \eqref{eq:RTf}, we obtain
$$\int_K (x\psi_1^2 +y\psi_2^2) \dd \mathbf{x}=0$$
from which it follows as both terms are positive that they are both zero. So $ \mathbf{p}=0$ which was what we needed to prove.
\end{proof}

\begin{remark}
Obviously, the procedure used here to construct the Finite Element, by first setting the boundary degrees of freedom to ensure the required continuity and then complete with the needed inner degrees of freedom can also be applied to quads.
\end{remark}




\paragraph{The Brezzi-Douglas-Marini (BDM) Finite Element.}
$K$ is a non degenerate triangle of vertices $ (\mathbf{a}_1, \mathbf{a}_2, \mathbf{a}_3)$. 
Here $P$ is the full polynomial space in each direction $P = BDM_k = (\mathbb{P}_k)^2 $. As continuity of the normal component is needed on each of the three faces, these requires at least three degrees of freedom. So and $ H(\textrm{div}, \Omega)$ conforming $BDM_k$ element can only be constructed for $k\geq 1$ as $\dim BDM_0=2$. In order to define the degrees of freedom, we need the following space
$$N_k = (\mathbb{P}_k)^2 + \begin{pmatrix} y\\ -x \end{pmatrix} \bar{ \mathbb{P}}_k, $$
This space has obviously the same dimension as the corresponding $RT_k$: $\dim N_k=\dim RT_k=(k+1)(k+3)$.
We can now define the degrees of freedom for the $BDM_k$ element:
introducing $(\psi_j)_0\leq j\leq k$ a basis of $ \mathbb{P}_k$ on each edge (this could be the monomials $s^j$ as we used for Raviart-Thomas or any other basis), and $(\boldsymbol{\varphi}_j)_{0\leq (k-1)(k+1)}$ a basis of $N_{k-2}$
\begin{multline*}
\Sigma=\left\{ \mathbf{p}\mapsto\int_{e_i} \mathbf{p}\cdot \un \psi_j\dd s, 0\leq j\leq k,~i=1,2,3, \right.\\ \left.
\mathbf{p}\mapsto\int_K \mathbf{p} \cdot \boldsymbol{\varphi}_j \dd \mathbf{x} , ~~ 0\leq j\leq  (k-1)(k+1) \right\}.
\end{multline*}
We first notice, that the number of degrees of freedom is $3(k+1) + (k-1)(k+1)=(k+1)(k+2)=\dim P$. So that proving that  $ \mathbf{p}\in P$ must vanish if all these degrees of freedom are zero is enough to prove unisolvence. Similarly as for the Raviart-Thomas element this follows from the following:
\begin{lemma} Let $k\geq 1$, and $ \mathbf{p}\in \mathbb{P}_k(K)^2$ then
\begin{align}
\int_{e_i} \mathbf{p}\cdot \un \,\psi \dd s &=0, \quad \forall \psi\in \mathbb{P}_k(e_i), \label{eq:BDMe}\\
\int_K \mathbf{p}  \cdot \mathbf{q}\dd \mathbf{x} &=0 \quad\forall \mathbf{q}\in N_{k-2}
\label{eq:BDMf}
\end{align}
implies that $ \mathbf{p}=0$.
\end{lemma}

\begin{proof}
Let $ \mathbf{p}\in \mathbb{P}_k(K)^2$ verifying \eqref{eq:BDMe} and \eqref{eq:BDMf}. 
For any $\varphi\in \mathbb{P}_{k-1}$, we have that $\nabla\varphi\in (\mathbb{P}_{k-2})^2\subset N_{k-2}$, hence from
\eqref{eq:BDMf} it follows that 
$$\int_K \mathbf{p}  \cdot \nabla\varphi \dd \mathbf{x} =0 = -\int_K \nabla\cdot \mathbf{p}\, \varphi \dd x
+\int_{\partial K} \mathbf{p}\cdot \un \, \varphi\dd s. $$

The last term is zero because of \eqref{eq:BDMe}. Then $\nabla\cdot \mathbf{p}$ is a polynomial of degree $k-1$ orthogonal to all polynomials of degree $k-1$. Hence it is zero.
Thus $ \mathbf{p}= (\partial_y \psi, -\partial_x \psi)$ for a $\psi\in \mathbb{P}_{k+1}$. Moreover due
to  \eqref{eq:BDMe},  $ \mathbf{p}\cdot \un=0$ on each edge, this implies that the tangential derivative of $\psi$ on each edge, and so $\psi$ is constant on all the edges. As it is defined up to a constant, on can choose $\psi$ such that it vanishes on the three edges. Then introducing the barycentric coordinates $\lambda_1, \lambda_2, \lambda_3$ of the triangle, it follows that each $\lambda_i$ divides $\psi$, so that
$\psi=\lambda_1\lambda_2\lambda_3 \tilde{\psi}$ with $\tilde{\psi}\in \mathbb{P}_{k-2}$.

Let us now take $ \mathbf{q} =   \begin{pmatrix} y\\ -x \end{pmatrix} \tilde{\psi} \in N_{k-2}$. Then from 
\eqref{eq:BDMf} it follows that
$$\int_K \mathbf{p}  \cdot \mathbf{q}\dd \mathbf{x} = \int_K \begin{pmatrix} \partial_y \psi \\ -\partial_x \psi \end{pmatrix}\cdot \begin{pmatrix} y\\ -x \end{pmatrix} \tilde{\psi}\dd \mathbf{x}=
\int_K k\lambda_1\lambda_2\lambda_3 \tilde{\psi}^2 \dd x=0$$
which implies that $ \tilde{\psi}=0$ and thus $ \mathbf{p}=0$, which is the desired result.
\end{proof}

\begin{figure}[ht]
\centerline{
\includegraphics[width=.3\textwidth]{figures/part_4/RT1.pdf}~~~~~~~~ 
\includegraphics[width=.3\textwidth]{figures/part_4/BDM2.pdf}}
\caption{\label{fig:HdivFE} $H(div)$ Finite Elements: (Left) $RT_1$ Raviart-Thomas Element, (Right) $BDM_2$ Brezzi-Douglas-Marini Element}
\end{figure}

\begin{remark}
The extension to 3D is straightforward, the degrees of freedom stay the same, the triangles being replaced by tetrahedra and the edges being replaced by faces.
\end{remark}


\subsection{ $H(\textrm{curl}, \Omega)$ conforming Finite Elements}

Recall that for a 3D vector $\mathbf{u}=(u_1,u_2,u_3)^\top$, the curl is defined by the relation
$$ \nabla\times \mathbf{u} = \begin{pmatrix}
\partial_2 u_3 - \partial_3 u_2 \\ \partial_3 u_1 - \partial_1 u_3 \\ \partial_1 u_2 - \partial_2 u_1 
\end{pmatrix}.$$
In 2D, there is no dependency on the third coordinate and the curl degenerates into the vector curl of a scalar, for the first two components and the scalar curl of a vector for the last component:
$$ \nabla\times \mathbf{u} =
  \begin{pmatrix}
  \mathbf{curl}\, u_3 \\  \mathrm{curl}\, \underline{\mathbf{u}}
  \end{pmatrix}~~~~
  \mbox{ with }  
  \mathbf{curl}\, u_3 =  \begin{pmatrix}
\partial_2 u_3  \\  - \partial_1 u_3
\end{pmatrix}, ~~
\underline{\mathbf{u}} =  \begin{pmatrix}
u_1  \\  u_2
\end{pmatrix}, ~~
\mathrm{curl}\, \underline{\mathbf{u}} = \partial_1 u_2 - \partial_2 u_1.
$$

$ H(\textrm{curl}, \Omega)$ conforming elements have a continuous tangential component. They can thus be simply obtained in 2D by exchanging the two components of the vector in the tensor product case.

For triangles, the natural finite element space, built as the Raviart-Thomas element is the N\'ed\'elec Element defined by $K=( \mathbf{a}_1, \mathbf{a}_2, \mathbf{a}_3)$, a non degenerate triangle,
$$P=N_k = (\mathbb{P}_k)^2 + \begin{pmatrix} y\\ -x \end{pmatrix} \bar{ \mathbb{P}}_k, $$
and the degrees of freedom are obtained from the Raviart-Thomas degrees of freedom by replacing the normal  $\un=(\nu_1,\nu_2)^\top$ by the tangent $ \ut=(\nu_2,-\nu_1)$. 
Unisolvence is also easily adapted from the Raviart-Thomas case and is based on the following 
\begin{lemma} Let $k\geq 0$, and $ \mathbf{p}\in RT_k(K)$ then
\begin{align}
\int_{e_i} \mathbf{p}\cdot \ut \,r \dd s &=0, \quad \forall r\in \mathbb{P}_k(e_i), \label{eq:Ne}\\
\int_K \mathbf{p}  \cdot \mathbf{q} \dd \mathbf{x} &=0 \quad\forall \mathbf{q} \in (\mathbb{P}_{k-1})^2
\label{eq:Nf}
\end{align}
implies that $ \mathbf{p}=0$.
\end{lemma}


\begin{figure}[ht]
\centerline{
\includegraphics[width=.3\textwidth]{figures/part_4/Ned0.pdf}~~~~~~~~ 
\includegraphics[width=.3\textwidth]{figures/part_4/Ned1.pdf}}
\caption{\label{fig:HcurlFE} $H(curl)$ Finite Elements: (Left)  N\'ed\'elec Element $N_0$ of order 1, (Right)  Brezzi-N\'ed\'elec Element $N_1$ of order 2}
\end{figure}

\begin{remark}
The extension to 3D for tensor product element is also quite natural by considering the components separetely and noticing that in 3D there are two tangential components and three normal components.

Simplex based $H(\textrm{curl}, \Omega)$ Finite Elements, which are constructed on tetrahedra in 3D are fundamentally different from   their $H(\textrm{div}, \Omega)$ counterpart, unlike in 2D where one could go from one to the other by a simple rotation of the components. There are also two classes of finite elements, like Raviart-Thomas and Brezzi-Douglas-Marini in 2D, which are called respectively N\'ed\'elec elements of first type and of second type.

The 3D N\'ed\'elec space in which the elements of first type live and which is used to define the degrees of freedom of the N\'ed\'elec elements of second type reads
\begin{equation}\label{nedelec3D}
N_k= ( \mathbb{P}_k)^3 \oplus \left( \mathbf{x}\times ( \mathbb{P}_k)^3 \right)
\end{equation}
and the degrees of freedom are defined by the following unisolvence lemma
\begin{lemma} Let $K$ be a non degenerate tetrahedron, with faces $(f_i)_{0\leq i \leq 4}$, and edges
$(e_i)_{0\leq i \leq 6}$
$k\geq 0$, and $ \mathbf{p}\in N_k(K)$ then
\begin{align}
\int_{e_i} \mathbf{p}\cdot \ut_i \,r \dd s &=0, \quad \forall r\in \mathbb{P}_k(e_i), \label{eq:Ne3d}\\
\int_{f_i} (\mathbf{p}\times \un_i) \cdot \mathbf{s} \dd s &=0, \quad \forall  \mathbf{s}\in (\mathbb{P}_{k-1}(f_i))^2, \label{eq:Nf1}\\
\int_K \mathbf{p}  \cdot \mathbf{q} \dd \mathbf{x} &=0 \quad\forall \mathbf{q} \in (\mathbb{P}_{k-2})^2
\label{eq:Nt}
\end{align}
implies that $ \mathbf{p}=0$.
\end{lemma}
We denote here by $\ut_i$ the unit vector along the edge $e_i$ and by $ \un_i$ the outward unit normal on face $f_i$.
\end{remark}

\section{Change of local basis}

The local degrees of freedom $\Sigma$ of a Finite Element defined by $(K,P,\Sigma)$ can also be used to compute the element matrices with respect to the matrices corresponding to a simpler or easily computable basis of the same space $P$ of dimension $N$. This could for example be an orthonormal basis.
This allows to compute the matrices once for all in some basis and then to get the matrices in any other basis by just computing the corresponding generalised Vandermode matrix as follows.

Let us denote by $(\phi_l)_{0\leq l\leq N}$ the reference basis in which the element matrices are known. 
Let now $(\hat{p}_j)_{0\leq j\leq N}$ be another basis of the same linear space $P$ associated to the degrees of freedom $(\sigma_i)_{0\leq i\leq N}$.
Then $\hat{p}_j$ can be expressed in the basis $(\phi_l)_{0\leq l\leq N}$:
\begin{equation}\label{eq:pphi}
\hat{p}_j(\mathbf{x}) = \sum_{l=1}^{N} \alpha_{j,l} \phi_l(\mathbf{x}).
\end{equation}
Denoting by $\hat{\mathbf{p}}=(\hat{p}_1,\dots,\hat{p}_{N})^\top$, $\boldsymbol{\phi}=(\phi_1,\dots,\phi_{N})^\top$ and $A=((\alpha_{j,l}))_{1\leq j,l \leq N}$, this relation can be written in matrix form
$$
\hat{\mathbf{p}}(\mathbf{x}) = A \boldsymbol{\phi}(\mathbf{x}).
$$
Applying the linear forms $\sigma_i$ to relation  \eqref{eq:pphi} for $1\leq i\leq N$, we get 
\begin{equation}\label{av}
\mathbb{I}_{N} = VA^\top,
\end{equation}
where $ \mathbb{I}_{N}$ is the identity matrix and $V=((\sigma_i(\phi_j)))_{1\leq i,j \leq N}$. The matrix $V$ is called generalised Vandermonde matrix, as in the case when  $((\phi_j))_{1\leq j \leq N}$ is the monomial basis $(1, x, \dots, x^{N-1})$ of $ \mathbb{P}_k$ in
1D and for a Lagrange Finite Element where $\sigma_i(\hat{p}_j)= \hat{p}_j(x_i)$, the matrix $V$ is the classical Vandermonde matrix $V=((x_i^{j-1}))_{1\leq i,j \leq N}$.

The generalised  Vandermonde matrix $V$ is explicitly  computable when the basis $(\phi_j)_j$ and the degrees of freedom $(\sigma_i)_i$ are explicitly known. 
From relation (\ref{av}), it follows immediately that $A=V^{-\top}$, where we denote by $V^{-\top}$
the inverse of the transpose of $V$.

One can also take the derivative of formula \eqref{eq:pphi}. Thus
$$\partial_{x}\hat{p}_j(\mathbf{x}) = \sum_{l=1}^{N} \alpha_{j,l} \partial_x\phi_l(\mathbf{x}),$$
and the same for the other variables if needed.
Applying again the linear form $\sigma_i$ to this equation for $1\leq i\leq N$,  we find
$$D_x\hat{\mathbf{p}} = D_x\boldsymbol{\phi}A^\top=D_x\boldsymbol{\phi}V^{-1} ,$$
where the matrices $D_x\hat{\mathbf{p}}$ and  $D_x\boldsymbol{\phi}$ are defined respectively
by their generic element $((\sigma_i(\partial_x \hat{p}_j)))_{1\leq i,j \leq N}$ 
and $((\sigma_i(\partial_x \phi_j)))_{1\leq i,j \leq N_k}$. Note that if
$(\phi_j)_j$ has been chosen such $D_x\phi$ is explicitly computable, we can, thanks to this relation explicitly compute the matrix $D_x\hat{\mathbf{p}}$ and similarly the derivative matrices with respect to other variables.

We can express $\hat{M}$ the mass matrix on the reference element  $\hat{K}$ using the Vandermonde matrix $V$. Using formula \eqref{eq:pphi}, 
$$\hat{p}_j(\mathbf{x})=  \sum_{l=1}^{N} \alpha_{j,l} \phi_l(\mathbf{x}).$$
Hence $$\int_{\hat{K}} \hat{p}_i(\mathbf{x})\hat{p}_j(\mathbf{x})\,d\mathbf{x}
=  \sum_{l=1}^{N} \sum_{m=1}^{N}\alpha_{i,l}\alpha_{j,m} \int_{\hat{K}}\phi_l(\mathbf{x})\phi_m(\mathbf{x})\,d\mathbf{x} = \sum_{l=1}^{N}\alpha_{i,l}\alpha_{j,m},$$
if the basis functions $(\phi_j)_j$ are  orthonormal on $\hat{K}$. It follows that
$\hat{M}=AA^\top = V^{-\top}V^{-1}$.

If the space $P$ is stable by derivation,
as is the case for the polynomial space $\mathbb{P}_k$, we can also express the derivative 
 $\partial_x G_h$ or other derivatives in the  basis  
 $(p_i)_i$ and this derivative can also be characterised by the degrees of freedom 
$(\sigma_i(\partial_x G_h))_{1\leq i \leq N_k}$. We denote by $D_x\mathbb{G}$
the vector containing these degrees of freedom, which can be expressed using $\mathbb{G}$
as follows. We have
$$\partial_x G_h(\mathbf{x}) = \sum_{j=1}^{N_k} \sigma_j(G_h)\partial_x p_j(\mathbf{x}),$$
and so for $1\leq i\leq N$,
$$\sigma_i(\partial_x G_h) = \sum_{j=1}^{N_k} \sigma_j(G_h)\sigma_i(\partial_x p_j).$$
This can also be written $D_x\mathbb{G}= (D_x\mathbf{p})\mathbb{G}$,
denoting by $D_x\mathbf{p}$ the matrix with generic term $((\sigma_i(\partial_x p_j)))_{1\leq i,j \leq N}$.

\section{Problems}

\begin{exercise}
  TODO
\end{exercise}

% ...................................................................
\chapter{TODO}
\section{TODO}
\section{TODO}
\section{TODO}


\section{Problems}

\begin{exercise}
  TODO
\end{exercise}




\chapter{Historical Notes}
\label{ch:approximation-historical-notes}

TODO



%----------------------------------------------------------------------------------------
%	PART 5: Linear Algebra 
%----------------------------------------------------------------------------------------
\part{Linear Algebra}
\chapter{Kronecker Algebra}
\label{ch:linalg-kronecker}

\section{Kronecker algebra}
\label{sec:produit_kronecker_sec}
In this section, we present an overview about an interesting subject, which is the Kronecker Algebra, and which will be of a big interest in the Fast-IGA approach. Most of the presented results were taken from \cite{Graham_book,Bernstein_book}.

\begin{definition}[The $\mathbf{vec}$ operator]
Let $A=(a_{ij}) \in \mathcal{M}_{n \times m}$, the $\mathbf{vec}$ operator is defined as,
\begin{align}
\mathbf{vec} A = \left(\begin{array}{c}
 A_{:,1}
\\
\vdots
\\
 A_{:,m}
\end{array}\right) \in \mathbb{R}^{mn}
\end{align}
which is simply a vector composed by stacking all the columns of $A$. Where we denote $ A_{:,j}$ the $j^{th}$ column of $A$.
\\
We also define the inverse operator of $\mathbf{vec}$ by,
\begin{align}
A = \mathbf{vec}^{-1} \mathbf{vec} A
\end{align}
\end{definition}

\begin{definition}[Kronecker product]
Let $A=(a_{ij}) \in \mathcal{M}_{m \times n}$ and $B=(b_{ij}) \in \mathcal{M}_{r \times s}$ be two matrices. The Kronecker product of $A$ and $B$, denoted by $A \otimes B  \in \mathcal{M}_{mr \times ns}$, defines the following matrix:
\begin{align}
A \otimes B = 
\left(\begin{array}{cccc}
a_{11}B & a_{12}B & \cdots & a_{1n}B 
\\
a_{21}B & a_{22}B & \cdots & a_{2n}B  
\\
\vdots & \vdots &  & \vdots 
\\
a_{m1}B & a_{m2}B & \cdots & a_{mn}B 
\end{array}\right)
\end{align}
\end{definition}

\subsubsection{Example}
Let 
\begin{align*}
A = 
\left(\begin{array}{cc}
a_{11} & a_{12}
\\
a_{21} & a_{22}
\end{array}\right),~~~
B = 
\left(\begin{array}{cc}
b_{11} & b_{12}
\\
b_{21} & b_{22}
\end{array}\right)
\end{align*}
then their Kronecker product is,
\begin{align}
A \otimes B = 
\left(\begin{array}{cccc}
a_{11}b_{11} & a_{11}b_{12} & a_{12}b_{11} & a_{12}b_{12}
\\
a_{11}b_{21} & a_{11}b_{22} & a_{12}b_{21} & a_{12}b_{22}
\\
a_{21}b_{11} & a_{21}b_{12} & a_{22}b_{11} & a_{22}b_{12}
\\
a_{21}b_{21} & a_{21}b_{22} & a_{22}b_{21} & a_{22}b_{22}
\end{array}\right)
\end{align}


\subsubsection{Properties}
\begin{proposition}
If $\alpha$ is a scalar, then 
\begin{align}
A \otimes \alpha B = \alpha A \otimes B
\end{align}
\end{proposition}

\begin{proposition}
We have,
\begin{align}
( A + B ) \otimes C &= A \otimes C + B \otimes C
\\
A \otimes ( B + C ) &= A \otimes B + A \otimes C
\end{align}
\end{proposition}

\begin{proposition}[Associativity]
\begin{align}
A \otimes B \otimes C = A \otimes ( B \otimes C ) = ( A \otimes B ) \otimes C
\end{align}
\end{proposition}

\begin{proposition}[Mixed Product Rule]
\begin{align}
( A \otimes B ) ( C \otimes D ) = AC \otimes BD
\end{align}
and,
\begin{align}
( A \otimes B ) ^p = A^p \otimes B^p,~~~\forall p \in \mathbb{N}
\end{align}
\end{proposition}

\begin{proposition}
\begin{align}
( A \otimes B )^T = A^T \otimes B^T
\end{align}
\end{proposition}

\begin{proposition}
\begin{align}
( A \otimes B )^{-1} = A^{-1} \otimes B^{-1}
\end{align}
\end{proposition}

\begin{proposition}
\begin{align}
\label{kronecker_vec_abc}
\mathbf{vec} ( ABC ) = ( C^T \otimes A ) \mathbf{vec} ( B )
\end{align}
\end{proposition}

\begin{proposition}
\begin{align}
\mathbf{tr} ( A \otimes B ) = \mathbf{tr} ( B \otimes A ) =  \mathbf{tr} ( A ) \mathbf{tr} ( B )
\end{align}
\end{proposition}

\begin{proposition}
Let $A \in \mathcal{M}_{n \times n}$ and $B \in \mathcal{M}_{m \times m}$, we have,
\begin{align}
\label{kronecker_prod_spec}
\mathbf{mspec} ( A \otimes B ) = \{ \lambda \mu,~~\lambda \in \mathbf{mspec}(A),~\mu \in \mathbf{mspec}(B) \}
\end{align}
\end{proposition}

\begin{proposition}
Let $A \in \mathcal{M}_{n \times n}$ and $B \in \mathcal{M}_{m \times m}$, we have,
\begin{align}
\mathbf{det} ( A \otimes B ) = ( \mathbf{det} A )^m ( \mathbf{det} B )^n
\end{align}
\end{proposition}
We deduce from \ref{kronecker_prod_spec},
\begin{proposition}
Let $A \in \mathcal{M}_{n \times n}$, we have,
\begin{align}
\rho ( A \otimes A ) = \rho ( A )^2
\end{align}
\end{proposition}

\begin{proposition}
Let $f$ be an analytic function, $A \in \mathcal{M}_{n \times n}$ such that $f(A)$ exists, then we have,
\begin{align}
f(I_m \otimes A) = I_m \otimes f(A)
\\
f( A \otimes I_m ) = f(A) \otimes I_m 
\end{align}
\end{proposition}

\begin{proposition}
Let $X \in \mathbb{R}^{n}$ and $Y \in \mathbb{R}^{m}$, be two vectors. We have,
\begin{align}
X Y^T = X \otimes (Y^T) = (Y^T) \otimes X
\end{align}
moreover, we have,
\begin{align}
\mathbf{vec} (XY^T) = Y \otimes X 
\end{align}
\end{proposition}

\begin{definition}[Kronecker permutation matrix]
The Kronecker permutation matrix $P_{n,m} \in  \mathcal{M}_{nm \times nm}$, is defined by,
\begin{align}
P_{n,m} = \sum_{i,j=1}^{n,m} E_{i,j,n \times m} \otimes E_{j,i,m \times n}
\end{align}
\end{definition}

\begin{proposition}
Let $A \in \mathcal{M}_{m \times n}$, we have,
\begin{align}
\mathbf{vec} (A^T) = P_{m,n} \mathbf{vec} (A) 
\end{align}
\end{proposition}

\begin{proposition}
Let us consider the Kronecker permutation matrix $P_{n,m} \in  \mathcal{M}_{nm \times nm}$. Then we have,
\begin{itemize}
\item $P_{n,m}^T=P_{n,m}^{-1}=P_{m,n}$
\item $P_{n,m}$ is orthogonal,
\item $P_{n,m} P_{m,n} =I_{nm}$
\item $P_{n,n}$ is orthogonal, symmetric and involutory,
\item $P_{n,n}$ is a reflector,
\item $\mathbf{tr} P_{n,n} = n$,
\item $P_{1,m}=I_{m}$, and $P_{n,1}=I_{n}$
\item if $X \in \mathbb{R}^{n}$ and $Y \in \mathbb{R}^{m}$, then,
\begin{align}
P_{n,m} ( Y \otimes X ) = X \otimes Y
\end{align}
\item if $A \in \mathcal{M}_{n \times m}$ and $B \in \mathcal{M}_{r \times s}$, then
\begin{align}
P_{r,n} ( A \otimes B ) P_{m,s} = B \otimes A
\end{align}
\item if $A \in \mathcal{M}_{n \times n}$ and $B \in \mathcal{M}_{m \times m}$, then
\begin{align}
P_{m,n} ( A \otimes B ) P_{n,m} =  P_{m,n} ( A \otimes B ) P_{m,n}^{-1} = B \otimes A
\end{align}
Therefor, $A \otimes B$ and $B \otimes A$ are similar.
\end{itemize}
\end{proposition}

\begin{proposition}
Let $A \in \mathcal{M}_{n \times n}$ and $B \in \mathcal{M}_{m \times m}$, then we have the following properties,
\begin{itemize}
\item if $A$ and $B$ are diagonal, then $A\otimes B$ is diagonal,
\item if $A$ and $B$ are upper triangular, then $A\otimes B$ is upper triangular,
\item if $A$ and $B$ are lower triangular, then $A\otimes B$ is lower triangular,
\end{itemize}
\end{proposition}

\begin{proposition}
Let $A,C \in \mathcal{M}_{n \times m}$ and $B,D \in \mathcal{M}_{r \times s}$. If $A$ is (left equivalent, right equivalent, equivalent) to $C$, and assume that $B$ is (left equivalent, right equivalent, equivalent) to $D$. Then, $A \otimes B$ is (left equivalent, right equivalent, equivalent) to $C \otimes D$.
\end{proposition}

\begin{remark}
The use of Kronecker product preconditioners is well known \cite{vanloan2000,Langville_Stewart,Elisabeth_Ullmann,GRIGORI:2008:INRIA-00268301:5}, it is based on results of the form,
\begin{align} 
\mbox{Minimizing,}
~~~~~~~~
\phi_A(B,C) = \| A - B \otimes C  \|^2
\end{align}
for a chosen norm.
\end{remark}

\subsection{Kronecker sum}
\begin{definition}[Kronecker sum]
Let $A=(a_{ij}) \in \mathcal{M}_{n \times n}$ and $B=(b_{ij}) \in \mathcal{M}_{m \times m}$ be two matrices. The Kronecker sum of $A$ and $B$, denoted by $A \oplus B  \in \mathcal{M}_{mn \times mn}$, defines the following matrix:
\begin{align}
A \oplus B = A \otimes I_m + I_n \otimes B
\end{align}
\end{definition}

\begin{proposition}
Let $A \in \mathcal{M}_{n \times n}$ and $B \in \mathcal{M}_{m \times m}$, we have,
\begin{align}
\label{kronecker_sum_spec}
\mathbf{mspec} ( A \oplus B ) = \{ \lambda +  \mu,~~\lambda \in \mathbf{mspec}(A),~\mu \in \mathbf{mspec}(B) \}
\end{align}
\end{proposition}

\subsection{Solving $AX+XB=C$}
Let $A \in \mathcal{M}_{n \times n}$, $B \in \mathcal{M}_{m \times m}$ and $C \in \mathcal{M}_{n \times m}$. The aim of this section, is to solve the equation:
\begin{align}
\label{kronecker_eq1}
AX+XB=C
\end{align}
we can rewrite this equation in term of the Kronecker sum:
\begin{align}
(B^T \oplus A)\mathbf{vec}X =\mathbf{vec}C
\end{align}
or equivalently,
\begin{align}
G x = c
\end{align}
where,
$G = (B^T \oplus A)$, $x = \mathbf{vec}X $, and $c = \mathbf{vec}C$.
\\
Using the property \ref{kronecker_sum_spec}, we can easily check that \ref{kronecker_eq1} has a unique solution if and only if $G$ is nonsingular, \textit{i.e} $\lambda +  \mu \neq 0,~~\forall \lambda \in \mathbf{mspec}(A),~\forall \mu \in \mathbf{mspec}(B)$, which can be written in the form,
\begin{align}
\mathbf{mspec}(A) \cap \mathbf{mspec}(-B) = \emptyset
\end{align}

\begin{proposition}
If $\mathbf{mspec}(A) \cap \mathbf{mspec}(-B) = \emptyset$, then there exists a unique matrix $X \in \mathcal{M}_{n \times m}$, satisfying \ref{kronecker_eq1}. Moreover, the matrices $\left(\begin{array}{cc}
A & C
\\
0 & -B
\end{array}\right)$ and $\left(\begin{array}{cc}
A & 0
\\
0 & -B
\end{array}\right)$ are similar and verify,
\begin{align}
\left(\begin{array}{cc}
A & C
\\
0 & -B
\end{array}\right) = \left(\begin{array}{cc}
I & X
\\
0 & I
\end{array}\right)
\left(\begin{array}{cc}
A & 0
\\
0 & -B
\end{array}\right)
\left(\begin{array}{cc}
I & -X
\\
0 & I
\end{array}\right).
\end{align}
\end{proposition}

\subsection{Solving $AXB=C$}
Let $A,B,C$ and $X \in \mathcal{M}_{n \times n}$. As seen previously, using \ref{kronecker_vec_abc}, the equation 
\begin{align}
\label{kronecker_eq2}
AXB=C
\end{align}
can be written in the form,
\begin{align}
H x = c
\end{align}
where,
$H = (B^T \otimes A)$, $x = \mathbf{vec}X $, and $c = \mathbf{vec}C$.
\\
Using the property \ref{kronecker_prod_spec}, we can easily check that \ref{kronecker_eq2} has a unique solution if and only if $H$ is nonsingular, \textit{i.e} $\lambda  \mu \neq 0,~~\forall \lambda \in \mathbf{mspec}(A),~\forall \mu \in \mathbf{mspec}(B)$, which is equivalent to, $A$ and $B$ are both nonsingular.

\subsection{Solving $\sum_{i=1}^r A_i X B_i=C$}
Let $A_i,B_i,C,~~1 \leq i \leq r$ and $X \in \mathcal{M}_{n \times n}$. Using, the previous result, it is easy to show that the solution of:
\begin{align}
\label{kronecker_eq3}
\sum_{i=1}^r A_i X B_i=C
\end{align}
can be written in the form,
\begin{align}
H x = c
\end{align}
where,
$H = \sum_{i=1}^r (B_i^T \otimes A_i)$, $x = \mathbf{vec}X $, and $c = \mathbf{vec}C$.



% ...................................................................
\section{Problems}
\label{sec:linalg-kronecker-problems}
TODO





\chapter{Historical Notes}
\label{ch:fem-historical-notes}

TODO



%----------------------------------------------------------------------------------------
%	BIBLIOGRAPHY
%----------------------------------------------------------------------------------------

\chapter*{Bibliography}
\addcontentsline{toc}{chapter}{\textcolor{ocre}{Bibliography}} % Add a Bibliography heading to the table of contents

%------------------------------------------------

\section*{Articles}
\addcontentsline{toc}{section}{Articles}
\printbibliography[heading=bibempty,type=article]

%------------------------------------------------

\section*{Books}
\addcontentsline{toc}{section}{Books}
\printbibliography[heading=bibempty,type=book]

%----------------------------------------------------------------------------------------
%	INDEX
%----------------------------------------------------------------------------------------

%\cleardoublepage % Make sure the index starts on an odd (right side) page
%\phantomsection
%\setlength{\columnsep}{0.75cm} % Space between the 2 columns of the index
%\addcontentsline{toc}{chapter}{\textcolor{ocre}{Index}} % Add an Index heading to the table of contents
%\printindex % Output the index

%----------------------------------------------------------------------------------------

\end{document}
