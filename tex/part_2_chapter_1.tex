\chapter{Introduction}
\section{ Théorie de Lax-Milgram}

Dans cette section, on considère le  problème général $(LMPV)$  dans le cas $W=V$ et 

$V$ est un espce de Hilbert
$$
(LMPV) \text{ trouver } u \in V\;  \text{tel que } a(u, v)=L(v) \;\, \text{pour tout}\;\;  v \in V.
$$



(i) $a$ est une forme bilinéaire  sur  $V\times V$. 

(ii) $L :V\longrightarrow \mathbb{R}$ une application linéaire.




	
Le théorème de Lax-Milgram suivant apporte une réponse à l'existence, l'unicité et la stabilité de la solution du problème ($LMPV$) .

\begin{theorem}\
	
On suppose que  les formes $a$ et $L$ verifient les hypothèses suivantes :

1. Continuité de $L$ : $L\in V'$.

2. Continuité de $a$ : $|a(u, v)| \leq C_{a}\|u\|_V \|v\|_{V}$ pour tout $u, v \in V$.

3. Coercivité de $a$ :  $ a(u, u) \geq \alpha\|u\|_{V}^{2}$,  pour tout $u \in V$,  avec $\alpha>0$.

Alors,  il existe une solution unique $u$ du problème ($LMPV$) qui vérifie l'estimation 

$$
\|u\|_{V} \leq \frac{\|L\|_{V^{\prime}}}{\alpha}.
$$

Ce qui signifie que l'application $L\longmapsto u$ est continue par rapport à $L$.  

On dit alors que le problème ($LMPV$)  est bien-posé au sens de Hadamard.
\end{theorem}

\begin{proof}
L'estimation  s'établit en prenant $v=u$ dans ($LMPV$) puis en appliquant 

la continuité de $L$ et la coercivité de $a$ ce qui donne
$$
\alpha\|u\|_{V}^{2} \leq \|L\|_{V^\prime}\|u\|_{V}.
$$

D'où l'estimation.   Cette estimation nous donne aussi l'unicité de la solution.
\end{proof}


Pour l'existence, considérons d'abord le cas simple où $a$ est une forme symétrique. 

Dans ce cas, la continuité et la coercivité de $a$ montrent que $(u,v)\longmapsto a(u,v)$ est un  produit 

scalaire sur $V \times V$ et que la norme $\|v\|_{a}=\sqrt{a(v, v)}$ est équivalente à la norme $\|\cdot\|_{V}$. 

Puisque $L$ est continue de $V$ dans $\mathbb{R}$, elle l'est aussi par rapport a $\|\cdot\|_{a}$. 

D'où,  par le {\bf théorème de représentation de Riesz}, il existe un unique $u \in V$ tel que 

$L(v)=a(u, v)$ pour tout $v \in V$.



Dans le cas non-symétrique, on remarque que puisque $v \mapsto a(u, v)$ et $v \mapsto L(v)$ sont continues, on peut écrire, par théorème de représentation de Riesz, 
$$
a(u, v)=\langle A u, v\rangle, \qquad L(v)=\langle f, v\rangle, 
$$ 

où $A$ est un opérateur continu sur $X, f \in X$ et $\langle\cdot, \cdot\rangle$ un produit scalaire dans $V$. 

L'équation ($LMPV$) s'écrit donc $A u=f$ dans $V$. L'hypothèse de coercivité fournit  l'estimation
$$
\alpha\|v\|_{V} \leq\|A v\|_{V}
$$
pour tout $v \in V$.  Par une preuve séquentielle,  on montre que $\operatorname{Im}(A)$ est un sous-espace fermé de $V$.  Donc,    $V=\operatorname{Im}(A) \oplus(\operatorname{Im}(A))^{\perp} .$ 

Pour un  $w \in(\operatorname{Im}(A))^{\perp}$,  la coercivité montre que
$$
\alpha\|w\|_{V}^{2} \leq a(w, w)=\langle A w, w\rangle=0.
$$
Par conséquent $(\operatorname{Im}(A))^{\perp}=\{0\} $ et donc $\operatorname{Im}(A)=V$. D'où, l'existence de la solution $u$.

\begin{proposition}\label{lax}\
	
Soit $V$ un espace de Hilbert, soit $a$ une forme bilinéaire continue coercive symétrique sur $V$ et $L \in V^{\prime}$. Alors, $u$ est l'unique  solution du probleme ($LMPV$) ssi $u$ est solution du problème de minimisation suivant:
$$
(MP)\;  \left\{\begin{array}{l}
	u \in V \\
	J(u) \leq J(v), \quad \forall v\in V,
\end{array}\right.
$$

où $J$ est définie de $V$ dans $\mathbb{R}$ par :
$$
J(v)=\frac{1}{2} a(v, v)-L(v),    \quad v\in V.
$$


\end{proposition}

\begin{proof}
Soit $u \in V$ solution unique de ($LMPV$); montrons que $u$ est solution de ($MP$). 

Soit $w \in V$, on va montrer que $J(u+w) \geq J(u)$ :
$$
\begin{aligned}
	J(u+w) &=\frac{1}{2} a(u+w, u+w)-L(u+w) \\
	&=\frac{1}{2} a(u, u)+\frac{1}{2}[a(u, w)+a(w, u)]+\frac{1}{2} a(w, w)-L(u)-L(w) \\
	&=\frac{1}{2} a(u, u)+\frac{1}{2} a(w, w)+a(u, w)-L(u)-L(w) \\
	&=J(u)+\frac{1}{2} a(w, w)\geq J(u)+\frac{\alpha}{2}\|w\|^{2}.
\end{aligned}
$$
Donc $J(u+w)>J(u)$ sauf si $w=0$.
Réciproquement, supposons maintenant que $u$ est solution du problème de minimisation ($MP$)  et montrons que $u$ est solution du problème ($LMPV$).  
\end{proof}

Soit $w \in V$ et $t>0 .$ On a $$ J(u+t w)-J(u) \geq 0\qquad \text{et } \; J(u-t w)-J(u) \geq 0$$ 

car $u$ minimise $J$. On en déduit que :
$$
t a(u, w)+\frac{1}{2} t^{2} a(w, w) -tL(w)\geq 0 \text { et }-t a(u, w)+\frac{1}{2} t^{2} a(w, w) +tL(w)\geq 0
$$
Comme $t$ est strictement positif, on peut diviser ces deux inégalités par $t$ :
$$
a(u, w)-L(w)+\frac{1}{2} t a(w, w) \geq 0 \text { et }-a(u, w)+L(w)+\frac{1}{2} ta(w, w) \geq 0
$$
On fait alors tendre $t$ vers 0 et on obtient $a(u, w)-L(w)=0$ pour tout $w \in V$, ce qui montre que $u$ est bien solution $\mathrm{du}$ problème$LMPV$).  

\section{The Banach-Necas-Babuska (BNB) Theorem : inf-sup conditions}
Dans cette section, on donne un résultat plus général que le théorème de  Lax-Milgram.  Ce résultat connu sous le nom de  Banach-Necas-Babuska Theorem, ou BNB Theorem donne une  condition nécessaire et suffisante pour la résolution du problème variationel $(LMPV)$ dans le cas plus général de $V$ et $W$ des espaces de Banach. 


\begin{theorem}(Banach-Necas-Babuska)\
	
Soient  $W$ un espace de Banach  et $V$ un   Banach réflexif.  Soit  $a \in \mathcal{L}(W \times V ; \mathbb{R})$  et  $f \in V^{\prime}$. Alors, Then, problem (2.1) is well-posed if and only if:
$$
\begin{aligned}
	&(\mathrm{BNB} 1) \quad \exists \alpha>0, \quad \inf _{w \in W} \sup _{v \in V} \frac{a(w, v)}{\|w\|_{W}\|v\|_{V}} \geq \alpha \\
	&(\mathrm{BNB} 2) \quad \forall v \in V, \quad(\forall w \in W, a(w, v)=0) \Longrightarrow(v=0)
\end{aligned}
$$
Moreover, the following a priori estimate holds:
$$
\forall f \in V^{\prime}, \quad\|u\|_{W} \leq \frac{1}{\alpha}\|f\|_{V^{\prime}}.
$$

\end{theorem}


\begin{remark}\
	
	
On définit l'opérateur  $A \in \mathcal{L}\left(W ; V^{\prime}\right)$ par 
	$$
	\forall w \in W, \forall v \in V, \quad\langle A w, v\rangle_{V^{\prime}, V}=a(w, v),
	$$
	

	 
	Alors,  le problème variationnel $(LMPV)$ est équivalent   à chercher  $u \in W$ telle que  
	$$
	A u=L\quad \text{dans }\;\; V^{\prime}.
	$$ 
	
Donc, on peut voir, par des résultats de théorie d'opérateurs, que les conditions du 

Théorème BNB peut être traduite pour l'opéarteur $A$ comme suit :


	$$
	(\mathrm{BNB} 1) \Longleftrightarrow(\operatorname{Ker}(A)=\{0\}   \text{ et } \operatorname{Im}(A)  \; \text{est fermé} ) \Longleftrightarrow A^{*} \text{ est   surjectif }
	$$ 
	
	$$
	(\mathrm{BNB} 2) \Longleftrightarrow \quad\left(\operatorname{Ker}\left(A^{*}\right)=\{0\}\right) \quad \Longleftrightarrow A^{*}  \text{  est  injectif}.
	$$
	
2.  Dans le cas  de  $W=V$, on verra que la condtion de coercivité  du théorème de  Lax-Milgram implique les conditions $(BNB1)$ et $(BNB2)$. En effet, 
	
 Supposons la coercivité de $a$ et soit  $w \in V$.  La condition $(BNB1)$ découle de 
 
$$
\alpha\|w\|_{V} \leq \frac{a(w, w)}{\|w\|_{V}} \leq \sup _{v \in V} \frac{a(w, v)}{\|v\|_{V}}
$$

Soit maintenant $v \in V$.  Pour $w=v$, on a  

$$
\sup _{w \in W} a(w, v) \geq a(v, v) \geq \alpha\|v\|_{V}^{2}. 
$$

Donc, $\displaystyle \sup _{w \in W} a(w, v)=0$ implique  que $v=0$.  D'où  $(BNB2)$ est démontré.
\end{remark}

\subsection{Exemples : }

{\bf The Laplace equation :} 

Considérons l'équation aux dérivées partielles elleptique

\begin{equation}
	\begin{cases}
		-\Delta u=f \;\; \text{ in }\; \Omega\\
		u_{\mid \partial \Omega}=0.
	\end{cases}
\end{equation}

  Ce problème peut être reformulé sous la forme du problème $(LMPV)$ en  posant
$$
\left\{\begin{array}{l}
	W=V=H_{0}^{1}(\Omega) \\
	a(u, v)=\int_{\Omega} \nabla u \cdot \nabla v, 
\end{array}\right.
$$

et  $L(v)=\int_{\Omega} f v$ pour  $f \in L^{2}(\Omega)$.



{\bf The Stokes equation :}  

Considérons l'équation  elleptic Stokes PDEs   
\begin{equation}
\begin{cases}
-\Delta u+\nabla p=h, \quad \nabla \cdot u=g  \;\; \text{ in }\; \Omega\\
u_{\mid \partial \Omega}=0\\
\int_\Omega pdx=0.
\end{cases}
\end{equation}

Ce problème se met sous forme variationnelle en posant
$$
\left\{\begin{array}{l}
	W=V=\left[H_{0}^{1}(\Omega)\right]^{d} \times L_{0}^{2}(\Omega) \\
	a((u, p),(v, q))=\int_{\Omega} \nabla u:\nabla v-\int_{\Omega} p \nabla \cdot v+\int_{\Omega} q \nabla \cdot u, 
\end{array}\right.
$$

$\Omega$ est un ouvert de $\mathbb{R}^d$,   $L(v, q)=\int_{\Omega}(h \cdot v+g q)$ pour  $h\in\left[L^{2}(\Omega)\right]^{d}$ et  $g \in L^{2}(\Omega)$. 

 Ici, $L_{0}^{2}(\Omega)$ est un sous espace de $L^2(\Omega)$ de fonctions de de moyenne zéro sur  $\Omega$. 
 
 

{\bf The advection equation. }

Soit  $\beta \in\left[\mathcal{C}^{1}(\bar{\Omega})\right]^{d}$ un vecteur donné et  notons  $\partial \Omega^{-}=\{x \in \partial \Omega ;(\beta \cdot n)(x)<0\}$, 
$n$ est le vecteur normal sortant à  $\partial \Omega$. 

Considérons  l'équation d'advection  suivante : 

\begin{equation}
	\begin{cases}
	\beta \cdot \nabla u=f  \;\; \text{ in }\; \Omega\\
		u_{\mid \partial \Omega^{-}}=0. 
	\end{cases}
\end{equation}

 Ce problème se met sous forme variationnelle en posant
$$
\left\{\begin{array}{l}
	W=\left\{u \in L^{2}(\Omega) ; \beta \cdot \nabla u \in L^{2}(\Omega) ; u=0 \text { on } \partial \Omega^{-}\right\}, \quad V=L^{2}(\Omega) \\
	a(u, v)=\int_{\Omega} v(\beta \cdot \nabla u)
\end{array}\right.
$$

et  $L(v)=\int_{\Omega} f v$ pour  $f \in L^{2}(\Omega) $ et $v\in V$. Ici l'espace des solutions et l'espace test 

sont différents.

\subsection{Conditions aux bords non homogènes}

Considérons l'équation aux dérivées partielles elleptique, munie de conditions aux bords non homogènes suivante

\begin{equation}
	\begin{cases}
		-\Delta u=f \in  L^{2}(\Omega)\\
		u_{\mid \partial \Omega}=g\in H^{\frac12}(\partial \Omega), 
	\end{cases}
\end{equation}

où $H^{\frac12}(\partial \Omega)$ est l'espace de Sobolev "fractionnaire"  image de l'application trace 

$\gamma_0$ de $H^{1}(\Omega)$.
Cette équation  peut être formulée sous la forme du  problème variationnel 

"non homogène" suivant : trouver $u\in W=H^{1}(\Omega)$ telle  que 
$$
\left\{\begin{array}{l}
	a(u, v)=\int_{\Omega} \nabla u \cdot \nabla v=L(v), \quad \forall v\in V= H_{0}^{1}(\Omega)\\
	\gamma_0(u)=g\in B=H^{\frac12}(\partial \Omega),
\end{array}\right.
$$

Nous avons le résultat d'existence, d'uncité et stabilité d'un problème vartiationnelle non homogène abstrait, suivant :

\begin{proposition}\label{nonH}
Soit  $W, V$, and $B$ trois espaces de Banach , tel que $V$  est reflexif.  

Soient $\gamma_{0} \in \mathcal{L}(W ,B)$ et  $a \in \mathcal{L}(W \times V ; \mathbb{R}) .$ Supposons que  $\gamma_{0}$ est  surjective et que la  restriction de  $a$ à  $W_{0}\times V$, $W_{0} : =\operatorname{Ker}\left(\gamma_{0}\right)$,   satisfait les  conditions  BNB.  

Alors, le problème 

$$
(NHPV)\quad \left\{\begin{array}{l}
	\text { trouver  } u \in W \text { tel que} \\
	a(u, v)=L(v), \quad \forall v\in V\\
	\gamma_0(u)=g\in B,
\end{array}\right.
$$
est bien-posé et il existe  $c>0$ telle que, pour $L \in V^{\prime}$ et  $g \in B$, on a 
	$$
	\|u\|_{W} \leq c\left(\|L\|_{V^{\prime}}+\|g\|_{B}\right).
	$$
	
\end{proposition}

\begin{proof}
Puisque $\gamma_{0}$ est continue et  surjective,  alors, par Théorème de l'application ouverte, 

il  existe $c>0$ telle que,  pour tout  $g \in B$,  
il existe  $u_{g} \in W$  tel  que  

$\gamma_{0} u_{g}=g$  et $\left\|u_{g}\right\|_{W} \leq c\|g\|_{B}$. 
\end{proof}


Le problème $(NHPV)$ est équivalent à poser  $\phi=u-u_{g}$ et considérer le  problème 
$$
\left\{\begin{array}{l}
	\text { trouver  } \phi \in W_{0} \text { tel que} \\
	a(\phi, v)=L(v)-a\left(u_{g}, v\right), \quad \forall v \in V.
\end{array}\right.
$$
On a 
$$
\begin{aligned}
	\left|f(v)-a\left(u_{g}, v\right)\right| & \leq\left(\|f\|_{V^{\prime}}+\|a\|\left\|u_{g}\right\|_{W}\right)\|v\|_{V} \\
	& \leq\left(\|f\|_{V^{\prime}}+c\|a\|\|g\|_{B}\right)\|v\|_{V}.
\end{aligned}
$$

Alors, 
la forme  linéaire $L-a\left(u_{g}, \cdot\right)$ est continue sur $V$.  Le reste se déduit du Théorème BNB.
\section{Approximations de Galerkin}
Dans cette section, nous approchons le problème variationnel abstrait 
$$
(LMPV) \text{ trouver } u \in V\;  \text{tel que } a(u, v)=L(v) \;\, \text{pour tout}\;\;  v \in V
$$

par la méthode de Galerkin.

\subsection{Position du Problème}


L'dée principale des méthodes de Galerkin est de remplacer les espaces $W$ et $V$ par des espaces de dimension finie $W_{h}$ et $V_{h}$. L'espace $W_{h}$ est appelé espace de solutions ou espace d'essais, et l'espace $V_{h}$ est appelé espace de tests. On verra plus tard, comment construire de tels espaces par "les éléments finis". L'indice $h$ se référant à la taille du maillage. 


La méthode de Galerkin  suppose que  la solution  $u$ peut être  representée par une  solution  approchée :
$$
u=u_{0}+\sum_{j=1}^{N} a_{j} \phi_{j}
$$
où  les $\phi_{j}$  sont des  fonctions inconnues, $u_{0}$ est introduite  pour satsfaire les  conditions aux bourds, et les  $a_{i}$  sont des coefficients à determiner.



Ceci revient à  poser  l'espace
$$
W(h)=W+W_{h}.
$$
en supposant  qu'il est muni d'une  $\|\cdot\|_{W(h)}$ vérifiant :


(i) $\left\|w_{h}\right\|_{W(h)}=\left\|w_{h}\right\|_{W_{h}}$ pour tout  $w_{h} \in W_{h}$.

(ii) $\exists c>0: \|w\|_{W(h)} \leq c\|w\|_{W}$ pour tout  $w \in W$ ($W$ s'injecte  continuement dans  $W(h)$).

Dans sa forme générale, la méthode de Galerkin construit   une approximation 

de  la solution $u$ (des problèmes variationnels homogènes et non homogènes) 
en réslovant 

le problème approchée suivant :
$$
(PVA)\quad \left\{\begin{array}{l}
	\text { trouver  } u_{h} \in W_{h} \text { telle que  } \\
	a_{h}\left(u_{h}, v_{h}\right)=L_{h}\left(v_{h}\right), \quad \forall v_{h} \in V_{h}, 
\end{array}\right.
$$

où $a_{h}$ une approximation  de la forme bilinéaire $a$  et  $L_{h}$ une approximation de la forme  linéaire $L$ (ou $L_h-a_h(u_g, \cdot$)).

Un problème particulier de  $(PVA)$)  est,  lorsque  $W_h=V_{h}$,  est le suivant :

$$
\left\{\begin{array}{l}
	\text { Trouver  } u_{h} \in V_{h} \text { telle que  } \\
	a_{h}\left(u_{h}, v_{h}\right)=L_{h}\left(v_{h}\right), \quad \forall v_{h} \in V_{h}. 
\end{array}\right.
$$
Dans ce cas, on dit qu'on a une méthode  standard de Galerkin. Dans le cas $W_h\neq V_{h}$, la méthode est dite  méthode de Petrov-Galerkin, ou méthode  de Galerkin  
 non-standard.

Nous donnons des définitions de différentes approximations,  erreurs et convergence de la méthode.


\begin{definition}\
	
	1. (Conformité). L'approximation est dite  conforme   si  $W_{h} \subset W$  et  $V_{h} \subset V$. 
	
	Dans ce cas $W(h)=W$.   Sinon, elle est dite non-conforme.


2. (Approximabilité). L'approximation admet  la propriété d'approximabilité si 
$$
\forall w \in W, \quad \lim _{h \rightarrow 0}\left(\inf _{w_{h} \in W_{h}}\left\|w-w_{h}\right\|_{W(h)}\right)=0. 
$$

\end{definition}


\begin{definition}[Consistance  et  consistance  asymptotique]  Soit $u$ la solution du problème  $(LMPV)$.
	
(i) L'approximation est dite consistante  si  $a_{h}$ peut être étendue à $W(h) \times V_{h}$ et si la solution exacte $u$ satisfait  le problème approchée $(PVA)$, i.e., si
$$
\forall v_{h} \in V_{h}, \quad a_{h}\left(u, v_{h}\right)=L_{h}\left(v_{h}\right). 
$$

Dans le cas contraire,  l'approximation est dite non-consistante.

(ii) Si  $a_{h}$ est  uniformément  continue ( par rapport à $h$)  sur  $W_{h} \times V_{h}$, 

la méthode d'approximation est dite 
asymptotiquement  consistante s'il existe 

un  opérateur 
$\Pi_{h}: W \rightarrow W_{h}$ tel que : 


$$
\exists c>0: \quad \left\|\Pi_{h} w-w\right\|_{W(h)} \leq c \inf _{w_{h} \in W_{h}}\left\|w-w_{h}\right\|_{W(h)}, \quad \forall w \in W, 
$$ 

et 

$$
\lim _{h \rightarrow 0}\left(\sup _{v_{h} \in V_{h}} \frac{\left|L_{h}\left(v_{h}\right)-a_{h}\left(\Pi_{h} u, v_{h}\right)\right|}{\left\|v_{h}\right\|_{V_{h}}}\right)=0.
$$


L'erreur de consistance  $R_{h}(u)$ est définie par 
$$
R_{h}(u)=\sup _{v_{h} \in V_{h}} \frac{\left|L_{h}\left(v_{h}\right)-a_{h}\left(\Pi_{h} u, v_{h}\right)\right|}{\left\|v_{h}\right\|_{V_{h}}}.
$$

\end{definition}



L'une des propriétés notables des méthodes de Galerkin se trouvent dans le fait que l'erreur commise sur la solution $u$  est orthogonale aux sous-espaces d'approximation. C'est une conséquence immédiate de la consistance.


\begin{proposition}[Orthogonalité]
	
Si l'approximation est 	consistante,  on a la propriété d'orthogonalité 
$$
\forall v_{h} \in V_{h}, \quad a_{h}\left(u-u_{h}, v_{h}\right)=0. 
$$

\end{proposition}
\subsection{Etude du problème approché}

Dans cette section, on va montrer que le problème approché est bien-posé.  On distinguera le cas conforme, consistant et corecive et le cas général.


\subsubsection{Système Linéaire}
Le problème approché $(PVA)$  peut être écris sous la forme d'un système linéaire $Ax=b$. 

En effet, soit 

$$
M=\operatorname{dim} W_{h} \quad \text { et  } \quad N=\operatorname{dim} V_{h}
$$

Soient $\left\{\psi_{1}, \ldots, \psi_{M}\right\}$ une base de $W_{h}$  et  $\left\{\varphi_{1}, \ldots, \varphi_{N}\right\}$ une base  $V_{h}$.  

Ecrivons  $u_{h}$ dans  la base de  $W_{h}$,
$$
u_{h}=\sum_{i=1}^{M} U_{i} \psi_{i}.
$$

En introduisant cette formule dans le problème approché $(PVA)$ et prenant $\varphi_j$ comme fonctions tests, on obtient le système linéaire équivalent  

$$
\mathcal{A} U=b
$$
avec 
$$
\mathcal{A}_{i j}=a_{h}\left(\psi_{j}, \varphi_{i}\right), \quad 1 \leq i \leq N, 1 \leq j \leq M
$$

et  $b \in \mathbb{R}^{N}$ est le vecteur de  compoantes :
$$
b_{i}=L_{h}\left(\varphi_{i}\right), \quad 1 \leq i \leq N.
$$

D'où
$$
u_{h} \text { résout \,  }(PVA)\quad \Longleftrightarrow \quad \mathcal{A} U=b.
$$


\subsubsection{Cas conforme, consistant, corercif}
 Considérons le  of problème approché suivant 
$$
\left\{\begin{array}{l}
	\text {Trouver  } u_{h} \in V_{h} \text { telle que  } \\
	a\left(u_{h}, v_{h}\right)=L\left(v_{h}\right), \quad \forall v_{h} \in V_{h}
\end{array}\right.
$$

avec  $V_{h} \subset V$.  Notez qu'ici on a gardé les formes "continues" de l'application  bilinéaire $a$ et de forme linéaire  $L$, i.e. $a_h=a$ et $L_h=L$; donc, on est dans le cas consistant.  On a le résultat d'existence suivant :

\begin{proposition}\
	
Soient  $V$ un espace de Hilbert,  $a \in \mathcal{L}(V \times V ; \mathbb{R})$, et $L \in V^{\prime}$. 

Soit  $V_{h}$ un espace de dimension finie.

Supposons que : $a$ est  coercive sur   $V$ et  $V_{h} \subset V$.

Alors,   le  problème approché ci-dessus est bien posé. En particulier,  pour tout $L \in$ $V^{\prime}$, on a 
$$
\left\|u_{h}\right\|_{V} \leq \frac{1}{\alpha}\|L\|_{V^{\prime}}.
$$ 

\end{proposition}

\begin{proof}
Puisque  $V_{h} \subset V$, la forme bilinéaire $a$ is coercive sur  $V_{h}$ avec la même constante 

$\alpha$.  On conclut alors par le Théorème de  Lax-Milgram.
\end{proof}


\begin{remark}
Dans le cas  d'une approximation conforme et  consistante d'un  problème coercif 
($W=V$), 
la matrice $\mathcal{A}$ du système linéaire est une matrice {\bf carrée définie  positive} (Exercice).  
\end{remark}

\begin{remark}
Si  $a$ est  symétrique,  la matrice $\mathcal{A} $  est  symétrique.
\end{remark}




\subsection{Cas général : Cas  BNB}

On considère le cas général ou $W\neq V$, et  une approximation qui peut être non-conforme ou non constante.


En s'insipirant du  Théorème BNB,  pour  résoudre le problème approché $(PVA)$, on suppose 

les  conditions  discrètes:
$$
\begin{aligned}
	&\left(\mathrm{BNB} 1_{\mathrm{h}}\right) \quad  \exists \alpha_{h}>0, \quad \inf _{w_{h} \in W_{h}} \sup _{v_{h} \in V_{h}} \frac{a_{h}\left(w_{h}, v_{h}\right)}{\left\|w_{h}\right\|_{W_{h}}\left\|v_{h}\right\|_{V_{h}}} \geq \alpha_{h} \\
	&\left(\mathrm{BNB} 2_{\mathrm{h}}\right) \quad \forall v_{h} \in V_{h}, \quad\left( a_{h}\left(w_{h}, v_{h}\right), \forall w_{h} \in W_{h}=0\right) \Longrightarrow\left(v_{h}=0\right)
\end{aligned}
$$

Donnons une interprétation des conditions $\left(\mathrm{BNB} 1_{\mathrm{h}}\right)$  et  $\left(\mathrm{BNB} 2_{\mathrm{h}}\right)$ en terme de la matrice $\mathcal{A}$ 

du système linéaire.

\begin{proposition}\
	
	(i) $\left(\mathrm{BNB} 1_{\mathrm{h}}\right) \Longleftrightarrow(\operatorname{Ker}(\mathcal{A})=\{0\})$.
	
(ii) $(\mathrm{BNB} 2 \mathrm{~h}) \Longleftrightarrow\left(\operatorname{rank} \mathcal{A}=\operatorname{dim} V_{h}\right)$.

(iii) Si $\operatorname{dim} W_{h}=\operatorname{dim} V_{h},\quad \left(\mathrm{BNB} 1_{\mathrm{h}}\right) \Longleftrightarrow\left(\mathrm{BNB} 2_{\mathrm{h}}\right)$.

\end{proposition}

Par cette proposition, on peut montrer facilement le résultat,  d'existence de solution 

approchées pour le problème $(PVA)$, suivant :

\begin{theorem}\
	
Soient $V_{h}$ et  $W_{h}$  deux espaces de dimension finite munis des normes $\|\cdot\|_{W_{h}}$  et  $\|\cdot\|_{V_{h}}$.  

Supposons 

(i) $a_{h}$ est  bilinéaire continue  sur  $W_{h} \times V_{h}$ et  $L_{h}$ est  continue su  $V_{h}$.

(ii) La condition $\left(\mathrm{BNB} 1_{\mathrm{h}}\right)$ is satisfaite.

(iii) $\dim V_{h}=\dim W_{h}$.

Alors,  le problème approché $(PVA)$ est bien posé, et on a  $\left\|u_{h}\right\|_{W_{h}} \leq \frac{1}{\alpha_{h}}\left\|L_{h}\right\|_{V_{h}^{\prime}}$.
\end{theorem}

\begin{proof}
Par les hypothèses du Théorème et la Proposition précédente, la matrice $\mathcal{A}$ est carrée et inversible donc, le système linéaire admet une solution unique.
\end{proof}
\section{Analyse d'Erreur}

Dans cette section, nous dérivons des estimations de l'erreur d'approximation $u-u_{h}$, où  $u$ résout le problème  $(LMPV)$ et  $u_{h}$ solution du  problème $(PVA)$


\subsubsection{Cas général }

Supposons que :


(i) La condition $\left(\mathrm{BNB} 1_{\mathrm{h}}\right)$ est satisfaite uniformément en  $h$ et 

 $\operatorname{dim}\left(W_{h}\right)=\operatorname{dim}\left(V_{h}\right)$.
 
 
(ii) $a_{h}$ is uniformément continue en  $h$  sur $W_{h} \times V_{h}$.


(iii) L'approximation est   asymptotiquement  consistante.

(iv) L'approximation admet la propriété  d'approximabilité.

Alors, l'erreur de consistance  $R_{h}(u)$ vérifie :
$$
\left\|u-u_{h}\right\|_{W(h)} \leq \frac{1}{\alpha} R_{h}(u)+c \inf _{w_{h} \in W_{h}}\left\|u-w_{h}\right\|_{W(h)}
$$

et  $$
\lim _{h \rightarrow 0}\left\|u-u_{h}\right\|_{W(h)}=0.
$$


\subsubsection{Cas Particuliers }

{\bf Cas non-consistant, non-conforme :}


 On suppose que $a_{h}$ peut être étendue à $W(h) \times V_{h}$ tel que   $a_{h}\left(w, v_{h}\right)$ est définie pour  $w \in W$  et  $v_{h} \in V_{h}$. 
 
 Nous avons le résultat d'estimations de l'erreur suivant :
\begin{proposition} (Strang 2)\
	
	 On suppose : 
	 
(i)  La condition $\left(\mathrm{BNB} 1_{\mathrm{h}}\right)$ et   $\operatorname{dim}\left(W_{h}\right)=\operatorname{dim}\left(V_{h}\right)$.

(ii) $a_{h}$ est coninue  sur $W(h) \times V_{h}$.
Alors, 
$$
\begin{aligned}
	\left\|u-u_{h}\right\|_{W(h)} \leq &\left(1+\frac{\left\|a_{h}\right\|_{W(h), V_{h}}}{\alpha_{h}}\right) \inf _{w_{h} \in W_{h}}\left\|u-w_{h}\right\|_{W(h)} \\
	&+\frac{1}{\alpha_{h}} \sup _{v_{h} \in V_{h}} \frac{\left|f_{h}\left(v_{h}\right)-a_{h}\left(u, v_{h}\right)\right|}{\left\|v_{h}\right\|_{V_{h}}}.
\end{aligned}
$$
\end{proposition}


\begin{proof}
On utilise la formule 
$$
\begin{aligned}
	a_{h}\left(u_{h}-w_{h}, v_{h}\right) &=a_{h}\left(u_{h}-u, v_{h}\right)+a_{h}\left(u-w_{h}, v_{h}\right) \\
	&=L_{h}\left(v_{h}\right)-a_{h}\left(u, v_{h}\right)+a_{h}\left(u-w_{h}, v_{h}\right), \qquad \forall w_{h} \in W_{h},  \forall v_h\in V_h.
\end{aligned}
$$

l'hypothèse (i) et l'inégalité triangulaire des normes.
\end{proof}

{\bf Cas consistant et conforme :}

Pour une approximation  consistante et  conforme avec  $a_{h}=a$ and $L_{h}=L$, nous avons l'estimation suivante:

\begin{proposition}(Lemme de Céa). \
	
	On suppose 
	
	(i)  La condition $\left(\mathrm{BNB} 1_{\mathrm{h}}\right)$.
	
	
	(ii)  $\operatorname{dim}\left(W_{h}\right)=\operatorname{dim}\left(V_{h}\right)$.
	
	 (iii) $V_{h} \subset V$, $W_{h} \subset W, a_{h}=a$,  et  $L_{h}=L.$ 
	 
	 La solution  $u_{h}$ du problème  $(PVA)$ satisfait :
$$
\left\|u-u_{h}\right\|_{W} \leq\left(1+\frac{\|a\|_{W, V}}{\alpha}\right) \inf _{w_{h} \in W_{h}}\left\|u-w_{h}\right\|_{W}
$$

\end{proposition}
\begin{proof}
Pour  $w_{h} \in W_{h},$  l'othogonalité implique 
$$
\forall v_{h} \in V_{h}, \quad a\left(u_{h}-w_{h}, v_{h}\right)=a\left(u-w_{h}, v_{h}\right).
$$

La condition $\left(\mathrm{BNB} 1_{\mathrm{h}}\right)$, la continuité de  $a$ et l'inégalité triangulaire permet de conclure.
\end{proof}

Si en plus de la consistance, la conformité, on suppose la coercivité, alors 

 l'orthogonalité donne
$$
\forall v_{h} \in V_{h}, \quad a\left(u-u_{h}, u-u_{h}\right)=a\left(u-u_{h}, u-v_{h}\right).
$$

Par un raisonnement similaire, on obtient l'estimation plus petite suivante:

$$
\left\|u-u_{h}\right\|_{W} \leq\frac{\|a\|_{W, V}}{\alpha} \inf _{w_{h} \in W_{h}}\left\|u-w_{h}\right\|_{W}.
$$




\section{Problème de Points selles}
Dans cette section, on traitera un cas particulier du problème $(LMPV)$,  venant  de la formulation variationnelle du problème de  Stokes.  On lui  réfère  par "problème de point celle".  On  caractérisera son caratère bien-posé et analysera son  approximation par la méthode de  Galerkin.



\subsection{Position et Résolution du Problème}
On considère deux espaces de Banach réflexifs  $X$ et  $M$, $f \in X^{\prime}, g \in M^{\prime}$, et deux formes  bilinéaires $a \in \mathcal{L}(X \times X ; \mathbb{R})$ et  $b \in \mathcal{L}(X \times M ; \mathbb{R})$.  Le problème abstrait de point selle est :

$$
(PSP)\quad \left\{\begin{array}{l}
	\text {Trouver  } u \in X \text { et  } p \in M \text { tels que  } \\
	a(u, v)+b(v, p)=f(v), \quad \forall v \in X \\
	b(u, q)=g(q), \quad \forall q \in M. 
\end{array}\right.
$$


L'exemple prototype de $(PSP)$ est celui de  Stokes.  Dans ce cas,  $$X=\left[H_{0}^{1}(\Omega)\right]^{d}, \; M=L_{0}^{2}(\Omega),$$
$$
a(u, v)=\int_{\Omega} \nabla u: \nabla v,\quad  b(v, p)=-\int_{\Omega} p \nabla \cdot v, \quad f(v)=\int_{\Omega} f \cdot v, \,   g(q)=-\int_{\Omega} g q dx.
$$

pour un $f\in \left[L^{2}(\Omega)\right]^{d}$ et $g\in L^{2}(\Omega)$. 

On peut écrire le problème de point selle $(PSP)$ comme un problème 

particulier de $(LMPV)$

$$
\left\{\begin{array}{l}
	\text { Trouver  }(u, p) \in V \text { telle que } \\
	c((u, p),(v, q))=k(v, q), \quad \forall(v, q) \in V,
\end{array}\right.
$$

pour 

$$W=V=X \times M, \quad c((u, p),(v, q))=a(u, v)+b(v, p)+b(u, q),
$$
$$L(v, q)=f(v)+g(q).
$$

Il est facile de voir que les deux problèmes $(PSP)$ and $(LMPV)$ sont équivalents. Par conséquent,  une condition  necéssaire et suffisante pour que $(PSP)$ soit bien-posé est les deux conditions $(BNB1)$ et  $(BNB2)$  pour la forme bilinéaire $c$.  Cependant,  vu  sa particularité, il est possible de formuler les conditions  $(BNB1)$  et  $(BNB2)$ en terme des formes bilinéaires $a$ et $b$. Pour cela, on introduit le problème "opérateur associé" équivalent

$$
\left\{\begin{array}{l}
	\text { Trouver } u \in X \text {  et  } p \in M \text { tels que  } \\
	A u+B^{*} p=f \\
	B u=g,
\end{array}\right.
$$



 $A$ and $B$ sont les opérateurs définis par  $A: X \rightarrow X^{\prime},    \quad \langle A u, v\rangle_{X^{\prime}, X}=a(u, v)$,   
 
 $B: X \rightarrow M^{\prime}$, avec $(B v, q\rangle_{M^{\prime}, M}=b(v, q)$  et 
 $B^{*}: M=M^{\prime \prime} \rightarrow X^{\prime}$ son opérateur adjoint. 

Introduisons  le noyau de $B$ par  
$$
\operatorname{Ker}(B)=\{v \in X ; \forall q \in M, b(v, q)=0\}
$$ 

%l'opérateur $$\pi A: \operatorname{Ker}(B) \rightarrow \operatorname{Ker}(B)^{\prime}$$ 
%
% $$\langle\pi A u, v\rangle_{X^{\prime}, X}=\langle A u, v\rangle_{X^{\prime}, X}, \quad \forall u, v \in \operatorname{Ker}(B).
%$$


On a alors le résultat d'existence, unicité et stabilité du point selle suivant :


\begin{theorem}\label{pointselle}
	
	
	 Le Problème $(PSP)$ est bien posé ssi 
\begin{equation}\label{psp}
\left\{\begin{array}{l}
	\exists \alpha>0, \displaystyle \inf _{u \in \operatorname{Ker}(B)} \sup _{v \in \operatorname{Ker}(B)} \frac{a(u, v)}{\|u\|_{X}\|v\|_{X}} \geq \alpha \\
	\forall v \in \operatorname{Ker}(B), \quad( a(u, v)=0, \; \forall u \in \operatorname{Ker}(B)) \Rightarrow \; v=0
\end{array}\right.
\end{equation}

et 

\begin{equation}\label{psp1}
\exists \beta>0, \quad \inf _{q \in M} \sup _{v \in X} \frac{b(v, q)}{\|v\|_{X}\|q\|_{M}} \geq \beta.
\end{equation}

En plus, les estimations suivantes sont satisfaites

$$
\left\{\begin{array}{l}
	\|u\|_{X} \leq c_{1}\|f\|_{X^{\prime}}+c_{2}\|g\|_{M^{\prime}} \\
	\|p\|_{M} \leq c_{3}\|f\|_{X^{\prime}}+c_{4}\|g\|_{M^{\prime}}
\end{array}\right.
$$

avec  $c_{1}=\frac{1}{\alpha}, c_{2}=\frac{1}{\beta}\left(1+\frac{\|a\|}{\alpha}\right), c_{3}=\frac{1}{\beta}\left(1+\frac{\|a\|}{\alpha}\right)$, et  $c_{4}=\frac{\|a\|}{\beta^{2}}\left(1+\frac{\|a\|}{\alpha}\right)$.

\end{theorem}


\begin{remark}\
	
	
	
(i)  Si  $a$ est  coercive sur  $\operatorname{Ker}(B)$,  alors les  conditions  \eqref{psp} sont vérifées, en particulier si  $a$ est  coercive sur l'espace  $X$.




(ii)  On peut voir que les conditions \eqref{psp} et \eqref{psp1} sont satisfaites par $a$ et $b$ ssi  les  conditions  $(\mathrm{BNB} 1)$ et $(\mathrm{BNB} 2)$ sont satisfaites par la forme bilinéaire $c$. 


\end{remark}



On va  définir  maintenant le vrai sense d'un point selle de la  solution du  problème $(PSP)$. 

\begin{definition}\
	
Soient  $X$  et  $M$ deux espaces, et une application  $\mathcal{L}: X \times M \rightarrow$ $\mathbb{R}$. 

Le couple  $(u, p)$ est dit un point selle de  $\mathcal{L}$ si 
$$
\forall(v, q) \in X \times M, \quad \mathcal{L}(u, q) \leq \mathcal{L}(u, p) \leq \mathcal{L}(v, p). 
$$
\end{definition}

Le point selle est caractérisé par :

\begin{lemma}\
	
	
	Le couple  $(u, p)$ est un point selle de  $\mathcal{L}$ ssi
$$
\inf _{v \in X} \sup _{q \in M} \mathcal{L}(v, q)=\sup _{q \in M} \mathcal{L}(u, q)=\mathcal{L}(u, p)=\inf _{v \in X} \mathcal{L}(v, p)=\sup _{q \in M} \inf _{v \in X} \mathcal{L}(v, q).
$$

\end{lemma}

La proposition suivante donne la solution du problème $(PSP)$ comme point selle d'une application. La démonstration se base sur la proposition \ref{lax}. 

\begin{proposition}\
	
	
Supposons que $a$ est  symmetrique et positive.  Alors, le couple  $(u, p)$ est solution du problème  $(PSP)$  ssi $(u, p)$ est un point selle de la fonction
 Lagrangienne 
$$
\mathcal{L}(v, q)=\frac{1}{2} a(v, v)+b(v, q)-f(v)-g(q).
$$
\end{proposition}



\subsection{Approximations du problème de point selle}

Cette subsection étudie les approximations conformes au problème $(PSP)$. Soit $X_h$
un sous-espace de $X$ et soit $M_h$  un sous-espace de $M$ de dimensions finies

Considérons le problème approché :

$$
(PSPA)\quad \left\{\begin{array}{l}
	\text { Trouvet  } u_{h} \in X_{h} \text { et  } p_{h} \in M_{h} \text { tels que  } \\
	a\left(u_{h}, v_{h}\right)+b\left(v_{h}, p_{h}\right)=f\left(v_{h}\right), \quad \forall v_{h} \in X_{h} \\
	b\left(u_{h}, q_{h}\right)=g\left(q_{h}\right), \quad \forall q_{h} \in M_{h}
\end{array}\right.
$$


Soit  $B_{h}: X_{h} \rightarrow M_{h}^{\prime}$ est l'operator induit  par  $b$ tel que  $\left\langle B_{h} v_{h}, q_{h}\right\rangle_{M_{h}^{\prime}, M_{h}}=$ $b\left(v_{h}, q_{h}\right) .$

 Soit  $\operatorname{Ker}\left(B_{h}\right)$ le noyau de  $B_{h}$, i.e.,
$$
\operatorname{Ker}\left(B_{h}\right)=\left\{v_{h} \in X_{h} ; \forall q_{h} \in M_{h}, b\left(v_{h}, q_{h}\right)=0\right\}. 
$$

Etudions tout d'abord le caractère bien-posé de du prblème  approché $(PSPA)$.


\begin{proposition}
	Le Problème $(PSPA)$ est bien posé ssi 
	\begin{equation}\label{psp}
		\left\{\begin{array}{l}
			\exists \alpha_h>0, \displaystyle \inf _{u \in \operatorname{Ker}(B_h)} \sup _{v \in \operatorname{Ker}(B_h)} \frac{a(u, v)}{\|u\|_{X_h}\|v\|_{X_h}} \geq \alpha_h \\
			\forall v \in \operatorname{Ker}(B_h), \quad( a(u, v)=0, \; \forall u \in \operatorname{Ker}(B_h)) \Rightarrow \; v=0
		\end{array}\right.
	\end{equation}
	
	et 
	
	\begin{equation}\label{psp1}
		\exists \beta_h>0, \quad \inf _{q \in M_h} \sup _{v \in X_h} \frac{b(v, q)}{\|v\|_{X_h}\|q\|_{M_h}} \geq \beta_h.
	\end{equation}
	
\end{proposition}


Nous terminons ce chapitre par une estimation d'erreur de la solution du problème du point selle, similaire à celle du lemme de Céa.
  



 
\begin{proposition}\
	


Sous les conditions \eqref{psp}-\eqref{psp1}, la solution $\left(u_{h}, p_{h}\right)$ du problème $(PSP)$   

satisfait les estimations
 $$
 \begin{array}{l}
 	\left\|u-u_{h}\right\|_{X} \leq c_{1 h} \inf _{v_{h} \in X_{h}}\left\|u-v_{h}\right\|_{x}+c_{2 h} \inf _{q_{h} \in M_{h}}\left\|p-q_{h}\right\|_{M} \\
 	\left\|p-p_{h}\right\|_{M} \leq c_{3 h} \inf _{v_{h} \in X_{h}}\left\|u-v_{h}\right\|_{X}+c_{4 h} \inf _{q_{h} \in M_{h}}\left\|p-q_{h}\right\|_{M}
 \end{array}
 $$
 
 avec  $c_{1 h}=\left(1+\frac{\|a\|}{\alpha_{h}}\right)\left(1+\frac{\|b\|}{\beta_{h}}\right)$, \
 
 $c_{2 h}=\frac{\|b\|}{\alpha_{h}}$  si  
 $\operatorname{Ker}\left(B_{h}\right) \not \subset \operatorname{Ker}(B)$ et  $c_{2 h}=0$ sinon, \
 
 $c_{3 h}=c_{1 h} \frac{\|a\|}{\beta_{h}}$, et  $c_{4 h}=1+\frac{\|b\|}{\beta_{h}}+c_{2 h} \frac{\|a\|}{\beta_{h}}$.


\end{proposition}






\section{Problems}

\begin{exercise}
  TODO
\end{exercise}


