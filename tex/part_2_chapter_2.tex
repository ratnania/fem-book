\chapter{Problèmes Coercifs}
Ce chapitre traite des problèmes dont la formulation faible est dotée d'une propriété de coercitivité. 

Les exemples clés étudiés désormais sont des EDP elliptiques scalaires, i.e.,  la solution $u$ est une fonction à valeurs réelles. 
L'objectif est double : Premièrement, mettre en place
un cadre mathématique pour le caractère bien-posé de ces équations. Deuxièment, étudier les approximations  par éléments finis,  conformes et non-conformes,  basées sur les méthodes de Galerkin.

Les estimations d'erreurs sont dérivées des résultats théoriques des chapitres  précédents 
et sont illustrés numériquement. La dernière section de ce chapitre concerne la 
 perte de la  coercitivité. 
 
\section{Equations aux dérivées Partielles Scalaires}


Soit  $\Omega$ un ouvert de  $\mathbb{R}^{d}$. Considérons  l'opérateur différential   $\mathcal{L}$ 
$$
\mathcal{L} u=-\nabla \cdot(\sigma \cdot \nabla u)+\beta \cdot \nabla u+\mu u
$$

où  $\sigma$ est une fonction définie sur $\Omega$ à valeurs  dans l'ensemble des matrices  carrées d'ordre $d$,  $\beta$ et $\mu$ des   fonctions, sur $\Omega$,  à valeurs dans $\mathbb{R}^{d}$ et  $\mathbb{R}$, respectivement. 

Donnons une fonction $f: \Omega \rightarrow \mathbb{R}$, et considérons le  problème  de trouver une  fonction $u: \Omega \rightarrow \mathbb{R}$ telle que 

\begin{equation}\label{eleq}
\left\{\begin{array}{ll}
	\mathcal{L} u=f & \text { in } \Omega \\
	\mathcal{B} u=g & \text { on } \partial \Omega, 
\end{array}\right.
\end{equation}

avec  $\mathcal{B}$  un opérateur qui donne les conditions aux bords.   Ce  modèle  apparait dans plusieurs  applications :


(i) Transfert de Chaleur: 

$u$ est la temérature, $\sigma=\kappa \mathcal{I}$,  le coefficient de  conductivité de la température, $\beta$ est le champ du flux de chaleur, $\mu=0$, and $f$ est une sourche de chaleur externe exercée sur $\Omega$.



(ii) Advection-diffusion: 

$u$ est  la concentration d'une sunstance  transportée dans un champ d'écoulemen t$\beta$. La matrice $\sigma$ modélise la diffusion de la substance.  La production ou la  destruction de la substance pour une réaction chimique est modilisée par le terme  $\mu u$, et  $f$ désigne une source fixée.

Dans toute la suite,  les  hypothèses suivantes sont faites sur les données : 

$$f \in L^{2}(\Omega), \sigma \in\left[L^{\infty}(\Omega)\right]^{d, d}, \beta \in\left[L^{\infty}(\Omega)\right]^{d}, \nabla \cdot \beta \in L^{\infty}(\Omega), \;\;
\mu \in L^{\infty}(\Omega) .
$$ 


En plus, l'opérateur $\mathcal{L}$ est supposé "uniformément" elliptic dans le sense suivant : 

\begin{definition}
	
L'opérateur $\mathcal{L}$ ci-dessus est dit elliptic s'il existe $\sigma_{0}>0$ telle que 

\begin{equation}\label{elip}
\forall \xi \in \mathbb{R}^{d}, \quad \sum_{i, j=1}^{d} \sigma_{i j} \xi_{i} \xi_{j} \geq \sigma_{0}\|\xi\|_{d}^{2} \quad \text { p.p. dans  } \;  \Omega.
\end{equation}

L'équation \eqref{eleq} est dit une EDP elleptic.
\end{definition}


Exemple : Un exemple fondamental d'opérateur elleptic est :  

$\mathcal{L}=-\Delta$, obtenu pour $\sigma=\mathcal{I}, \beta=0$, et  $\mu=0$.


Dans ce cas, l'équation est dite  un problème de  Poisson.

\subsection{Formulation Variationnelle et Solutions Fortes}

Nous donnons une formulation variationnelle du probléme 

\begin{equation}\label{eleq1}
	\left\{\begin{array}{ll}
		\mathcal{L} u=f & \text { in } \Omega \\
		\mathcal{B} u=g & \text { on } \partial \Omega, 
	\end{array}\right.
\end{equation}

{\bf Condtions aux bords de Dirichlet homogènes}: 

Considérons le cas des conditions aux bords de Dirichlet homogènes, i.e.,   

$$\mathcal{B} u=u_{|\partial \Omega}=g=0.
$$ 


Multiplions  l'équation $\mathcal{L} u=f$ par une fonction test  $v\in H^1_0(\Omega)$, et intégrons  sur  $\Omega$. 

Par la formule de  Green, 

\begin{equation}\label{green}
\int_{\Omega}-\nabla \cdot(\sigma \cdot \nabla u) v=\int_{\Omega} \nabla v \cdot \sigma \cdot \nabla u-\int_{\partial \Omega} v(n \cdot \sigma \cdot \nabla u),
\end{equation}

on a  

$$
\int_{\Omega} \nabla v \cdot \sigma \cdot \nabla u+v(\beta \cdot \nabla u)+\mu u v=\int_{\Omega} f v.
$$

On obtient alors le problème variationnel suivant 

\begin{equation}\label{eqV1}
\left\{\begin{array}{l}
\text{Trouver} \; u\in H_{0}^{1}(\Omega) : \\
	a_{\sigma,\beta,\mu}(u, v):=\int_{\Omega} \nabla v \cdot \sigma \cdot \nabla u+v(\beta \cdot \nabla u)+\mu u v=\int_{\Omega} f v=L(v), \forall v\in  H_{0}^{1}(\Omega).
\end{array}\right.
\end{equation}

Une solution de ce  problème variationnel est une solution forte de l'EDP  \eqref{eleq1} :

\begin{proposition}
	
 Si $u$ est solution de  \eqref{eqV1}, alors  $\mathcal{L} u=f$ p.p. dans  $\Omega$ et  $u=0$ p.p. sur  $\partial \Omega$.
 
\end{proposition}

\begin{proof}
Soit    $u$  solution to  \eqref{eqV1}.  Donc,
$$
\begin{aligned}
	&\int_{\Omega} \sigma \cdot \nabla u \cdot \nabla \varphi =\int_{\Omega}(f-\beta \cdot \nabla u-\mu u) \varphi, \quad \forall \varphi \in \mathcal{D}(\Omega).
\end{aligned}
$$

Par définition de $H^1(\Omega)$, on a $ \sigma \cdot \nabla u\in  (H^1(\Omega))^d$ et 

$$
\nabla \cdot( \sigma \cdot \nabla u)=f-\beta \cdot \nabla u-\mu u\in L^2(\Omega),
 $$
i.e., $\mathcal{L} u=f$ dans  $L^{2}(\Omega) .$ D'où, $\mathcal{L} u=f$ p.p. dans  $\Omega$. En plus,  $u=0$ p.p. sur  $\partial \Omega$ par définition de  $H_{0}^{1}(\Omega)$.  D'où le résultat.
\end{proof}

{\bf Conditions aux bords de Dirichlet non-homgènes }:  

Considérons maintenant le cas où  $u=g$ sur  $\partial \Omega$, pour une fonction  $g\in H^{\frac12}( \partial \Omega )$ donnée. Donc, il existe  $u_{g}\in H^{1}(\Omega)$ tel que  $u_{g}=g$ sur  $\partial \Omega$.    Comme ci-dessus, nous obtenons alors la formulation variationnelle :

\begin{equation}\label{eqV2}
\left\{\begin{array}{l}
	\text { Trouver  } u \in H^{1}(\Omega) \text { telle que } \\
	u=u_{g}+\phi, \quad \phi \in H_{0}^{1}(\Omega) \\
	a_{\sigma, \beta, \mu}(\phi, v)=\int_{\Omega} f v-a_{\sigma, \beta, \mu}\left(u_{g}, v\right), \quad \forall v \in H_{0}^{1}(\Omega).
\end{array}\right.
\end{equation}

Par un raisonnement similaire, nous montrons le résultat suivant :

\begin{proposition}
	
Soit $g \in H^{\frac{1}{2}}(\partial \Omega) .$	Si $u$ est solution de  \eqref{eqV2}, alors  $\mathcal{L} u=f$ p.p. dans  $\Omega$ et  $u=g$ p.p. sur  $\partial \Omega$.
	
\end{proposition}



{\bf Conditions aux bords de Neumann }: 

Soit  $g: \partial \Omega \rightarrow \mathbb{R}$, et considérons le cas  $n \cdot \sigma \cdot \nabla u=g$ sur  $\partial \Omega$. Dans le cas  $\sigma=\mathcal{I}$, tla condition de  Neumann  fait intervenir  la dérivée normale de  $u$ puisque $n \cdot \nabla u=\partial_{n} u$. Dans le cas général, $n \cdot \sigma \cdot \nabla u$ est la dérivée conormale de $u$ par rapport à $\mathcal{L}$.  

Par  la formule de Green \eqref{green},  et la condition de Neumann, $n \cdot \sigma \cdot \nabla u=g$,  nous obtenons le problème variationnel 



\begin{equation}\label{eqv3}
	\left\{\begin{array}{l}
	\text { Trouver  } u \in H^{1}(\Omega) \text { telle que  } \\
	a_{\sigma, \beta, \mu}(u, v)=\int_{\Omega} f v+\int_{\partial \Omega} g v, \quad \forall v \in H^{1}(\Omega).
\end{array}\right.
\end{equation}

Nous avons le résultat de solution forte suivant :


\begin{proposition}
	Soit  $g \in L^{2}(\partial \Omega) .$ Si $u$ est solution de  \eqref{eqv3}, alors  $\mathcal{L} u=f$ p.p. dans  $\Omega$ et  $n \cdot \sigma \cdot \nabla u=g$ p.p. sur  $\partial \Omega$.
\end{proposition} 

\begin{proof}
Comme pour le cas Dirichlet,  considérer les fonctions tests dans  $\mathcal{D}(\Omega)$  implique que  $\mathcal{L} u=f$ p.e.,  dans  $\Omega$.
Par suite, $-\nabla \cdot(\sigma \cdot \nabla u) \in L^{2}(\Omega) .$

Par la formule de Green et le faite, par le problème variationnel et $\mathcal{L} u=f$, on trouve 

$$
\forall \phi \in H^{\frac12}(\partial \Omega), \quad\int_{\partial \Omega} (n \cdot \sigma \cdot \nabla u)\phi =\int_{\partial \Omega} g \phi.
$$

D'où, 
$n \cdot \sigma \cdot \nabla u=g$ dansd $L^{2}(\partial \Omega)$.  
\end{proof}



\textbf{Robin boundary condition :}

Soeint  $g, \gamma: \partial \Omega \rightarrow \mathbb{R}$ deux fonctions données.  On considère maintenant la condition 

aux bords  

$$\gamma u+n \cdot \sigma \cdot \nabla u=g    \text{ sur }\;\; \partial \Omega, 
$$ 

dite conditions aux bords de Robin. 

Comme au paravant, par la formule de Green, une solution de $\mathcal{L}u=f$ dans $L^2(\Omega)$ satisfait 

le problème variationnel suivant :
\begin{equation}\label{eqv4}
\left\{\begin{array}{l}
	 u \in H^{1}(\Omega) \\
	a_{\sigma, \beta, \mu}(u, v)+\int_{\partial \Omega} \gamma u v=\int_{\Omega} f v+\int_{\partial \Omega} g v, \quad \forall v \in H^{1}(\Omega).
\end{array}\right.
\end{equation}

Le résultat suivant se démontre de la même façon  que le précédent.


\begin{proposition}
	Soient  $g \in L^{2}(\partial \Omega) $ et $g \in L^{\infty}(\partial \Omega)$.  Si $u$ est solution de  \eqref{eqv4}, alors  $\mathcal{L} u=f$ p.p. dans  $\Omega$ et  $\gamma u+n \cdot \sigma \cdot \nabla u=g$ p.p. sur  $\partial \Omega$.
\end{proposition} 


\begin{remark}
	
	On peut considérer aussi  des problèmes elleptiues  avec conditions aux bords mixtes Dirichlet et Neumann, e.g.  si $\partial \Omega =\Gamma_1\cup \Gamma2$, $u=0$ sur $\Gamma_1$ et $n \cdot \sigma \cdot \nabla u=0$ sur  $\Gamma_2$? Cette EDP peut formulé comme un problème variationnelle similaire aux prcédents sur l'espace $V=\{v\in H^1(\Omega ):  u=0  \;  \text{ sur }  \; \;  \Gamma_1\}$. 
\end{remark} 
\subsection{Existence de solutions faible}

Par  solutions faibles, on désigne les solutions des problèmes variationnels de la sous-section précédente. Pour montrer leur  existence, on appliquera le Théorème de Lax-Milgram. On vérfie facilement que les applications bilinéaires associées sont continues et que les seconds membres sont des forrmes linéaires coninues. Il restera à montrer la coercivité.

Dans le théorème  suivant, nous rassemblons les résulats d'existence pour les différentes conditions aux bords.

\begin{theorem}\label{thm12}\
	
Soient $f \in L^{2}(\Omega)$,  $\sigma \in\left[L^{\infty}(\Omega)\right]^{d, d}$  tel que  \eqref{elip} est satisfaites, et  $\beta \in\left[L^{\infty}(\Omega)\right]^{d}$ 

avec  $\nabla \cdot \beta \in L^{\infty}(\Omega)$, et $\mu \in L^{\infty}(\Omega) .$ Posons 


$p=\operatorname{infess}_{x \in \Omega}\left(\mu-\frac{1}{2} \nabla \cdot \beta\right)$ et  $c_{\Omega}$ la constante de l'inégalité de  Poincaré.



(i) Les problèmes variationnels avec conditions aux bords  Dirichlet  homogènes  et  

non-homogènes sont bien posés si
$$
\sigma_{0}+\min \left(0, \frac{p}{c_{\Omega}}\right)>0
$$

(ii) Le problèmes variationnel avec conditions aux bords  Neumann est bien-posé  si

$$
p>0 \quad \text { and } \quad \operatorname{infess}_{x \in \partial \Omega}(\beta \cdot n) \geq 0.
$$

(iii) Pour $q=\operatorname{infess}_{x \in \partial \Omega}\left(\gamma+\frac{1}{2} \beta \cdot n\right)$,   le problèmes variationnel avec 

conditions aux bords  Robin est bien posé si
$$
p> 0, \quad q \geq 0.
$$

\end{theorem} 

\begin{proof}
Montrons  $(i)$.  Par la Formule de Green, voir Proposition 11, on obtient l'identité 
\begin{equation}\label{ident}
\int_{\Omega} u(\beta \cdot \nabla u)=-\frac{1}{2} \int_{\Omega}(\nabla \cdot \beta) u^{2}+\frac{1}{2} \int_{\partial \Omega}(\beta \cdot n) u^{2}
\end{equation}

pour $u\in H^1_0(\Omega)$.   Donc, par l'ellipticité  de  $\mathcal{L}$, on a 

$$
\forall u \in H_{0}^{1}(\Omega), \quad a_{\sigma, \beta, \mu}(u, u) \geq \sigma_{0}|u|_{1, \Omega}^{2}+p\|u\|_{0, \Omega}^{2}.
$$

Pour $\delta=\min \left(0, \frac{p}{c_{\Omega}}\right)$,  l'inégalité de Poincaré montre que 
$$
\forall u \in H_{0}^{1}(\Omega), \quad a_{\sigma, \beta, \mu}(u, u) \geq\left(\sigma_{0}+\frac{\delta}{c_{\Omega}}\right)|u|_{1, \Omega}^{2} \geq \alpha\|u\|_{1, \Omega}^{2}
$$

avec 
$\alpha=\sigma_{0}+\frac{\delta}{c_{\Omega}}>0$, par hypothèse.  Le résultat découle par  Lax-Milgram. 
\end{proof}


Le problème non-homogène se déduit aussi de  la Proposition  \ref{nonH}.


Pour le cas des conditions aux bords de  Neumann, on utilise aussi l'identité  \eqref{ident}, et un 

raisonnement similaire.


Pour le cas Robin, posons  $a(u, v)=a_{\sigma, \beta, \mu}(u, v)+\int_{\partial \Omega} \gamma u v .$ 

On montre que  

$$
\forall u \in H^{1}(\Omega), \quad a(u, u) \geq \sigma_{0}|u|_{1, \Omega}^{2}+p\|u\|_{0, \Omega}^{2}+q\|u\|_{0, \partial \Omega}^{2}. 
$$

Si $p>0$ et   $q \geq 0$, la forme bilinéaire est  clairement  coercive sur $H^{1}(\Omega)$, avec la constante  $\alpha=\min \left(\sigma_{0}, p\right)$. 

D'où le problème est bien posé par  le théorème de Lax-Milgram.

\subsection{Approximation}


\section{Perte de Coercivité}

\subsection{Motivation}

Soit le système linéaire 

$$
\left(\begin{matrix}
\epsilon&\beta\\
0&\epsilon
\end{matrix}
\right)\left( \begin{matrix}
u_1\\u_2
\end{matrix}\right)= \left( \begin{matrix}
b_1\\b_2
\end{matrix}\right)
$$

Pour $\epsilon=0$, la matrice n'est pas bijective et donc il n'y pas uncité de la solution. 

Mais, pour $\epsilon>0$, elle est bijective.  

Analytiquement, la solution  est unique  et elle est donnée par 
$$
\left( \begin{matrix}
	u_1\\u_2
\end{matrix}\right)= \left(\begin{matrix}
\frac1{\epsilon}&-\frac{\beta}{\epsilon}\\
0&\frac1{\epsilon}
\end{matrix}
\right) \left( \begin{matrix}
	b_1\\b_2
\end{matrix}\right).
$$

Mais, numériquement si $\epsilon$ est trop petit, la solution devient très grande.  La  matrice est mal conditionnée.  

\subsection{Position du Problème}

Considérons le problème  variationnel 

$$
\left\{\begin{array}{l}
	\text { Trouver } u \in V \text { tel que } \\
	a_{\eta}(u, v)=f(v), \quad \forall v \in V,
\end{array}\right.
$$

où  $V$ est un  Hilbert, $f \in V^{\prime}$,  et  $a_{\eta}$ est continue, coercive, sur  $V \times V$. La forme $a_{\eta}$ dépend d'un parametre    $\eta$ qui peut prendre des petites valeurs. 

Posons $\left\|a_{\eta}\right\|:=\left\|a_{\eta}\right\|_{V, V}$ et par  $\alpha_{\eta}$ la constante de  coercivité de  $a_{\eta}$, i.e.,
$$
\alpha_{\eta}=\inf _{u \in V} \frac{a_{\eta}(u, u)}{\|u\|_{V}^{2}}.
$$
\begin{definition}

On a une perte de coercivité  si

$$
\lim _{\eta \rightarrow 0} \frac{\left\|a_{\eta}\right\|}{\alpha_{\eta}}=\infty. 
$$

\end{definition}
Par analogie avec la motivation,   la perte de  coercivité ici désigne que  $a_\eta$ est mal-conditionée.

Soit  $V_{h}$ une approximation conforme de  $V$ et supposons qu'on a l'estimation, vérifiée dans le cas des éléments finis,  
$$
\forall u \in W, \quad \inf _{v_{h} \in V_{h}}\left\|u-v_{h}\right\|_{V} \leq c_{i} h^{k}\|u\|_{W}, 
$$

où  $W$ est un sous-ensemble dense dans  $V$  et  $c_i$  est une constante d'interpolation.  

Soit  $u_{h}$  la solution du problème approché:
$$
\left\{\begin{array}{l}
	\text { Trouver  } u_{h} \in V_{h} : \\
	a_{\eta}\left(u_{h}, v_{h}\right)=f\left(v_{h}\right), \quad \forall v_{h} \in V_{h}.
\end{array}\right.
$$

Si la solution exacte $u$  est dans  $W$ et vérifie cette estimation, voir Chapitre 2, 
$$
\left\|u-u_{h}\right\|_{V} \leq \frac{\left\|a_{\eta}\right\|}{\alpha_{\eta}} c_{i} h^{k}\|u\|_{W}.
$$

Si le problème présente une perte de coercivité, cette estimation  ne donne aucun contrôle pratique pour  l'erreur. Évidemment, en gardant $\eta$ fixe, quand  $h \rightarrow 0$, la convergence est atteinte.  Mais, si $\eta$ est trop petite, le $h$ ne peut pas compenser la valeur très grande de $\frac{\left\|a_{\eta}\right\|}{\alpha_{\eta}}$. 


Considérons l'exemple de l'équation  d'advection-diffusion :
$$
\begin{aligned}
	-\nabla \cdot(\epsilon \nabla u)+b\cdot \nabla u &=f \quad \text { dans  } \Omega \\
	u &=0\quad \text { sur   }  \partial\Omega, 
\end{aligned}
$$
où  $\epsilon$ est  le  coefficient de diffusion, $b$ est le champ de  vitesse,  et $f$ est le terme  source.   

On suppose $\nabla \cdot b=0$. 

Nous avons montré dans le chapitre 2 (Théorème \ref{thm12}), que la forme bilinéaire $a_\epsilon$ associée 

à ce problème est coercive avec $\alpha=\epsilon$.  

Mais si $\epsilon =0$, $a_\epsilon$ n'est pas coercive.  On peut aussi montrer que 

$\frac{\left\|a_{\epsilon}\right\|}{\alpha_{\epsilon}}\longrightarrow \infty$ quand  $\epsilon\rightarrow 0$. 










\section{Problems}

\begin{exercise}
  TODO
\end{exercise}


