% ...................................................................
\chapter{Functional Analysis}
\label{ch:fem-functional-analysis}
\section{Notations and Preliminaries}

In the course of the lecture we shall work with the Sobolev spaces $ H^m(\Omega)$, $H(\textrm{div}, \Omega)$ and $H(\textrm{div}, \Omega)$ and recall here there basic properties without proof. For a more detailed presentation with proofs we refer to Section 2.1. of \cite{BoffiBook2013}.
%
\begin{itemize}
	\item we use blodface notation for spaces of vector functions. For instance in $3D$, $\Hgradv$ denotes the space $\left( H^1(\Omega) \right)^3$. 
        \item for an operator $d \in \{\Grad, \Curl, \Div \}$, we denote the kernel space $\mathcal{N}(d)$ while the range will be $\mathcal{R}(d)$.
%        \item we restrict our study to the unit logical domain. We therefor consider $\Omega = [0,1]^d$, with $d \in \{1,2,3\}$
\end{itemize}
%
% --- Sobolev Spaces
We shall need the following Sobolev spaces, given first without boundary conditions,
%
\begin{align}
    \Hgrad  &= \left\{ \varphi \in L^2(\Omega), ~ \nabla \varphi \in \Ltwov \right\} 
    \\
    \Hgradv &= \left\{ \PsiPsi \in L^2(\Omega), ~ \nabla \PsiPsi \in \Ltwov  \right\} 
    \\
    \Hcurl  &= \left\{ \PsiPsi \in L^2(\Omega), ~ \Curl \PsiPsi \in \Ltwov  \right\} 
    \\
    \Hdiv   &= \left\{ \PsiPsi \in L^2(\Omega), ~ \Div \PsiPsi \in \Ltwo  \right\} 
%  \label{}
\end{align}
%
and using the correspondant boundary conditions,
%
\begin{align}
    \Hgradzero &= \left\{ \varphi \in \Hgrad, ~ \varphi = 0 ~ \mbox{on} ~ \partial \Omega \right\}
    \\
    \Hgradvzero &= \left\{ \PsiPsi \in \Hgradv, ~ \PsiPsi = 0 ~ \mbox{on} ~ \partial \Omega \right\}
    \\
    \Hcurlzero &= \left\{ \PsiPsi \in \Hcurl, ~ \PsiPsi \times \nn = 0 ~ \mbox{on} ~ \partial \Omega \right\} 
    \\
    \Hdivzero &= \left\{ \PsiPsi \in \Hdiv, ~ \PsiPsi \cdot \nn = 0 ~ \mbox{on} ~ \partial \Omega \right\} 
    \\
    \Ltwozero  &= \left\{ \varphi \in L^2(\Omega); ~ \int_{\Omega} \varphi = 0  \right\}
%  \label{}
\end{align}
%
Scalar-valued test functions will be denoted by $\varphi$, while $\varphi_i$ will denote a scalar basis function, after reordering all the basis functions of a given discrete space.
\\
Vector-valued functions will be written in bold, like $\uu, \vv$. $\boldsymbol{\Psi}$ will denote a vector-valued test function, while$\boldsymbol{\Psi}_i$ will denote a vector-valued basis function (after reordering the basis functions). 
\\
Even if most of what follows holds for both the $2D$ and $3D$ cases, we will restrict our studies to the $2D$ one. We recall that in $2D$, there are two \textit{curl} operators, one acting on scalars $\Rotv \phi = \left( \partial_y \phi, - \partial_x \phi \right)$ and one acting on vectors $\Rots \PsiPsi = \partial_x \PsiPsi^y - \partial_y \PsiPsi^x$. Differential operators that return a vector (grad, curl) will be written in bold ($\Grad, \Rots$). 
%
%
We also recall the Green formula for the \textit{divergence} and \textit{curl/rotational} operators
%
\begin{align}
  \boxed{
    \int_{\Omega} \left( \Div \FF \right) G = - \int_{\Omega} \FF \cdot \Grad G 
    + \int_{\partial \Omega} \left( \FF \cdot \nn \right) G, \quad \forall \FF \in \Hdiv, \forall G \in H^1(\Omega)
  }
  \label{eq:green_div}
\end{align}
%
\begin{align}
  \boxed{
    \int_{\Omega} \left( \Rotv G \right) \cdot \FF = \int_{\Omega} G \Rots \FF  
    - \int_{\partial \Omega} \left( G \times \nn \right) \cdot \FF
    , \quad \forall \FF \in \Hcurl, \forall G \in H^1(\Omega)
  }
  \label{eq:green_rot}
\end{align}
%
If $\Omega \subset \mathbb{R}^d$, $\uu = \left( u_1, u_2, \ldots, u_d \right)$ and $\PsiPsi = \left( \Psi_1, \Psi_2, \ldots, \Psi_d \right)$ , we recall the notation 
%
\begin{align}
  \int_{\Omega} \Grad \uu : \Grad \PsiPsi = \sum_{i=1}^d \int_{\Omega} \Grad u_i \cdot \Grad \Psi_i  
%  \label{}
\end{align}

\section{Sobolev spaces}
We shall denote by $\mathcal{D}(\Omega)$ the space of distribution.
\subsection{The Sobolev space $W^{s,m}$}
We start by recalling the definition of Sobolev spaces:
\begin{definition}
  Let $s$ and $m$ be two integers with $s \geq 0$ and $1 \leq m \leq \infty$. The Sobolev space $W^{s,m}(\Omega)$ is defined as
  \begin{align}
    W^{s,m}(\Omega) = \{ u \in \mathcal{D}^\prime(\Omega), ~ D^{\alpha} u \in L^m(\Omega), | \alpha | \leq s \}
    \label{eq:sobolev-space}
  \end{align}
\end{definition}
The space $W^{s,m}(\Omega)$ can be equipped with norm
\begin{align}
  \| u \|_{W^{s,m}(\Omega)} := \sum\limits_{|\alpha| \leq s} \| D^{\alpha} u \|_{L^p(\Omega)}
  \label{eq:sobolev-space-norm}
\end{align}
%
\subsection{The Sobolev space $H^m$}

For any integer $m\geq 1$, one can define
\begin{equation}\label{eq:hm}
H^m(\Omega) = W^{m,2}(\Omega) := \{  v \in L^2( \Omega) \;|\; D^\alpha v \in L^2( \Omega), ~|\alpha|\leq m  \} 
\end{equation}
$H^m(\Omega)$ is a Hilbert space equipped with the scalar product 
\begin{align}
  \left( u,v \right)_{m, \Omega} := \sum\limits_{|\alpha| \leq m} \int_{\Omega} D^\alpha u D^\alpha v 
  \label{eq:sobolev-space-scalar-product}
\end{align}
the associated norm will be denoted $\| \cdot \|_{s,\Omega}$.

The most classical  second order operator is the Laplace operator, which reads in an arbitrary dimension $d$ (generally $d=1,2$ or $3$), $$\Delta u = \sum_{i=1}^d \frac{\partial^2 u}{\partial x_d^2}.$$
The classical Green formula for the Laplace operator reads: 
for $u\in H^1(\Omega)$ and $v\in H^1(\Omega)$ 
\begin{equation}\label{eq:greenlap}
  -\int_\Omega\Delta u\, v\dd x=\int_\Omega\nabla u\cdot\nabla v\, dx -\int_{\partial\Omega} \fracp{u}{n}
v\dd \sigma.
\end{equation}


For essential boundary conditions related with this Green formula we shall define the space
$$H^1_0(\Omega) = \{ v \in H^1(\Omega) \;|\; v_{|\partial\Omega} =0 \}.$$

Another classical operator which comes from elasticity is the bilaplacian operator $\Delta^2 = \Delta \Delta$, which is a fourth order operator. The Green formula needed for variational formulations of PDEs based on the bilaplacian reads
\begin{equation}\label{eq:greenbilap}
\int_\Omega \Delta^2 u\, v \dd \mathbf{x} = \int_\Omega u\,  \Delta^2 v \dd \mathbf{x}
+ \int_{\partial\Omega} \left(   u \fracp{\Delta v}{n} - v \fracp{\Delta u}{n} + \Delta u \fracp{v}{n} -\Delta v \fracp{u}{n}  \right)\dd\sigma 
\end{equation}

$$H^2_0(\Omega) = \{ v \in H^1(\Omega) \;|\; v_{|\partial\Omega} =0, \; \fracp{v}{n}\bigl|_{\partial\Omega}=0 \}.$$

\subsection{Inequalities}
\begin{lemma}[Poincar\'e]
  Let $1 \leq p \leq \infty$ and $\Omega$ be a bounded open set. Then, there exists a constant $C=C(p, \Omega)$, such that
  \begin{align}
    \forall v \in W_0^{1,p}(\Omega), \quad C \| v \|_{L^p(\Omega)} \leq  \| \nabla v \|_{L^p(\Omega)} 
    \label{eq:poincare-inequality}
  \end{align}
%  \label{}
\end{lemma}



\section{The Sobolev space $ H(\textrm{curl}, \Omega)$}

%The Green's formula that will be useful for variational problems involving the $\textrm{curl}$ operator reads
%\begin{equation}\label{eq:greencurl}
%\int_\Omega \mathbf{u}\cdot \nabla\times \mathbf{v} = \int_\Omega \nabla\times\mathbf{u}\cdot  \mathbf{v} 
%+ \int_{\partial\Omega} (\mathbf{u}\times \mathbf{n})\cdot \mathbf{v}\dd s.
%\end{equation}

\section{The Sobolev space $ H(\textrm{div}, \Omega)$}

%\begin{equation}\label{eq:greendiv}
%\int_\Omega\nabla\cdot \mathbf{u}\, v\dd x = -\int_\Omega \mathbf{u}\cdot\nabla v\dd x + \int_{\partial\Omega} \mathbf{u}\cdot \un \, v \dd \sigma
%\end{equation}

\section{DeRham sequences}
For any function $u \in \Hgrad$ we have $\Curl \Grad u = 0$. On the other hand, we have for any function $\uu \in \Hcurl$, $\Div \Curl \uu = 0$. We just have shown that $\Grad(\Hgrad) \subset \mathcal{N}(\Curl)$ and $\Curl(\Hcurl) \subset \mathcal{N}(\Div)$. This is summarized in the following diagram, known as DeRham sequence, without boundary conditions in this case,
%
\begin{align}
  \mathbb{R} \hookrightarrow \Hgrad  \xrightarrow{\quad \Grad \quad}  \Hcurl  \xrightarrow{\quad \Curl \quad}   \Hdiv  \xrightarrow{\quad \Div \quad}  \Ltwo  \xrightarrow{\quad} 0  
  \label{eq:derham-nobc}
\end{align}
and using the correspondant boundary conditions,
\begin{align}
  \Hgradzero  \xrightarrow{\quad \Grad \quad}  \Hcurlzero  \xrightarrow{\quad \Curl \quad}   \Hdivzero  \xrightarrow{\quad \Div \quad}  \Ltwozero  \xrightarrow{\quad} 0  
  \label{eq:derham-bc}
\end{align}
%
In fact, DeRham complexes are sequences of spaces $V_i$ and operators $\diff_i$ such that $\diff_{i+1} \circ \diff_i = 0$. It leads to a sepcific algebraic structure that has been subject to active research in Analysis and Algebraic Geometry.
or in a differential forms setting
%
\begin{align}
\mathbb{R} \hookrightarrow \Lambda^0  \xrightarrow{\quad d \quad}  \Lambda^1  \xrightarrow{\quad d \quad} \Lambda^2  \xrightarrow{\quad d \quad}  \Lambda^3 \xrightarrow{\quad} 0 
\end{align}
%
where $d$ stands for the exterior derivative, while $\Lambda^k$ is the space of $k$-forms.

\begin{equation*}
\begin{array}{ccccccc}
            & \Grad &                 & \Curl &                    & \Div   &             \\
\Hgrad & \longrightarrow & \Hcurl & \longrightarrow  & \Hdiv & \longrightarrow & \Ltwo    \\
            &       &                 &       &                    &        &             \\
\Pigrad \Big\downarrow   &     & \Picurl \Big\downarrow  &   & \Pidiv \Big\downarrow &  & \Piltwo \Big\downarrow       \\
            & \Grad &                 & \Curl &                    & \Div   &             \\
\Vgrad &\longrightarrow  & \Vcurl &\longrightarrow   & \Vdiv & \longrightarrow & \Vltwo   \\
%
\\
\end{array}
\end{equation*}

\subsection{Exact discrete DeRham sequence}


%***********************************************************************


\subsection{Space decompositions}
%
More details can be found in \cite{monk_book, girault1986a}.
%
\begin{align}
	\mathcal{N}(\Curl) = \nabla \Hgradzero \oplus \left( \nabla \Hgradzero \right)^{\perp} 
%	\label{}
\end{align}
where $ \left( \nabla \Hgradzero \right)^{\perp}$ is the orthogonal of $\Hgradzero$ in $\Hcurlzero$ with respect to its inner product. We denote this space $K_N(\Omega)$, which is refered as \textit{the normal cohomology space}.  
\begin{align}
	K_N(\Omega) := \left( \nabla \Hgradzero \right)^{\perp} 
%	\label{}
\end{align}
$K_N(\Omega)$ is the space of harmonic functions which vanish on the exterior and are constant in the interior connected components of $\partial\Omega$. Note that $K_N(\Omega) \subset \Hgradvzero$.
%
\begin{theorem}
	If $\uu \in \Hcurlzero$, such that $\Curl \uu = 0$, then there exists a unique scalar potential $p \in \Hgradzero$ and a function $\ff_N \in K_N(\Omega)$ such that
	\begin{align}
		\uu = \nabla p + \ff_N
%		\label{}
	\end{align}
%	\label{}
\end{theorem}
%
\begin{theorem}[Helmoltz Decomposition]
	For every $\uu \in \Ltwov$ there exist a unique
	\begin{itemize}
		\item $p \in \Hgradzero$ 
		\item $\ff_N \in K_N(\Omega)$ 
		\item $\mathbf{A} \in \{ \ww \in \Hcurl, ~ \Div \ww = 0, ~ \nn \cdot \ww = 0 ~ \partial\Omega, ~ < \ww \cdot \nn, 1 >_{\Gamma_l} = 0  \}$ 
	\end{itemize}
	ensuring the following decomposition
	\begin{align}
		\uu = \nabla p + \Curl \mathbf{A} + \ff_N 
%		\label{}
	\end{align}
%	\label{}
\end{theorem}
%
\begin{remark}
	If $\Omega$ is homotopy equivalent to a ball, then
	\begin{align}
		\mathcal{N}(\Curl) = \nabla \Hgradzero
%		\label{}
	\end{align}
	Therefor, the Helmoltz decomposition writes
	\begin{align}
		\uu = \nabla p + \Curl \mathbf{A}
%		\label{}
	\end{align}
%	or using the function spaces
%	\begin{align}
%		\Ltwov = \nabla \Hgradzero \oplus \Curl \AA
%%		\label{}
%	\end{align}
\end{remark}
%
\begin{theorem}[Regular Decomposition of $\Hcurl$]
%
    For any $\uu \in \Hcurl$ there exists a $\vv \in \Hgradv$ such that
    \begin{enumerate}
            \item $\Curl \vv = \Curl \uu$ 
            \item $\| \vv \|_{\Ltwov} \lesssim \| \uu \|_{\Ltwov}$
            \item $\| \vv \|_{\Hgradv} \lesssim \| \Curl \uu \|_{\Ltwov}$
    \end{enumerate}
%	\label{}
\end{theorem}
%
The following result can be found in \cite{Pasciak2002a} and \cite{Zhao2002a}.
%
\begin{theorem}[Regular Decomposition of $\Hcurlzero$]
    For any $\uu \in \Hcurlzero$ there exists a $\vv \in \Hgradvzero$ such that
    \begin{enumerate}
            \item $\Curl \vv = \Curl \uu$ 
            \item $\| \vv \|_{\Ltwov} \lesssim \| \uu \|_{\Ltwov}$
            \item $\| \vv \|_{\Hgradvzero} \lesssim \| \Curl \uu \|_{\Ltwov}$
    \end{enumerate}
    \label{th:regular-decomposition-bc}
\end{theorem}
%
%%\subsection{Edge Space decompositions}
%%%
%%%
%%%\subsubsection*{The HX decomposition}
%%%
%%%
%%%\subsubsection*{The scalar HX decomposition}
%%%
%%%
%%\subsection{Hitpmair-Xu decomposition}
%%%
%%%
%%\begin{theorem}
%%%\hspace{-3cm}$\clubsuit$\hspace{3cm} 
%%  Every $\uu_h \in \Vcurl$, has a, non unique, decomposition
%%  \begin{align}
%%      \uu_h = \ww_h + \Picurl \vv_h + \nabla p_h 
%%%    \label{}
%%  \end{align}
%%  where
%%  \begin{itemize}
%%    \item $\ww_h \in \Vcurl$ 
%%    \item $\vv_h \in \Vgradv$ 
%%    \item $p_h \in \Vgrad \oplus K_{N,h}(\Omega)$ 
%%    \item $\left( h^{-2} + \mu \right) \|\ww_h\|_{\Ltwo}^2 + \| \vv_h \|_{\mu}^2 + \mu \| \nabla p_h\|_{\Ltwo}^2 \lesssim \left( \| \Curl \uu_h\|_{\Ltwo}^2 + \mu \| \uu_h\|_{\Ltwo}^2 \right) $
%%  \end{itemize}
%%  with
%%  \begin{align}
%%    \|\vv_h \|_{\mu}^2 := \| \vv_h\|_{\Hgradvzero}^2 + \mu \|\vv_h \|_{\Ltwo}^2  \quad \mbox{(HX-1)} 
%%%    \label{}
%%  \end{align}
%%  or
%%  \begin{align}
%%    \|\vv_h \|_{\mu}^2 := \| \Curl \Picurl \vv_h\|_{\Ltwo}^2 + \mu \|\Picurl \vv_h \|_{\Ltwo}^2  \quad \mbox{(HX-2)} 
%%%    \label{}
%%  \end{align}
%%%  \label{}
%%\end{theorem}
%%%
%%\begin{proof}
%%Using (Corollary \ref{cor:discrete-decomposition}), we can write $\uu_h = \Picurl \vv + \nabla p_h$. Thanks to (Lemma \ref{lemma:stable-component}), there exists a stable component $\vv_h \in \Vgrad$ of $\vv$. Now let's define $\ww_h := \Picurl \left( \vv - \vv_h \right)$, we have
%%%
%%\begin{align}
%%  \uu_h &= \Picurl \vv + \nabla p_h = \Picurl \left( \vv - \vv_h \right) + \Picurl \vv_h + \nabla p_h  
%%  \\
%%        &= \ww_h + \Picurl \vv_h + \nabla p_h  
%%  \label{}
%%\end{align}
%%\end{proof}
%%%
%%\subsection{Scalar Hitpmair-Xu decomposition}
%%%
%%Let us first introduce the component-wise operators
%%%
%%\begin{align}
%%\Picurlx v := \Picurl (v, 0, 0)  
%%\\
%%\Picurly v := \Picurl (0, v, 0)  
%%\\
%%\Picurlz v := \Picurl (0, 0, v)  
%%%  \label{}
%%\end{align}
%%%
%%The aim of this section, is to provide another version of the HX decomposition, based only on scalar projectors. We the above notations we can write
%%%
%%\begin{align}
%%  \Picurl \vv = \sum_{k=1}^{3} \Picurlk v^k, \quad \forall \vv = \left( v^1, v^2, v^3 \right)
%%%  \label{}
%%\end{align}
%%%
%%\begin{theorem}
%%\hspace{-3cm}$\clubsuit$\hspace{3cm} 
%%  Every $\uu_h \in \Vcurl$, has a, non unique, decomposition
%%  \begin{align}
%%      \uu_h = \ww_h + \sum_{k=1}^{3} \Picurlk v_h^k + \nabla p_h 
%%%    \label{}
%%  \end{align}
%%  where
%%  \begin{itemize}
%%    \item $\ww_h \in \Vcurl$ 
%%    \item $\vv_h \in \Vgradv$ 
%%    \item $p_h \in \Vgrad \oplus K_{N,h}(\Omega)$ 
%%    \item $\left( h^{-2} + \mu \right) \|\ww_h\|_{\Ltwo}^2 + \sum_{k=1}^{3} \| v_h^k \|_{k,\mu}^2 + \mu \| \nabla p_h\|_{\Ltwo}^2 \lesssim \left( \| \Curl \uu_h\|_{\Ltwo}^2 + \mu \| \uu_h\|_{\Ltwo}^2 \right) $
%%  \end{itemize}
%%  with
%%  \begin{align}
%%    \|v_h^k \|_{k,\mu}^2 := \| \Curl \Picurlk v_h^k\|_{\Ltwo}^2 + \mu \|\Picurlk v_h^k \|_{\Ltwo}^2  
%%%    \label{}
%%  \end{align}
%%%  \label{}
%%\end{theorem}
%%%
%%\begin{proof}
%%  TODO
%%\end{proof}

% ...................................................................
% ...................................................................
\section{Problems}
\label{sec:fem-functional-analysis-problems}
TODO



