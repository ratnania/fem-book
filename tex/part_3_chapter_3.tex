%\chapter{Time evolution problems}
%\label{ch:fem-mixed}
%%
%\section{Time Harmonic Maxwell problem}
%%
%The Time Harmonic Maxwell problem writes
%%
%\\
%\textit{Find } $\EE \ne \mathbf{0}$ \textit{such that} 
%\begin{align}
%  \left\{
%    \begin{array}{rll}
%      \Rotv \Rots \EE &= \omega^2 \EE & \mbox{in} ~ \Omega, 
%      \\
%      \EE \times \nn                  &= 0   & \mbox{on} ~ \partial \Omega,
%    \end{array}
%    \right.
%  \label{eq:maxwell_harmonic}
%\end{align}
%%
%Two cases occure: either $\omega$ is provided and then we only look for $\EE$. The other case is known as an eigenvalue problem, and the problem must be solved for both $\EE$ and $\omega$. 
%\\
%If we multiply by a vector-valued function $\PsiPsi$, and integrating by parts using the \textit{curl/rotational} Green formula, we get
%%
%\begin{align}
%  \int_{\Omega} \Rots \EE \cdot \Rots \PsiPsi = \omega^2 \int_{\Omega} \EE \cdot \PsiPsi
%%  \label{}
%\end{align}
%%
%Again, the space where $\EE$ lives was not mentioned; let us find what this space can be.
%\\
%We need both $\EE, \PsiPsi, \Rots \EE$ and $\Rots \PsiPsi$ to be in $L^2(\Omega)$, while satisfying the boundary condition $\EE \times \nn = 0$ on the boundary $\partial \Omega$. 
%\\
%Therefor, we need the Hilbert space $\Hcurlzero$ defined as
%%
%\begin{align}
%  \boxed{
%    \Hcurlzero = \left\{ \PsiPsi \in \Hcurl, ~ \PsiPsi \times \nn = 0 ~ \mbox{on} ~ \partial \Omega \right\} 
%  }
%%  \label{}
%\end{align}
%%
%where
%%
%\begin{align}
%  \boxed{
%    \Hcurl = \left\{ \PsiPsi \in L^2(\Omega), ~ \Rots \PsiPsi \in L^2(\Omega) \right\} 
%  }
%%  \label{}
%\end{align}
%%
%Therefor, in weak form, the problem writes
%\\
%%
%\textit{Find } $\EE \in \Hcurlzero$, $\EE \ne \mathbf{0}$ \textit{such that}, $\forall \PsiPsi \in \Hcurlzero$ 
%\begin{align}
%  \int_{\Omega} \Rots \EE \cdot \Rots \PsiPsi = \omega^2 \int_{\Omega} \EE \cdot \PsiPsi
%%  \label{}
%\end{align}
%%
%\section{Time Domain Maxwell problem}
%%
%The Maxwell equations write
%%
%\begin{align}
%  \left\{
%    \begin{array}{rl}
%      \Div \DD &= \rho
%      \\
%      \Div \BB &= 0
%      \\
%      \partial_t \BB &= - \Rots \EE
%      \\
%      \partial_t \DD &= \Rots \HH - \JJ
%    \end{array}
%    \right.
%  \label{eq:maxwell}
%\end{align}
%%
%In the case of linear, isotropic and non-dispersive materials, we have two additional relations: $\BB = \mu \HH$ and $\DD = \epsilon \EE$. Furthermore, we assume that $\JJ = \sigma \EE$. Pluging these relations in (Eq. \ref{eq:maxwell}), we get
%%
%\begin{align}
%  \left\{
%    \begin{array}{rl}
%      \Div \left( \epsilon \EE \right) &= \rho
%      \\
%      \Div \left( \mu \HH \right) &= 0
%      \\
%      \mu \partial_t \HH &= - \Rots \EE
%      \\
%      \epsilon \partial_t \EE &= \Rots \HH - \JJ
%    \end{array}
%    \right.
%  \label{eq:maxwell2}
%\end{align}
%%
%Now, let us assume that both $\epsilon$ and $\mu$ are constants and equal to $1$. We also, restrict our study to the $2D$ case and we consider the Transverse Electric (TE) mode. Therefor, the problem is only involving the variables $E_x, E_y$ and $H_z$. In this case, the Time Domain Maxwell problem writes
%%
%\begin{align}
%  \left\{
%    \begin{array}{rl}
%      \partial_t H &= - \Rots \EE
%      \\
%      \partial_t \EE &= \Rots H - \JJ
%      \\
%      \Div \EE  &= \rho
%    \end{array}
%    \right.
%  \label{eq:maxwell_te_2d}
%\end{align}
%%
%where we consider here that $\EE = (E_x, E_y)$ and $H=H_z$.
%%
%\subsection{First formulation}
%%
%The first variational formulation, in the case of perfectly conductin boundary conditions, can be derived using the \textit{curl/rotational} Green formula in Ampere's law. Therefor, the TE Maxwell problem reads
%%
%\\
%\textit{Find} $(\EE,H)\in \Hcurlzero \times L^2(\Omega)$ 
%\textit{such that} $(\PsiPsi,\varphi)\in \Hcurlzero \times L^2(\Omega)$  
%\begin{align}
%  \left\{
%    \begin{array}{rl}
%      \frac{d}{dt}\int_{\Omega} \EE \cdot \PsiPsi 
%      - \int_{\Omega} H ~ \Rots \PsiPsi &=
%      -\int_{\Omega}\JJ \cdot \PsiPsi 
%      \\
%      \frac{d}{dt}\int_{\Omega}H\varphi
%      + \int_{\Omega}(\Rots \EE)\varphi &=0,
%    \end{array}
%    \right.
%\end{align}
%%
%\subsection{Second formulation}
%%
%Using the \textit{divergence} Green formula in Faraday's law yields the second variational formulation.
%\\
%\textit{Find} $(\EE,H)\in \Hdiv \times H^1(\Omega)$
%\textit{such that} $\forall (\PsiPsi,\varphi)\in \Hdiv \times H^1(\Omega)$ 
%\begin{align}
%  \left\{
%    \begin{array}{rl}
%      \frac{d}{dt}\int_{\Omega}\EE \cdot \PsiPsi 
%      -\int_{\Omega}(\Rots H)\cdot \PsiPsi 
%      &=
%      -\int_{\Omega} \JJ \cdot \PsiPsi 
%      \\
%      \frac{d}{dt}\int_{\Omega} H \varphi 
%      + \int_{\Omega}\EE \cdot(\Rots \varphi)
%      &=0
%    \end{array}
%    \right.
%\end{align}
%
%% ...................................................................
%\section{Problems}
%\label{sec:fem-mixed-problems}
%TODO
%
%
